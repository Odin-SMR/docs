% The document should be set in 11 pt. "openright" is necessary in order
% to make chapters start on right-hand, odd-numbered pages.
\documentclass[11pt, openright]{report}


% A4 settings for convenient printing on standard paper.
\usepackage[
 papersize={210mm,297mm}, twoside, headsep=5mm,
top=25mm, bottom=25mm, outer=25mm, inner=25mm]{geometry}
\special{papersize=210mm,297mm}

% Use the following to get a double spaced version suitable for
%  writing feedback comments.
% \renewcommand{\baselinestretch}{2}

\usepackage[latin1]{inputenc}	% Possible to use �, �, �

\usepackage[dvips]{graphicx}	% Using eps figures

\usepackage[small, bf]{caption}	% Customised captions

\usepackage{srcltx}		% Jump from xdvi to emacs

\usepackage{cite} 		% Write [11-15] instead of [11, 12,
				% 13, 14, 15]

%\usepackage{drftcite}		% Use this to show bibtex labels
				% instead of number. This is good when
				% editing your inserted references

\usepackage[dvips]{color}

%\usepackage{psfrag}             % This is for use with Laprint

\usepackage{longtable}		% For tables in need of page break,
				% included for the list of abbreviations

\usepackage{amsmath}

% Headers and footers %----------------------------------------------
\usepackage{fancyhdr}
% Define "localfancy" to be used here, see manual for fancyhdr called
% "Page layout in LaTeX"
\fancypagestyle{localfancy}{
\fancyhead{}% Clear
\fancyhead[LE]{\small \leftmark}
\fancyhead[RO]{\small \rightmark}
\fancyfoot{}
\fancyfoot[LE,RO]{\thepage}
\renewcommand{\headrulewidth}{0.4pt}
\renewcommand{\footrulewidth}{0pt}}
\pagestyle{localfancy}
\renewcommand{\chaptermark}[1]{\markboth{\chaptername\ \thechapter.\ #1}{}}
\renewcommand{\sectionmark}[1]{\markright{\thesection.\ #1}}
% Redefine plain pagestyle
\fancypagestyle{plain}{
\fancyhead{}% Clear
\fancyfoot{}
\fancyfoot[LE,RO]{\thepage}
\renewcommand{\headrulewidth}{0pt}
\renewcommand{\footrulewidth}{0pt}}
\newcommand{\be}{\begin{equation}}
\newcommand{\ee}{\end{equation}}
% end Headers and footers %------------------------------------------

% Write "References" instead of "Bibliography"
\renewcommand\bibname{References}

% Define ToDo notes
\usepackage{todonotes} 
\setlength{\marginparwidth}{5cm}
\newcommand{\comment}[2]{\todo[color=yellow!30,linecolor=black]{\textbf{{#1} says: }{#2}}}
\newcommand{\lcomment}[2]{\todo[inline,color=yellow!30,linecolor=black]{\textbf{{#1} says: }{#2}}}
\newcommand{\fixme}[2]{\todo[color=orange!40]{\textbf{FIXME({#1}): }{#2}}}
\newcommand{\lfixme}[2]{\todo[inline,color=orange!40]{\textbf{FIXME({#1}): }{#2}}}
\newcommand{\snippet}[1]{\textcolor{black!20}{#1}}
\newcommand{\addref}{\todo[color=green!40]{Add a reference.}}
\newcommand{\addrefs}{\todo[color=green!40]{Add references.}}


% Includes personal command definitions
%\input{cmd}

\begin{document} %---------------------------------------------------
\pagestyle{plain} % Use plain style until "Introduction"
\thispagestyle{empty}
\begin{center}
\large
\LARGE
\textsc{Odin-SMR \\ Algorithm Theoretical Basis Document}\\[20mm]
\LARGE
\large
Department of Earth and Space Sciences\\
Chalmers University of Technology\\
Goteborg, Sweden, 2015
\end{center}

\newpage

%\newpage
%\pagenumbering{roman}
%\setcounter{page}{3}
%\input{abstract.tex}

\tableofcontents
%\input{acknowledgements.tex}


%\newpage

\chapter{Aim and Scope of this Document}
\label{chapter:intro}

% The page numbering must be reset here inside the file
\pagenumbering{arabic}
\setcounter{page}{1}

The Sub-Millimetre Radiometer (SMR), aboard the Odin satellite, has been in operation
since 2001 and performs passive limb measurements of the atmosphere, mainly at 
frequencies around 500\,GHz. From these measurements, profiles of species that are of
interest for studying stratospheric and mesospheric chemistry and dynamics can be
derived, such as \chem{O_3}, \chem{ClO}, \chem{N_{2}O}, \chem{NO}, \chem{HNO_{3}},
\chem{H_{2}O}, \chem{CO}, and isotopologues of \chem{H_{2}O}, and \chem{O_{3}}.
These profiles are referred to as Level2 (L2) products of \smr. They are generated by
a Level2 processor. The input to the Level2 processor is geolocated and calibrated
measurements (\smr\ Level1B data, described in \citet{atbdl1b}) and dynamic and
static auxilliary/ancilliary data.

The aim of this document is to provide an user-friendly description of the L2 data.
More details about the L2 processing system and the algorithms applied by the L2
processor are available in the Input/Output Data Definition Document (IODD) \citep{iodd}
and in the Algorithms Theoretical Basis Document (ATBD) - Level 2 processing \citep{atbdl2},
respectively. 
  


%\newpage 

\chapter{Notations}
\label{chapter:notations}
\pagestyle{localfancy}
\pagenumbering{arabic}


\section{Definition of common terms}

\lcomment{PE}{Don't follow, please, give one example.}


\section{Abbreviations} 

\lcomment{PE}{Please, set up a table structure and give one example.}



%%% Local Variables: 
%%% mode: latex
%%% TeX-master: "main"
%%% End: 


%\newpage 

\chapter{Overview}
\label{chapter:overview}


\section{Level 1 Processing}

\section{Level 2 Processing}




%%% Local Variables: 
%%% mode: latex
%%% TeX-master: "L2_ATBD"
%%% End: 


\chapter{Level 1 Algorithm}
\label{chapter:L1algorithms}
\pagestyle{localfancy}
\pagenumbering{arabic}




%%% Local Variables: 
%%% mode: latex
%%% TeX-master: "main"
%%% End: 


\chapter{Level 2 Algorithm}
\label{chapter:L2algorithms}
\pagestyle{localfancy}
\pagenumbering{arabic}

\section{Optimal Estimation Method}

\subsection{Physics of the Problem}

Text describing the physics of this particular algorithm...

\textbf{ Input Data:}
\begin{itemize}
\item Number of chickens
\item Temperature in the coop
\item Number of foxes present
\end{itemize}


\textbf{Output Data:}
\begin{itemize}
\item Omelett
\item Chicken pie
\end{itemize}


\subsection{Mathematical Description of the Algorithm}

\begin{enumerate}
\item In order to verticaly displace the yellow of the egg into the frying pan the shell must be removed using a laser incident on the chicken while in free fall inside a vacuum. The terminal speed due to the tastefield of the egg is approximated using 

\begin{equation}
A=Bx
\end{equation}


where 

$A$ \hspace{1cm} [m/s] is the terminal speed of the chicken \\
$B$ \hspace{1cm} [-] is the number of eggs \\
$x$ \hspace{1cm} [m$^2$/kg] the taste coefficient of the egg \\


\item Step 2

\item Step 3



\end{enumerate}



%%% Local Variables: 
%%% mode: latex
%%% TeX-master: "main"
%%% End: 


%\bibliographystyle{plain}	% A standard citation style
%\bibliographystyle{unsrt}	% A standard citation style
\bibliographystyle{ieeetr}	% IEEE citation style
%\bibliographystyle{verbose}	% Local verbose citation style
%\bibliographystyle{thesis}	% Local citation style for use in thesis
%\bibliographystyle{PoPstyle}	% PoP citation style thesis

\bibliography{references}

\newpage \
\end{document}
