\chapter{\smr\ Level2 algorithms and input data}
\section{Overview}
The basic target of the Level2 algorithms is to
convert observed and calibrated Level1B spectra
to atmospheric species profiles.
%The mapping between an observed set of spectra (a scan)
%and the underlying atmospheric species profile is not unique.
The \smr\ level2 data products will be derived by an
optimal estimation method (OEM) implementation, as described
in the ATBD for Level2 processing. 
The OEM implementation combines measurement information with 
Ancillary -- Auxiliary data and applies a forward 
model. Requirements for each component is described in the
following sections.

\section{Forward model requirements}

Forward model requirements are described in Sect. 2.2
and 2.4.2 in the ATBD for level2 processing.
In short, these requirements are

\begin{itemize}


\item limb sounding requires that refraction and the spherical shape
of the Earth is considered when determining the propagation path,
which excludes all plane-parallel models 

\item the forward model must be able to deliver Jacobians

\item the forward model must be able to incorporate effects of
sensor characteristic (antenna angular response, sideband filtering,
frequency response of each spectrometer channel) 

\end{itemize}

ARTS, the Atmospheric radiative transfer simulator, 
is a publicly available software that is both developed and 
maintained as an open source project.
ARTS is in compliance with the requirements for \smr\ Level2
data processing. Thus, no further refinements of ARTS must be performed.



\section{OEM implementation requirements}

\lcomment{BR}{How can the requirements be formulated?}

\section{Level1B data requirements}

\lcomment{BR}{How can the requirements be formulated?}

\section{Ancillary -- Auxiliary data requirements}
\subsection{Climatological \textit{a priori} data}
\textit{a priori} data, or a climatology database covering all species of interest
are required, as the OEM implementation needs a starting guess for each profile
to be retrived. 
The climatology must cover latitudinal and seasonal variations.


\subsection{Weather data?}
Temperature, pressure, altitude. Data from the same source
should be available for the complete Odin mission.
Data should contain no known artificial trend.

\subsection{Sensor characteristic data}  
antenna angular response, sideband filtering,
frequency response of each spectrometer channel.

\lcomment{BR}{How can the requirements be formulated?}


