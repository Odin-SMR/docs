\chapter{\smr\ Level2 algorithms and input data}
\section{Overview}
The basic target of the Level2 algorithms is to
convert observed and calibrated Level1B spectra
to atmospheric species profiles.
%The mapping between an observed set of spectra (a scan)
%and the underlying atmospheric species profile is not unique.
The \smr\ level2 data products will be derived by an
optimal estimation method (OEM) implementation, as described
in the ATBD for Level2 processing. 
The OEM implementation combines measurement information with 
Ancillary -- Auxiliary data and applies a forward 
model. Requirements for each component is described in the
following sections.

\section{Forward model requirements}

Forward model requirements are described in Sect. 2.2
and 2.4.2 in the ATBD for level2 processing.
In short, these requirements are

\begin{itemize}


\item limb sounding requires that refraction and the spherical shape
of the Earth is considered when determining the propagation path,
which excludes all plane-parallel models 

\item the forward model must be able to deliver Jacobians

\item the forward model must be able to incorporate effects of
sensor characteristic (antenna angular response, sideband filtering,
frequency response of each spectrometer channel) 

\end{itemize}

ARTS, the Atmospheric radiative transfer simulator, 
is a publicly available software that is both developed and 
maintained as an open source project.
ARTS is in compliance with the requirements for \smr\ Level2
data processing. Thus, no further refinements of ARTS must be performed.


\subsection{Spectroscopic data}

Spectroscopic data can be seen as basic input to the forward model,  
and it is of high importance that this data is as correct as possible. 
The position and strength of molecular transitions 
are known with a relatively high accuracy. In contrast, pressure 
broadening and non-resonant absorption are less well known.
Spectroscopic parameters can for instance be found in the 
high-resolution transmission molecular absorption (HITRAN) database.
However, a requirement is that a literature review is performed,
in order to identify the best possible spectroscopic data to apply. 



\section{OEM implementation requirements}

\lcomment{BR}{How can the requirements be formulated?}

\section{Level1B data requirements}

The accuracy of the Level1B dataset is important for the quality of the
Level2 dataset for obvious reasons. \smr\ has been in operation
for \(\sim\)14 years, and thus, the Level2 dataset can be applied for
scientifically interesting trend analysis. It is therefore of
high importance that the Level1B dataset is accurate and stable.

Best available \smr\ Level1B data should be applied.
The Level1B data must not contain any errors that are known.


\section{Ancillary -- Auxiliary data requirements}
\subsection{Climatological \textit{a priori} data}
\textit{a priori} data, or a climatology database covering all species of interest
are required, as the OEM implementation needs a starting guess for each profile
to be retrived. 
The climatology must also cover the relevant vertical range for each species, and
latitudinal and seasonal variations.
Climatlogy data must have an accuracy better than ? \(\%\).

A requirement is that a literature review is performed,
in order to identify the best possible climatological data.


\subsection{Weather data}
External weather data (here pressure, temperature, altitude) is given as input
to the Level2 processing. Data from the selected source must be available
for the complete Odin mission. Furthermore, the data should be as accurate
as possible (typically temperature errors less than 2\,K?), 
and contain no known artificial trend. 

A commonly used data source, for this type of application, is 
European Centre for Medium-Range Weather Forecasts (ECMWF).
In principle, there are two possible ECMWF products to choose
between, i.e. data from the operational model or from   
ERA-Interim. ERA-Interim is a global atmospheric reanalysis from 1979,
and is based on the same version of assimilation system,
whereas the operational product has changed assimilation
version during the Odin mission. 
 
Thus, a requirment is that temperature trend analsysis is performed 
for candidate data sources, in order to verify that the data contain
no spurious trend, that can propagate to an artificial trend in
\smr\ Level2 data.
 

\subsection{Sensor characteristic data}  
The forward model simulation must take into account of sensor characteristic data
such as the antenna angular response, sideband filtering, and frequency response 
of each spectrometer channel. A requirement is that best possible
data is applied. 



