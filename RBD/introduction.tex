\chapter{Introduction}
\label{chapter:introduction}


% The page numbering must be reset here inside the file
\pagenumbering{arabic}
\setcounter{page}{1}


\section{Aim and scope of this document}
\label{sec:aim}
\smr\ performs passive limb measurements of the atmosphere,
mainly at wavelengths and frequencies around 0.6\,mm and 500\,GHz,
respectively.
From these measurements, profiles of 
\chem{O_3}, \chem{ClO}, \chem{N_{2}O}, \chem{HNO_{3}}, 
\chem{H_{2}O}, \chem{CO}, and isotopologues of \chem{H_{2}O}, and \chem{O_{3}},
that are species of interest for studying stratospheric and 
mesospheric chemistry and dynamics, can be derived. 
\smr\ has been in operation for approximately 14 years, and thus, the Level2
dataset can potentially be applied for scientifically interesting trend analysis.
The current Level2 data version is Version 2-0 (for the 544 GHz band)
and 2-1 (for all other data). 
Following various validation exercises a number of issues have been identified. 
In particular various artefacts were noted in the level 1 data used, 
including strange baselines, glitches and jumps in the power level. 
It has become clear that many of these are caused by the implementation 
of the calibration scheme, in particular in how the load spectra were selected.
The main aim of the project is to reprocess all \smr\ data in order to 
create a fully consistent and homogeneous dataset of stratospheric species profiles. 
This should be done using the latest versions of the calibration scheme and settings 
for the inversion algorithm, and revised ancillary and auxiliary data.

The aim of this document is to define \smr\ Level2 products
to be generated, and then establish a structured set of 
individual high-level requirements for
\begin{itemize}
\item the products to be generated,
\item the algorithms to be implemented either in the form of refinements to existing
  algorithms or as newly developed alternative algorithms,
\item all input data required to generate the data products,
\item methods and sources to be used to validate the products.
\end{itemize}

\section{Document structure}

This document is organized as follows:
Chapter 2 describes the \smr\ Level2 data products.
Chapter 3 describes the requirements for
the \smr Level2 algorithms and input data.
Chapter 4 describes requirements for validation
of \smr data products.


