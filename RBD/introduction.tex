\chapter{Introduction}
\label{chapter:introduction}


% The page numbering must be reset here inside the file
\pagenumbering{arabic}
\setcounter{page}{1}


\section{Aim and scope of this document}
\label{sec:aim}
\smr\ performs passive limb measurements of the atmosphere,
mainly at wavelengths and frequencies around 0.6\,mm and 500\,GHz,
respectively.
From these measurements, profiles of 
\chem{O_3}, \chem{ClO}, \chem{N_{2}O}, \chem{HNO_{3}}, 
\chem{H_{2}O}, \chem{CO}, and isotopologues of \chem{H_{2}O}, and \chem{O_{3}},
that are species that are of interest for studying stratospheric and 
mesospheric chemistry and dynamics, can be derived. 
\smr\ has been in operation for \(\sim\)14 years, and thus, the Level2 
dataset can potentially be applied for scientifically interesting trend analysis.

The aim of this document is to define \smr\ Level2 products
to be generated, and then establish a structured set of 
individual high-level requirements for
\begin{itemize}
\item the products to be generated,
\item the algorithms to be implemented either in the form of refinements to existing
  algorithms or as newly developed alternative algorithms,
\item all input data required to generate the data products,
\item methods and sources to be used to validate the products.
\end{itemize}

\section{Document structure}
  
This document is organized as following:
Chapter 2 describes the \smr\ Level2 data products.
Chapter 3 describes the requirements for
the \smr Level2 algorithms and input data.
Chapter 4 describes requirements for validation
of \smr data products.


