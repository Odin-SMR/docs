\chapter{Data validation}
\label{validation}

\section{Validation objective and method}

The basic idea of validation is to check that a product  
meets its specification. An ideal validation exercise
would be to compare \smr\ Level2 data profiles to the truth,
and to evaluate if the differences between \smr\ and the 
truth is within the specified errors. Furthermore, validation can 
be seen as an iterative process: validation results are used to 
trace systematic errors in the instrument and the applied 
processing algorithms, and thus to improve Level1 and Level2 products.

The problem is that the true trace gas species concentration,
within the \smr\ field of view in the upper atmosphere, is seldom known. 
Thus, a true validation is complicated. The best one can do is therefore
to compare \smr\ Level2 data to best possible correlative datasets,
and to derive a mean bias against various reference datasets.

The "validation" must be performed in at least two phases.
The first phase is validation for a diagnostic/verification dataset, i.e. a
sub-part of the \smr\ dataset. This dataset will be used to judge the
performance of the updated processing chain, and to allow continuous
improvements to the processing chain during development.
The dataset should be representative in the sense that it
contains all types of data (both "good" and "problematic") while
not being overly large (less than 10\% of the complete dataset).
The second phase is to perform validation for the finally selected processing
setup and for the complete dataset.


\section{Level1B verification}

The quality of the Level1B dataset should be verified.
A minimum requirement is that calibrated spectra should 
be compared with radiative transfer model simulations where 
the observed signal can be accurately predicted. 
In practice, this means observations with low tangent point 
at high latitudes and with low tropospheric temperature gradient. 
Furthermore, simulations should be based on the best possible input data.
Time-series of measurements and simulations should be evaluated
in order to identify/verify the calibration accuracy and any drifts. 


\section{Level2 validation}

%The situataion is a bit different for the various species.
All main Level2 data species must be compared to correlative datasets
from other satellite instruments, such as Microwave Limb Sounder (MLS)
on the Aura satellite, and the Michelson Interferometer for Passive
Atmospheric Sounding (MIPAS), a mid-infrared emission spectrometer
mounted on the European ENVIronmantal SATellite (ENVISAT).
Furthermore, \smr ozone profiles can be compared to high quality balloon-borne 
measurements, and such comparisons are presented in \citep{jegou:techn:08}.
However, ozone profiles derived from balloon-borne 
measurements do not cover the upper atmosphere (typically maximum
altitude is below 35\,km).

A minimum requirement for the \smr\ Level2 products is that mean bias
profiles against reference datasets are established. 
The following reference datasets must be considered:
\begin{itemize}

\item the previous \smr\ Level2 dataset for \chem{O_3}, \chem{ClO}, \chem{N_{2}O}, \chem{HNO_{3}},

\item best possible correlative dataset derived from satellite-borne measurements
      for \chem{O_3}, \chem{ClO}, \chem{N_{2}O}, and \chem{HNO_{3}},

\item ballon-borne measurements (between 20--35\,km) for \chem{O_3}.

\end{itemize}



 
