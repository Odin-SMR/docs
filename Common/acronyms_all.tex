

\def\LM{LM}
\def\LMlong{Levenberg\,-\,Marquardt}

\def\OEM{OEM}
\def\OEMlong{Optimal estimation method}

\def\SMR{SMR}
\def\SMRlong{Sub-millimetre radiometer}


