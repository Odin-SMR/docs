\chapter{Introduction}
\label{chapter:introduction}


% The page numbering must be reset here inside the file
\pagenumbering{arabic}
\setcounter{page}{1}


\section{Aim and scope of this document}
\label{sec:aim}
\smr\ performs passive limb measurements of the atmosphere,
mainly at wavelengths and frequencies around 0.6\,mm and 500\,GHz,
respectively.
From these measurements, profiles of 
\chem{O_3}, \chem{ClO}, \chem{N_{2}O}, \chem{HNO_{3}}, 
\chem{H_{2}O}, \chem{CO}, and isotopologues of \chem{H_{2}O}, and \chem{O_{3}},
that are species that are of interest for studying stratospheric and 
mesospheric chemistry and dynamics, can be derived. 
\smr\ has been in operation for approximately 18 years, and thus, the Level2
dataset can potentially be applied for scientifically interesting trend analysis.

A new \smr\ Level2 product dataset has been generated, and in~\cite{dds}
thew new \smr\ Level2 products are compared to correlative measurements from other
instruments, as well as with the Level2 data from the older 2.0/2.1 versions of
the Odin/SMR processing chain. A general description
of the \smr\ retrieval process is described in~\cite{atbdl2}. That is, 
\cite{atbdl2} describes the basic configuration of the forward
model deployed, the optimal estimation method implementation, and the retrieval
variables for the \smr\ processing. In practise, the exact configuration
of the \smr\ level2 processing is product or frequency mode specific,
and all detailed settings are not reported in \cite{atbdl2}.

The aim of this report is mainly to document the configuration of \smr\ Level2 processing
in details. The information has been collected from the QSMR-Data repository
(\url{https://phabricator.molflow.com/diffusion/QQD/}), that is actually used
by the \smr\ Level2 processing system. 
The information presented in this document is relevant for future validation studies,
as it for example should allow for a reproduction of the results.
 

\section{Document structure}

Chapter 2 defines the configuration of the \smr\ Level2
processing, and the actual setting deployed for the various
frequency modes.  
