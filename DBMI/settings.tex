\chapter{Level2 processing configuration}

\section{Configuration by frequency mode}
The \smr\ Level2 processing is controlled by a tailored matlab software
package. The retrieval settings for the various modes of \smr\
are defined by two matlab functions:
 
\begin{itemize}
\item[{Q\_STND}]
  handles frequency mode 1, 2, 8, and 17
\item[{Q\_MESO}]
  handles frequency mode 13, 14, 19, 21, 22, and 24.
\end{itemize}

A description of the configuration parameters are found
in Sect.~\ref{sec:config}, and the actual values
of the parameters for the standard and mesospheric modes
are described in Sect.~\ref{sec:standard} and Sect.~\ref{sec:meso},
respectively.

\subsection{Configuration description}
\label{sec:config}

This section contains a brief description of the settings considered by the
Qsmr retrieval system. These settings are packed into a structure denoted
as Q. This structure must contain an exact set of fields; all fields must be
present and no additional ones are allowed. The defined fields are described
below, in alphabetical order.

Qsmr operates also with two other structures. For the various
pre-calculations a structure denoted as P is used. The fields of P are
defined and are shortly described in the file p\_std.m. The structure R works
as repository for internal variables and data. That is, no fields of R are
set by the user.

Most fields of Q are of simple types such scalar value, vector or string,
but some fields are structures. This more complex type is only used for
retrieval quantities, in order to allow simple communication with some
functions copied from Qarts.

All fields listed affect the inversions, either in the pre-calculation phase
or when doing the actual inversion. The settings are applied by various
functions and if you are using individual functions of Qsmr you need to
figure out what settings that have an effect or not. For example, the fields
TB\_SCALING\_FAC and TB\_CONTRAST\_FAC are not applied directly when loading the
data, but are applied on the L1B data by a special function.

\begin{changemargin}{2.5cm}{0.0cm} 

\begin{itemize}
\item[{ABS\_P\_INTERP\_ORDER}]
An integer. The polynomial order to apply for pressure interpolation of the
absorption look-up table. See further the ARTS workspace variable with the
same name.

\item[{ABS\_SPECIES}]
An array of structures. Each array element provides settings for a gas
species. The fields of the structure are as follows. TAG: Definition of the
species following the ARTS format, e.g. O3-*-501e9-503e9. SOURCE: A string
describing from where temperature a priori shall be taken. Handled options
are 'WebApi' and 'Bdx'. RETRIEVE: A boolean, flagging if the species shall
be retrieved or not. All fields below are ignored if RETRIEVE is false. L2: A
boolean, flagging if the species is part of the L2 data of the frequency mode.
GRID: Retrieval grid to use for the species. UNC\_REL and UNC\_ABS: Minimum
relative and absolute uncertainty (1 std dev), respectively. The absolute
and relative values are compared using the a priori profile and the largest
of the two is selected (but not exceeding 1e3 in relative value). CORRLEN:
Correlation length, in meter, to use when creating Sx. LOG\_ON: Set to true
to impose a positive constrain for the species. L2NAME: A string. The name
to give the retrieved product. ISOFAC: A scalar. The isotopologue fraction
assumed in ARTS. If the complete species VMR is the
output, this field shall be set to 1.

Note that when creating L2 data the outermost points of GRID are removed.

\item[{ABS\_T\_INTERP\_ORDER}]
An integer. The polynomial order to apply for temperature interpolation of the
absorption look-up table. See further the ARTS workspace variable with the
same name.

\item[{ARTS}]
A string. Name or path of the ARTS executable to use.

\item[{BACKEND\_NR}]
An integer. Index of expected backend. Index coding described in L1B ATBD.

\item[{BACKEND\_FILE}]
Empty or a string. If empty, the name of file to read is generated
automatically following the channel seperation. If you don't want to use
these default files, you put the name of the alternative file in this field.

\item[{BASELINE}]
A structure. Definition of baseline off-set retrieval. The fields of the
structure are as follows. RETRIEVE: A boolean, flagging if baseline off-set
shall be retrieved or not. UNC: A priori uncertainty (1 std dev). MODEL:
A string. If set to 'common' a single off-set is retrieved for each
spectrum. If set to 'module' an off-set for each active AC module is
retrieved, i.e. up to four off-set per spectrum are derived. If set to
'adaptive' then all modules contributing with more than 125 channels are
grouped and a common off-set for these modules is retrieved, while separate
off-sets are retrieved for remaining modules (<= 125 channels).

\item[{DZA\_GRID\_EDGES}]
A vector. Complements DZA\_MAX\_IN\_CORE in the specification of the angular
grid used for pencil beam calculations. The vector specifies the values to
add outside the lower and upper boresight direction. These are relative angles
(in degrees), where 0 shall not be included.

\item[{DZA\_MAX\_IN\_CORE}]
A scalar value. Determines the maximum spacing (in degrees) of the angular
grid used for pencil beam calculations. This value sets the spacing between
the lower and upper boresight direction.

\item[{FOLDER\_ABSLOOKUP}] 
A string. Full path to folder containing the different versions of absorption
look-up tables. That is, this folder is expected to contain folders. The
exact folder specification is a result of this field and ABSLOOKUP\_OPTION.

\item[{FOLDER\_ANTENNA}] 
A string. Full path to folder containing antenna pattern response files.

\item[{FOLDER\_BACKEND}] 
A string. Full path to folder containing backend channel response files.

\item[{FOLDER\_BDX}] 
A string. Full path to folder containing files of the Bordeaux a priori
database. Files having .mat format are expected.

\item[{FOLDER\_FGRID}] 
A string. Full path to folder containing frequency grid files.

\item[{FOLDER\_MSIS90}] 
A string. Full path to folder holding the MSIS90 climatology (version taken
from arts-xml-data). Only needed of temperature data taken from MSIS90,
which is the case for some pre-calculations.

\item[{FOLDER\_WORK}] 
A string. Full path to a folder where temporary files and/or folders can
be placed. If this field is set to '/tmp', a temporary folder is created and
all files are placed in this folder, and the folder is removed when the
calculations are done. Otherwise, temporary files are placed directly in the
specified folder, and these are left when the calculations are done. This
option is useful for debugging, but note that just a single Qsmr process can
use a folder for debugging. If several Qsmr processes are given the same debugging
folder, files will be overwritten and the calculations will crash or be incorrect.

\item[{FREQMODE}] 
An integer. The frequency mode. See L1B ATBD for definition of existing
frequency modes.

\item[{FREQUENCY}] 
A structure. Definition of frequency off-set retrieval. The fields of the
structure are as follows. RETRIEVE: A boolean, flagging if frequency off-set
shall be retrieved or not. UNC: A priori uncertainty (1 std dev). NPOLY: The
polynomial order to apply for frequency fit. For example, 0 means that a
constant off-set is retrived and 1 means that the varies is assumed to vary
linearly with zenith angle. The value -1 has a special meaning. With -1, an
off-set for each tangent altitude is retrieved. The a priori uncertainty is
set to following UNC for all polynomial coefficients, or for each off-set
of NPOLY is -1.

\item[{FRONTEND\_NR}] 
An integer. Index of expected frontend. Index coding described in L1B ATBD.

\item[{F\_RANGES}] 
A matrix, having two columns. This matrix specifies the frequency ranges to
include in the retrieval, where the first and second column give the lower
and upper frequency limit, respectively. Each row specifies a new frequency
range to include.

\item[{F\_GRID\_NFILL}] 
An integer. If set to > 0, the sensor response matrix will include a cubic
frequency interpolation of the spectra, with F\_GRID\_NFILL points added
between existing grid points. See further the ARTS workspace method
sensor\_responseFillFgrid. If set to 0, no such interpolation is made.

\item[{F\_LO\_NOMINAL}] 
A scalar value. Nominal value of the LO frequency.

\item[{GA\_FACTOR\_NOT\_OK}] 
A scalar value. The factor with which the Marquardt-Levenberg factor is
increased when not a lower cost value is obtained. This starts a new
sub-iteration. This value must be > 1.

\item[{GA\_FACTOR\_OK}] 
A scalar value. The factor with which the Marquardt-Levenberg factor is
decreased after a lower cost values has been reached. This value must be > 1.

\item[{GA\_MAX}] 
A scalar value. Maximum value for gamma factor for the Marquardt-Levenberg
method. The inversion is halted and flagged as unsuccessful if this value is
reached. This value must be > 0.

\item[{GA\_START}] 
A scalar value. Start value for gamma factor for the Marquardt-Levenberg
method. See the L2 ATBD for a definition of the gamma factor. This value must
be >= 0.

\item[{INVEMODE}] 
A string. A short string naming the inversion set-up used.

\item[{LO\_COMMON}] 
A boolean. If true, the initial value of LO frequencies are set to be
constant over the scan. This value is set following LO\_ZREF. If false, the
L1B value for each altitude is used.


\item[{LO\_ZREF}] 
A scalar value. Reference altitude for LO frequency. When performing
frequency cropping, frequencies are taken from the spectra with the closest
altitude. Further, if LO\_COMMON is set to true, the LO frequency is taken
from the L1B data of the spectrum closest to this altitude.

\item[{MIN\_N\_FREQS}] 
A scalar value. The required number of frequencies (i.e. channels) of spectra
to start an inversion. This number refers to the number of spectra after frequency
cropping and quality filtering.

\item[{MIN\_N\_SPECTRA}] 
A scalar value. The required number of spectra of a scan to start an
inversion. This number refers to the number of spectra after altitude
cropping and quality filtering.

\item[{NOISE\_CORRMODEL}] 
A string. Model of correlations inside Se. Only correlation between adjacent
channels of each spectrum is modelled. The options are as follows. 'none':
this generates a pure diagonal Se. 'empi': Uses empirically derived values
making Se a five-diagonal matrix. 'expo': Exponentially decreasing
correlation, approximating the empirically derived values.

\item[{POINTING}] 
A structure. Definition of pointing off-set retrieval. The fields of the
structure are as follows. RETRIEVE: A boolean, flagging if pointing off-set
shall be retrieved or not. UNC: A priori uncertainty (1 std dev).

\item[{PPATH\_LMAX}] 
A scalar value. The maximum distance between points of the propagation path.
See further the ARTS workspace variable with the same name.

\item[{PPATH\_LRAYTRACE}] 
A scalar value. The length to apply for ray tracing to consider the effect
of refraction. See further the ARTS workspace variable with the same name.

\item[{P\_GRID}] 
A vector. The pressure grid to be used. See further the ARTS workspace
variable with the same name. Note that this setting is also used when
pre-calculating absorption lookup tables.

\item[{QFILT\_FCORR}] 
A logical. So far only used for CO modes. Set to true to remove data were
the frequency correction failed or is uncertain.

\item[{QFILT\_LAG0MAX}] 
A logical. Sets the maximum allowed value of ZeroLagVar. This quality
filtering operates on AC sub-bands.

\item[{QFILT\_MOON}] 
A logical. Determines if data shall be filtered based on the MOON quality
flag. This quality filtering operates on tangent altitudes.

\item[{QFILT\_NOISE}] 
A logical. Determines if data shall be filtered based on the NOISE quality
flag. This quality filtering operates on tangent altitudes.

\item[{QFILT\_REF1}] 
A logical. Determines if data shall be filtered based on the REF1 quality
flag. This quality filtering operates on tangent altitudes.

\item[{QFILT\_REF2}] 
A logical. Determines if data shall be filtered based on the REF2 quality
flag. This quality filtering operates on tangent altitudes.

\item[{QFILT\_SCANNING}] 
A logical. Determines if data shall be filtered based on the SCANNING quality
flag. This quality filtering operates on tangent altitudes.

\item[{QFILT\_SPECTRA}] 
A logical. Determines if data shall be filtered based on the SPECTRA quality
flag. This quality filtering operates on tangent altitudes.

\item[{QFILT\_TBRANGE}] 
A logical. Determines if data shall be filtered based on the TB range quality
flag. This quality filtering operates on tangent altitudes.

\item[{QFILT\_TINT}] 
A logical. Determines if data shall be filtered based on the TINT quality
flag. This quality filtering operates on tangent altitudes.

\item[{QFILT\_TREC}] 
A logical. Determines if data shall be filtered based on the TREC quality
flag. This quality filtering operates on tangent altitudes.

\item[{QFILT\_TSPILL}] 
A logical. Determines if data shall be filtered based on the TSPILL quality
flag. This quality filtering operates on tangent altitudes.

\item[{SIDEBAND\_LEAKAGE}] 
A scalar or 'model'. If a scalar value, this is taken as the sideband
leakage. This leakage is assumed to be flat over the (main) frequency band.
If set to 'model', the sideband response is set according a model based on
Tcal and SBPATH.

\item[{STOP\_DX}] 
OEM stop criterion. The iteration is halted when the change in x
is < stop\_dx. Eq. 5.29 in the book by Rodgers is followed, but a
normalisation with the length of x is applied. This means that STOP\_DX
should in general be in the order of 0.01 (and not change with the
length of the state vector).

\item[{REFRACTION\_DO}] 
A boolean. Determines if refraction is considered or not by the forward
model. Set to true to include refraction.

\item[{T}] 
A structure. Definition of atmospheric temperature profile. The fields of
the structure are as follows. SOURCE: A string describing from where
temperature a priori shall be taken. Handled options are 'WebApi' and
'MSIS90'. RETRIEVE: A boolean, flagging if temperature shall be retrieved or
not. All fields below are ignored if RETRIEVE is false. L2: A boolean,
flagging if temperature is part of L2 data of the frequency mode. GRID:
Retrieval grid to use for temperature. UNC: A vector of length 5, with a
priori uncertainty (1 std dev)  at 100, 10, 1, 0.1 and 0.01 hPa (roughly
16, 32, 48, 64 and 80 km). CORRLEN: Correlation length, in meter, to use
when creating Sx. LIMITS: A vector of length 2, specifying allowed limits
for retrieved temperatures. The first and second value is the lower and
upper limit, respectively. L2NAME: A string. Will be used as L2.Product.

\item[{TB\_CONTRAST\_FAC}] 
A scalar value. This factor modifies the contrast of each spectrum part.
If this factor is denoted as c, the scaling is:
Tb\_new = c * ( Tb -Tb\_min ) + Tb\_min,
where Tb\_min as an estimate of the noise-free minimum value of each
spectrum part. This scaling is applied after TB\_SCALING\_FAC. This contrast
scaling is applied on each AC module separately. That is, the complete
spectrum is divided into four individual parts when performing this scaling.
To leave the data unchanged, set this field to {[}{]} or 1.

\item[{TB\_SCALING\_FAC}] 
A scalar value. The L1B brightness temperature data are scaled with this
factor. If this factor is denoted as c, the scaling is Tb\_new = c * Tb.
For example setting this field to 1.005 will convert an original  Tb-value
of 200 K to 201 K. To leave the data unchanged, set this field to {[}{]} or 1.

\item[{VERSION\_ARTS}] 
A string. This string shall match the version string provided by the
expected version of ARTS. For example: 'arts-2.3.562'

\item[{VERSION\_QSMR}] 
A string. This string shall match the version string found at the top of
Qsmr's ChangeLOg file. The version is expcted to be placed on line 3 and be
proceeded with a '*'.

\item[{ZTAN\_LIMIT\_BOT}] 
A vector of length 4. The lower limit for tangent altitudes to include in
the inversion. That is, this setting determines the lower limit when
cropping the scan range. The four values give the tangent altitude limit at
0, +-30, +-60 and +-90 degrees in latitude. That is, the tangent altitude
mask is assumed to be symmetric around the equator.

\item[{ZTAN\_LIMIT\_TOP}] 
A scalar value. The upper limit for tangent altitudes to include in the
inversion. That is, this setting determines the upper limit when cropping
the scan range.

\item[{ZTAN\_MIN\_RANGE}] 
A vector of length two. This field specifies the minimum altitude coverage of a
scan to start an inversion. The order between lower and upper limit is free.
The scan must have at least one tangent altude below and above the given
limits. This check is done after applying ZTAN\_LIMIT\_BOT/TOP.

\end{itemize}

\end{changemargin}

\clearpage
\newpage
\subsection{Q standard}
\label{sec:standard}
\verbinput{q_stnd.m}
\clearpage
\newpage

\subsection{Q meso}
\label{sec:meso}
\verbinput{q_meso.m}

\clearpage
\newpage
\section{Spectroscopic data and absorption continua}
\label{sec:spectroscopy}

The bulk of the data are taken from the latest HITRAN 
(high-resolution transmission molecular absorption)
database (HITRAN2016).
For the most important transitions for \smr\ some hand-picked
data are used to replace the corresponding value in HITRAN,
and this is described in Sect.~\ref{sec:linedata}.
Nitrogen and water vapour continua models applied
are described in Sect.~\ref{sec:continua}.


\subsection{Nitrogen and water vapour continua}
\label{sec:continua}
\verbinput{continua_std.arts}

\subsection{\smr\ linedata}
\verbinput{h2o_notes.txt}
\label{sec:linedata}
\verbinput{smr_linedata.notes}

