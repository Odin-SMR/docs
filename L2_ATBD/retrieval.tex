\chapter{Retrieval variables}
\label{chapter:retvars}

This chapter reviews the variables that are part of the retrievals, either
directly or indirectly. The presentation follows the formalism presented in
Sec.~\ref{sec:formalism}.


\section{Basic variable choices}
\label{sec:retr:basics}
%
\smr\ performs measurements in several frequency bands simultaneously
(Sec.~\ref{sec:smr:details}). The set of frequency bands differs between observation
modes\addref. The altitude range scanned by \smr\ is not fixed, and a a \smr\
``scan'' is simply defined as the spectra recorded between two adjacent turning
points in altitude of the limb scanning\addref. The retrievals covered by this
\ATBD\ operate only on one scan, from one frequency band, at the time. Each
combination of scan and frequency band is inverted separately, no information
is carried between different retrievals.

There is no strict division line between thye state and forwamd model parameter
vectors (\SttVct\ and \FrwMdlVct). Obviously, \SttVct\ shall contain a
representation of the quantities targeted by the measurements, that for \smr\
are gas species concentrations and atmospheric temperatures.\todo{Finish thsi
  discussion.}



\section{The measurement vector, \MsrVct}
\label{sec:y}
%
The measurement vector contains solely \smr\ brightness temperature values,
with spectra from different tangent altitudes appended. That is,
\begin{equation}
  \label{eq:y}
  \MsrVct =
  \begin{bmatrix}
    \MsrVct_1\\ 
    \MsrVct_2\\ 
    \vdots\\ 
    \MsrVct_s
  \end{bmatrix},
\end{equation}
where $\MsrVct_1$/$\MsrVct_x$ is the spectrum from the highest/lowest tangent
altitude etc. 

As mentioned, each retrieval does only contain data from a single scan, but
not necessary the complete scan. For each mode an upper and lower tangent
altitude limit is defined, that determines which spectra of a scan that will be
part of \MsrVct. The altitude limits are defined using geometrical tangent
altitudes (that is, refraction neglected) as those are the tangent altitudes
found in the L1b data.

The tangent altitude range is allowed to vary as a function of latitude, an
example is given in Fig.~?\todo{Make figure}. These tangent altitude masks are
throughout defined at 90\degree S, \dots\todo{Add values}\ and 90\degree N,
with linear interpolation applied between these latitudes. The altitude limits
for each frequency band are found ?\todo{Where?}.



\section{Observation uncertainty covariance matrix, \aCvrMtr{o}}
\label{sec:So}
%
\lcomment{PE}{Write this section later, as details are not yet clear. Can the
  L1b data be used to calculate thermal noise level? Exactly what expression to
  use? Do we consider the noise correlation between channels? With Hanning the
  correlation is significant, while without Hanning a pure diagonal matrix is a
  better approximation.}


\section{The state vector, \SttVct}
\label{sec:x}
%
The state vector contains a mix of quantities. Besides the 118\,GHz band, the
quantities always present are: gas species profiles $(\VctStl{v})$, the
temperature profile $(\VctStl{t})$, a set of brightness temeprature off-sets
$(\Delta\VctStl{o})$, a pointing off-set $(\Delta\theta)$ and a frequency
off-set $(\Delta\nu)$\todo{Correct?}. With just these quantities retrieved, and
a case involving two gas species, the state vector becomes:
\begin{equation}
  \label{eq:measvec}
  \SttVct =
  \begin{bmatrix}
    \VctStl{v}_1\\ 
    \VctStl{v}_2\\ 
    \VctStl{t}\\ 
    \VctStl{\Delta\VctStl{o}}\\ 
    \Delta\theta\\ 
    \Delta\nu 
  \end{bmatrix}.
\end{equation}
For 118\,GHz the state vector contains no species profiles as only oxygen,
having a known VMR, has to be considered.

A species or temperature ``profile'' is a set of values describing the vertical
variation, at a set of specified pressures, that is, the retrieval grid (see
also Sec.~\ref{sec:arts:aspects}). Such retrieval grids are discussed further
in Sec.~\ref{sec:b:pgrid}, together with the basic pressure grid of the forward
model. Temperatures are given in Kelvin.
\\
\lcomment{PE}{We go for log-retrievals for species? Discuss basic aspects of
  the log conversion here.}

As discussed in Sec.~\ref{sec:smr:errors}, there is not a perfect compensation
for gain variations, causing an off-set in brightness temperatures. This
off-set has been found to be flat over the frequency bands, but differ from one
tangent altitude to next. The vector \VctStl{\Delta\VctStl{o}} contains one
brightness temeprature off-set for each included spectrum, to compensate for
the ``baseline shift''. That is, a correction of the brightness temperatures is
estimated, and corrected spectra can be expreessed as
\begin{equation}
  \label{eq:retr:baseline}
  \hat{T}_b^{ij} = T_b^{ij} + \Delta\VctStl{o}_i
\end{equation}
where $T_b^{ij}$ is brightness temperature oc channel $j$ and tangent altitude
$i$ and $\Delta\VctStl{o}_i$ is element $i$ of $\Delta\VctStl{o}$.


The pointing and frequency off-sets are scalar values. This implies that these
correction factors are valid for the complete scan. In \ARTS\ the viewing
direction, of each \LOS, is specified as the angle from zenith. These angles can
not be determined perfectly, but the error has been judged to be constant
during a scan and a single correction value should be sufficient. Hence, the
retrieved zenith angles are
\begin{equation}
  \label{eq:retr:point}
  \hat{\theta_i} = \theta_i + \Delta\theta,  
\end{equation}
where $\theta_i$ is the L1b zenith angle of tangent altitude $i$. A positive 
$\Delta\theta$ means that the retrieval estimated lower tangent altitudes than
obtained by the attitude data. As a rule of thumb, $\Delta\theta=\pm0.01\degree$
matches about $\mp550$\,m in tangent altitude.

There is a similar uncertainty in the exact frequencies observed. This
uncertainty originates in the \LO-signal. Already pre-launch tests hows that
\LO-frequency varies somewhat with temperature. A correction table was created,
but some frequency error remains and there seems also to be some ageing on the
\LO-chain affecting the final frequencies. The final frequency uncertainty is
totally correlated between the backend channels inside each frequency band. As
long as the \LO-signal is phase-locked, it is judged that the frequency off-set
is constant during a limb scan. However, the phase-locking of the 118 and
575\,GHz front-ends have been failing basically throughout the mission, and the
assumption of a constant frequency off-set is more uncertain for these receiver
chains. The switch from version 1 to 2 of \ARTS\ enables the retrieval of a
non-constant frequency off-set, but this possibility will just be actived later
if found necessary. The retrieved frequency of channel $i$ is
\begin{equation}
  \label{eq:retr:freq}
  \hat{\nu_j} = \nu_j + \Delta\nu,  
\end{equation}
where $\nu_j$ is the frequency found given by L1b data.




\section{The a priori vector, \aSttVct{a}}
\label{sec:xa}
%
\dots


\section{A priori uncertainty covariance matrix, \aCvrMtr{a}}
\label{sec:Sa}
%
\dots



\section{Forward model parameters, \FrwMdlVct}
\label{sec:b}
%
Please note that some/all\todo{Should it be some or all here?}\ static forward
model settings, that are common for all simulations, are introduced already in
Sec.~\ref{sec:arts:config}.

\subsection{Pressure grids}
\label{sec:b:pgrid}
%
\dots (also retrieval grids)


\subsection{Spectroscopic data and absorption continua}
\label{sec:b:absdata}
%
\dots


\subsection{Absorption look-up tables}
\label{sec:b:abstable}
%
\dots



\subsection{Radiative transfer}
\label{sec:b:rt}
%
The propagation path used when evaluating Eq.~\ref{eq:rteq} is defined by a set
of points. A point is added at each crossing of the path crosses with a
pressure levels (that is, the pressures defining the vertical grid). Additional
points are inserted to ensure that the distance along the path does not exceed
an user defined threshold. This threshold is denoted $l_\mathrm{max}$ in the
tables of ?\todo{WHere?}.
\\
\lcomment{PE}{What about $I_0$?. Is this a static variable? That is, hEq.~\ref{eq:rteq}ow do we
  handle moon in LOS?}




%%% Local Variables: 
%%% mode: latex
%%% TeX-master: "L2_ATBD"
%%% End: 
