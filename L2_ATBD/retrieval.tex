\chapter{Retrieval set-up}
\label{chapter:retrieval}

\section{The optimal estimation method}
\label{sec:oem}
%
The optimal estimation method (\OEM) is introduced in
Sec.~\ref{sec:setup:inverse}, while details more specific for the \smr\ L2
processing are treated in this section.

\subsection{Implementation}
\label{sec:oem.m}
%
\lcomment{PE}{How and where is the complete data flow described? Shall this
  sub-section be placed elsewhere?}
%
The \smr\ retrievals are performed by a Matlab implementation of \OEM, found as
part of the Atmlab package. This package is avaliable through the \ARTS\ site,
at \url{www.radiativetransfer.org/tools}. The implementation of \OEM\ in Atmlab
\texttt{oem.m}, aims at being generic, that is, it shall be possible to couple
the function to different forward models. The Atmlab package contains the
required functions to interface \texttt{oem.m} with \ARTS. This functionality
has been used to create the second version of the Qpack
\citep{eriksson:qpack:05} inversion system. This version, Qpack2, is used by
several groups operating ground-based microwave radiometers
\citep[e.g.][]{acp-15-5099-2015}. The function \texttt{oem.m} was coupled to a lidar
forward model by \citet{sica2015retrieval}.

To be clear, earlier L2 processing was based on the Qpack system (version
1), but this is not longer the case to allow an optimisation for \smr. Instead,
the core functionality, on which Qpack2 is based, is used to set up a retrieval
system targeting the needs raised by \smr. The main components taken from
Atmlab are \texttt{oem.m}, functions to read/write \ARTS\ output/input files,
functions to interpolate data of climatology character (denoted as ``atmdata''
inside Atmlab) and functions to create parametric variance-covariance matrices.
All these functions are also used by Qpack2 and, hence, are well tested.


\subsection{Iteration scheme}
\label{sec:ml}
%
The \smr\ processing offers a non-linear retrieval problem and an iterative
procedure is required to determine the solution (Eq.~\ref{eq:mincost}). A
number of iteration schemes has been suggested. The simplest version is
Gauss-Newton \citep[see][Sec.~?]{rodgers:00}. This iteration scheme assumes a
robust decrease of the cost function (\CstFnc, Eq.~\ref{eq:costfun}) during the
iterations, but this is seldom the case in practical retrievals and
Gauss-Newton can result in iterations where the cost instead is increasing. The
standard choice among iteration approaches, including a manner to enforce a
decrease of the cost function at each step, is the Levenberg\,-\,Marquardt 
method (LM), and it is also selected for these retrievals.

\LM\ operates with a parameter $\gamma$. With $\gamma=0$, \LM\ becomes equal to
Gauss-Newton iteration. Each iteration step is then ?. For large values
$\gamma$, \LM\ instead behaves as a steepest descent method. This results in a
relatively small change in \SttVct, and then also a slow convergence rate, but
the change in \SttVct\ results more likely in a decrease of \CstFnc. See
\citep[see][Sec.~?]{rodgers:00} for a more detailed presentation of \LM.

There exists different versions of \LM. The version applied here is
\begin{equation}
  \label{eq:ml}
  \RtrVct_c = \RtrVct_i + 
  \left[ (1+\gamma)\aCvrMtr{a}^{-1} + 
          \aWfnMtrTrp{i}\aCvrMtr{o}^{-1}\aWfnMtr{i} \right]^{-1}
  \left[ \aWfnMtrTrp{i}\aCvrMtr{o}^{-1}(\MsrVct-\FrwMdl(\RtrVct_i,\FrwMdlVctHat)) -
         \aCvrMtr{a}^{-1}(\RtrVct_i-\aSttVct{a})\right]
\end{equation}
where $\RtrVct_t$ is the candidate solution for iteration $i+1$, $\RtrVct_i$ is
the (final) solution after iteration $i$, and \aWfnMtr{i}\ is the Jacobian
(\aWfnMtr{\SttVct}) using $(\RtrVct_i,\FrwMdlVctHat)$ as linearisation point.
Other variables are introduced in Sec.~\ref{sec:formalism}. The iteration is
started by setting $\RtrVct_0 = \aSttVct{a}$, and $\RtrVct_c$ becomes
$\RtrVct_{i+1}$ if $\CstFnc(\RtrVct_c) < \CstFnc(\RtrVct_i)$.

The scheme for updating $\gamma$, that must consider both succesful
$(\CstFnc(\RtrVct_c) < \CstFnc(\RtrVct_i))$ and
unsuccesful $(\CstFnc(\RtrVct_c) \geq \CstFnc(\RtrVct_i))$ iterations, is:
\begin{itemize}
\item[0] $\gamma$ is set to a start value, $\gamma=\gamma_0$.
\item[1] Calculate $\RtrVct_c$ by Eq.~\ref{eq:ml}.
\item[2] If $\CstFnc(\RtrVct_c) < \CstFnc(\RtrVct_i)$:
  \begin{itemize}
  \item[2a] Set $\RtrVct_{i+1}=\RtrVct_c$.
  \item[2b] If convergence has reached (see Sec.~\ref{sec:conv}), set
    $\RtrVct=\RtrVct_{i+1}$ and jump to 4.
  \item[2c] Decrease $\gamma$ with a factor $f_s$. That is, $\gamma$ is updated
    as $\gamma\leftarrow\gamma/f_s$. 
  \item[2d] If $\gamma$ becomes smaller than a lower
    treshold value, $\gamma<\gamma_\mathrm{min}$, set $\gamma=0$.
  \item[2e] Continue iterations by jumping to 1.
  \end{itemize}
\item[3] If $\CstFnc(\RtrVct_c) \geq \CstFnc(\RtrVct_i)$:
  \begin{itemize}
  \item[3a] Increase $\gamma$ with a factor $f_u$. That is, $\gamma$ is updated
    as $\gamma\leftarrow f_u\cdot\gamma$.
  \item[3b] If $\gamma$ becomes larger than a upper treshold value,
    $\gamma>\gamma_\mathrm{max}$, set $\RtrVct=\RtrVct_i$ and jump to 4.
  \item[3c] Re-do the iteration with new $\gamma$ by moving to 1.
  \end{itemize}
\item[4] Perform retrieval characterisation
  (Chapter~\ref{chapter:characterisation}), using $\aWfnMtr{\SttVct}=\aWfnMtr{i}$.
\end{itemize}



\subsection{Convergence criteria}
\label{sec:conv}
%

\begin{equation}
  \label{eq:stopcrit}
  \left( \RtrVct_{i+1} - \RtrVct_i \right)^T
  \left( \aCvrMtr{a}^{-1} + 
         \aWfnMtrTrp{i}\aCvrMtr{o}^{-1}\aWfnMtr{i} \right)
  \left( \RtrVct_{i+1} - \RtrVct_i \right) < \Delta\SttVct_\mathrm{stop} \cdot n
\end{equation}



\subsection{Some notes on the matrix operations}
\label{sec:matrixops}
%
\dots Normalisation, matrix inverses




\section{The measurement vector}
\label{sec:y}
%
\dots Test: \MsrVct


\section{The state vector}
\label{sec:x}
%
\dots


\section{The a priori vector}
\label{sec:x}
%
\dots


\section{Covariance matrices}
\label{sec:x}
%
\dots



\section{Forward model paramaters}
\label{sec:b}


\subsection{Absorption}
\label{sec:b:abs}
%
\dots






\subsection{???}
\label{sec:b:???}
%
$I_0$




%%% Local Variables: 
%%% mode: latex
%%% TeX-master: "L2_ATBD"
%%% End: 
