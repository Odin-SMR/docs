\chapter{Retrieval variables}
\label{chapter:retvars}

This chapter reviews the variables that are part of the retrievals, either
directly or indirectly. The presentation follows the formalism presented in
Sec.~\ref{sec:formalism}.


\section{Basic variable choices}
\label{sec:retr:basics}
%
\smr\ performs measurements in several frequency bands simultaneously
(Sec.~\ref{sec:smr:details}). The set of frequency bands differs between
observation modes \citep{atbdl1b}. The altitude range scanned by \smr\ is not
fixed, and a \smr\ limb ``scan'' is simply defined as the spectra recorded
between two adjacent turning points, in altitude, of the limb scanning
\citep{atbdl1b}. The retrievals covered by this \ATBD\ operate only on one
scan, from a single frequency band, at the time. Each combination of scan and
frequency band is treated separately, no information is carried between
different retrievals.

Following the discussion in Sec.~\ref{sec:setup:inverse:xb}, the state vector
does not only contain the quantities targeted directly by the \smr\
measurements, that is, gas species concentrations and atmospheric temperatures,
but also a number of other variables are part of \SttVct\ (Sec.~\ref{sec:x}).
These variables are either instrument correction factors or describe some
interfering atmospheric effect. However, a restriction to effects of random
nature is made, no fixed forward model variables (such as spectroscopic data)
are placed in \SttVct. That is, variables causing systematic retrieval errors
are not included in \SttVct\ (and neither in \aCvrMtr{o}, Sec.~\ref{sec:So}).

Covariance matrices are implemented as sparse matrices, while all other
variables are stored as full vectors or matrices. For the retrieval problem at
hand, $\aWfnMtr{\SttVct}$ is relatively dense and the overhead related with
a sparse representation results in better performance keeping
$\aWfnMtr{\SttVct}$ as a full matrix.



\section{The measurement vector, \MsrVct}
\label{sec:y}
%
The measurement vector contains solely \smr\ brightness temperature values,
with spectra from different tangent altitudes appended. That is,
\begin{equation}
  \label{eq:y}
  \MsrVct =
  \begin{bmatrix}
    \MsrVct_1\\ 
    \MsrVct_2\\ 
    \vdots\\ 
    \MsrVct_s
  \end{bmatrix},
\end{equation}
where $\MsrVct_1$/$\MsrVct_s$ is the spectrum from the highest/lowest tangent
altitude etc. 

Each retrieval only contains data from a single scan, as mentioned, but all
spectra of the scan are not included. For each mode an upper and lower tangent
altitude limit are defined, that determine which spectra of a scan that will be
part of \MsrVct. The altitude limits are defined using geometrical tangent
altitudes (that is, refraction neglected) as those are the tangent altitudes
found in the L1b data. A minimum scan range is defined in the same manner; no
retrieval is made if a limb scan does not cover this minimum range. The tangent
altitude masks for each frequency band are found in \citet{atbdl2data}.



\section{Observation uncertainty covariance matrix, \aCvrMtr{o}}
\label{sec:So}
%
All variables of random nature that have a significant impact on the
measurements, including indirect effects through the forward model, are part of
the state vector (Secs.~\ref{sec:retr:basics} and \ref{sec:x}). This means that
the term $\aWfnMtr{\FrwMdlVct}\aCvrMtr{\FrwMdlVct} \aWfnMtr{\FrwMdlVct}^T$ in
Eq.~\ref{eq:So} should give a relatively small contribution (ignoring
systematic components) to $\aCvrMtr{\MsrErrVct_o}$. Accordingly, \aCvrMtr{o} is
set to only represent thermal noise:
\begin{equation}
  \aCvrMtr{o} = \aCvrMtr{\MsrErrVct_n}.
\end{equation}
The standard deviations to include in $\aCvrMtr{\MsrErrVct_n}$ are set
according to \citet{atbdl1b}. The correlation of noise between adjecent
channels of each spectrum is considered. The magnitude of this correlation
depends on if a Hanning window has been applied or not. No noise correlation
between spectra from different altitudes is included.


\section{The state vector, \SttVct}
\label{sec:x}
%
The state vector contains a mix of quantities (Sec.~\ref{sec:retr:basics}).
Besides the 118\,GHz band, the quantities always present are: gas species
profiles $(\VctStl{v})$, the temperature profile $(\VctStl{t})$, a set of
brightness temperature off-sets $(\Delta\VctStl{o})$, a pointing off-set
$(\Delta\theta)$ and a frequency off-set $(\Delta\nu)$. With
just these quantities retrieved, and a case involving two gas species, the
state vector becomes:
\begin{equation}
  \label{eq:x:example}
  \SttVct =
  \begin{bmatrix}
    \VctStl{v}_1\\ 
    \VctStl{v}_2\\ 
    \VctStl{t}\\ 
    \VctStl{\Delta\VctStl{o}}\\ 
    \Delta\theta\\ 
    \Delta\nu 
  \end{bmatrix}.
\end{equation}
For 118\,GHz the state vector contains no species profiles as only oxygen,
having a known VMR, has to be considered for this frequency band.

A species or temperature ``profile'' is a set of values describing the vertical
variation, at a set of specified pressures, that is, the retrieval grid (see
also Sec.~\ref{sec:arts:aspects}). Retrieval grids are discussed further in
Sec.~\ref{sec:b:pgrid}. Atmospheric temperatures are in Kelvin. Gas species are
reported as \VMR, but the actual retrieval is internally performed using one of
the two following units:
\begin{itemize}
\item Default is to operate with the relative value with respect to a priori.
  In this ``relative unit'' a priori matches a value of 1.
\item A constrain of ensuring positive VMR values is achived by instead
  retrievaing the natural logarithm of the unit above. Using this unit, the a
  priori state corresponds to a vector of zeros.
\end{itemize}
Proper rescalings of the state vector and the weighting function matrix are
performed, following the retrieval unit applied.


As discussed in Sec.~\ref{sec:smr:errors}, there is not a perfect compensation
for gain variations, causing an off-set in brightness temperatures. This
off-set has been found to be flat over the frequency bands, but differ from one
tangent altitude to next. The vector \VctStl{\Delta\VctStl{o}} contains one
brightness temperature off-set for each included spectrum, to compensate for
the ``baseline shift''. That is, a correction for the brightness temperatures is
estimated, and corrected spectra can be expressed as
\begin{equation}
  \label{eq:retr:baseline}
  \hat{T}_b^{ij} = T_b^{ij} + \Delta\VctStl{o}_i
\end{equation}
where $T_b^{ij}$ is brightness temperature of channel $j$ and tangent altitude
$i$ and $\Delta\VctStl{o}_i$ is element $i$ of $\Delta\VctStl{o}$.

The pointing and frequency off-sets are scalar values. This implies that these
correction factors are valid for the complete scan. In \ARTS\ the viewing
direction, of each \LOS, is specified as the angle from zenith. These angles
cannot be determined perfectly from attitude data, but the error has been
judged to be constant during a scan and a single correction value should be
sufficient. Hence, the retrieved zenith angles are
\begin{equation}
  \label{eq:retr:point}
  \hat{\theta_i} = \theta_i + \Delta\theta,  
\end{equation}
where $\theta_i$ is the L1b zenith angle of tangent altitude $i$. A positive 
$\Delta\theta$ means that the retrieval estimated lower tangent altitudes than
obtained by the attitude data. As a rule of thumb, $\Delta\theta=\pm0.01\degree$
matches about $\mp550$\,m in tangent altitude.

There is a similar uncertainty in the exact frequencies observed. This
uncertainty originates in the \LO-signal. Already pre-launch tests showed that
\LO-frequencies vary somewhat with temperature. A correction table was created,
but some frequency error remains and there seems also to be some ageing on the
\LO-chain affecting final frequencies. The resulting frequency uncertainty is
totally correlated between the backend channels inside each frequency band. As
long as the \LO-signal is phase-locked, it is judged that the frequency off-set
is constant during a limb scan. However, the phase-locking of the
575\,GHz front-end has been failing basically throughout the mission, and the
assumption of a constant frequency off-set is more uncertain for this receiver
chain. The switch from version 1 to 2 of \ARTS\ enables the retrieval of a
non-constant frequency off-set, but this possibility will only be activated later
if found necessary. The retrieved frequency of channel $i$ is
\begin{equation}
  \label{eq:retr:freq}
  \hat{\nu_j} = \nu_j + \Delta\nu,  
\end{equation}
where $\nu_j$ is the frequency found given by L1b data.


%\section{The a priori vector, \aSttVct{a}}
%\label{sec:xa}
%
%\lcomment{PE}{Write later. For species we need to decide if we go for the old
%  a priori database, originally from Bordeaux. Or can we get a better, and much
%  newer, alternative? The MIPAS climatology, or from any of all ongoing
%  projects?}
%\lcomment{PE}{For temperature we need to decide if ECMWF-operational or
%  ERA-Interim shall be used as source. And how exactly the transition to MSIS
%  is made.}

%All sensor correction terms are assumed to have an average of zero. Accordingly, the
%a priori value of $\Delta\VctStl{o}$, $\Delta\theta$ and $\Delta\nu$ is
%throughout set to zero.


\section{A priori uncertainty covariance matrix, \aCvrMtr{a}}
\label{sec:Sa}
%
The setting of \aCvrMtr{a}\ is the most controversial part of \OEM, see
Sec.~\ref{sec:setup:inverse:xb} for some comments. This subject is not
elaborated further, it is just noticed that for VMR and atmospheric
temperature, in general, there exist no direct measurements on which total
covariance matrices can be based and ad hoc values are applied to a large
extent. In addition, the values selected results in that \aCvrMtr{a}\ in
general should over-estimate variability. Accordingly, \OEM\ is not used in a
strict sense, but the retrievals can still be classified is regularisation
\citep{ungermann2011tomographic} and close to optimal result should be obtained
as long as \aCvrMtr{a}\ is given ``reasonable'' values
\citep{eriksson:analy:00}. On the other hand, the uncertainties in the
instrumental variables are known fairly well and the corresponding parts of
\aCvrMtr{a}\ should be quite accurate.
\\
\lcomment{PE}{Species: Details here depends on retrieval unit selected, and
  text must be written latter. But likely, we use a $\sim$50\% variability,
  with a constrain of a minimum variability defined in \VMR\ (converted to
  relative values).} \lcomment{PE}{Temperature: For altitudes up to about
  45\,km, the standard deviation should be $\approx$2\,K. For higher
  altitudes, use fixed values, or does CIRA or MSIS contain tables on natural
  variability that can be used?}

The atmosphere exhibits vertical correlation, see \citet{eriksson:stati:02} for
examples on ozone and temperature correlations. In lack of direct measurements
covering all altitudes, vertical correlations are modelled by parametric
expressions, using a correlation length $l_c$. The correlation is modelled to
follow a Gaussian function,
\begin{equation}
  \label{eq:corr:gau}
  \rho(z_1,z_2) = exp\left(-4[(z_1-z_2)/(l_c(z_1)+l_c(z_2))]^2\right),
\end{equation}
or an exponential one,
\begin{equation}
  \label{eq:corr:exp}
  \rho(z_1,z_2) = exp\left(-2\left|z_1-z_2\right|/(l_c(z_1)+l_c(z_2))\right),
\end{equation}
where $\rho(z_1,z_2)$ is the correlation coefficient between altitudes $z_1$
and $z_2$. Note that the mean of the correlation length at $z_1$ and $z_2$ is
used, that is,  $l_c(z_1)/2+l_c(z_2)/2$.\todo{Where do we report $l_c$ used?}

The frequency and pointing off-sets are scalar values, and only a standard
deviation ($\sigma$) for each quantity must be specified. These standard deviations
are set to $\sigma_{\Delta\nu}=?$\,MHz and $\sigma_{\Delta\theta}=?^\circ$, respectively.
The brightness temperature off-sets have been found to be uncorrelated between
spectra, and the variation to have no variation with altitude. This results in
that the covariance matrix for $\VctStl{\Delta\VctStl{o}}$ can be written as 
$\sigma_{\VctStl{\Delta\VctStl{o}}}^2\IdnMtr$. The standard deviation of
brightness temperature off-sets is throughout set to ?\,K\todo{Set value for
  all $\sigma$.}.

Any correlations between the different quantities in \SttVct\ are fully
ignored. That is, the total covariance matrix becomes:
\begin{equation}
  \label{eq:sa}
  \aCvrMtr{a} =
  \begin{bmatrix}
    \aCvrMtr{\VctStl{v}_1} & 0 & 0 & 0 & 0 & 0 \\
    0 & \aCvrMtr{\VctStl{v}_2} & 0 & 0 & 0 & 0 \\
    0 & 0 & \aCvrMtr{\VctStl{t}} & 0 & 0 & 0 \\
    0 & 0 & 0 & \sigma_{\VctStl{\Delta\VctStl{o}}}^2\IdnMtr & 0 & 0 \\
    0 & 0 & 0 & 0 & \sigma_{\Delta\theta}^2 & 0 \\
    0 & 0 & 0 & 0 & 0 & \sigma_{\Delta\nu}^2 \\
  \end{bmatrix}.
\end{equation}
The inverse of a block-diagonal matrix of this type is also a block-diagonal
matrix, consisting of the inverse of the individual matrices, and the inverse
of \aCvrMtr{a}\ is calculated in the following manner:
\begin{equation}
  \label{eq:sainv}
  \aCvrMtr{a}^{-1} =
  \begin{bmatrix}
    \aCvrMtr{\VctStl{v}_1}^{-1} & 0 & 0 & 0 & 0 & 0 \\
    0 & \aCvrMtr{\VctStl{v}_2}^{-1} & 0 & 0 & 0 & 0 \\
    0 & 0 & \aCvrMtr{\VctStl{t}}^{-1} & 0 & 0 & 0 \\
    0 & 0 & 0 & \sigma_{\VctStl{\Delta\VctStl{o}}}^{-2}\IdnMtr & 0 & 0 \\
    0 & 0 & 0 & 0 & \sigma_{\Delta\theta}^{-2} & 0 \\
    0 & 0 & 0 & 0 & 0 & \sigma_{\Delta\nu}^{-2} \\
  \end{bmatrix}.
\end{equation}



\section{Forward model parameters, \FrwMdlVct}
\label{sec:b}
%
Please note that static forward model settings, which are common for all
simulations, are presented already in Sec.~\ref{sec:arts:config}.
\\
\lcomment{PE}{A first version of everything below exists, but all details
  should be revised/optimised further. Write each section below, when later
  looking at each part.}


\subsection{Pressure grids}
\label{sec:b:pgrid}
%
\lcomment{PE}{Describe how pressure grids are set, including retrieval ones. We
  operate with a fixed set of pressure where integer pressure decades always
  are included (e.g.\ 100 and 1\,hPa). The spacing is given as number of points
  per pressure decade, as also done for MLS. Retrieval grids always contain a
  sub-set of the forward model pressure grid.}


\subsection{Monochromatic frequency grids}
\label{sec:b:fgrid}
%
\lcomment{PE}{Describe the role of these grids, and how they are handled inside
  \ARTS, resulting in a need to have a more dense spacing around transitions.
  Make an illustrative figure. Describe algorithm to set these grids. Test how
  the calculation time scales with the number of frequencies. Is it close to
  linear? If yes, then we should try to improve the existing algorithm to set
  the grids. The most obvious choice if to use the Ffill option, but that
  requires careful testing to check that there are no caveats.}


\subsection{Spectroscopic data and absorption continua}
\label{sec:b:absdata}
%
\lcomment{PE}{The bulk of the data are taken from latest HITRAN. A literature
  review will be performed to hand-pick values for our most important
  transitions. The hand-picked data will be put in tables, that can be used to
  replace the corresponding value in HITRAN by \ARTS. The nitrogen and water
  vapour continua should be set as done for the tropospheric retrievals. For
  nitrogen this means the expression from MPM93 multiplied with 1.34
  following some references in the literature. Details given later.}



\subsection{Absorption look-up tables}
\label{sec:b:abstable}
%
\lcomment{PE}{Describe how absorption tables are generated, basically just what
  reference atmosphere that is used, and what pressure and temperature grids
  that are used. And also state what interpolation orders that are applied.}



\subsection{Radiative transfer}
\label{sec:b:rt}
%
The propagation path used when evaluating Eq.~\ref{eq:rteq} is defined by a set
of points. A point is added at each crossing of the path with a
pressure level (that is, the pressures defining the vertical grid). Additional
points are inserted to ensure that the distance along the path does not exceed
an user defined threshold. This threshold is denoted $l_\mathrm{max}$ in the
tables of ?\todo{Where?}.





%%% Local Variables: 
%%% mode: latex
%%% TeX-master: "L2_ATBD"
%%% End: 
