\chapter{Retrieval characterisation}
\label{chapter:characterisation}
%
A full retrieval characterisation is only given for a set of representative
cases. This despite that such a characterisation is largely performed
internally for each retrieved scan, but to store the full averaging kernel and
error covariance matrices would increase the size of the L2 data files with
about two orders of magnitude and such detailed information is anyhow normally
not required. Instead, only some key, all-embracing, quantities are included in
the L2 data, and full characterisations are provided separately. The background
theory of this chapter is found in Sec.~\ref{sec:formalism}.
 


\section{Interpretation of errors}
%
The full retrieval error is given by Eq.~\ref{eq:Sdelta1}, where the total
retrieval error is presented as the sum of three terms. The calculation and
interpretation of the forward model parameter and thermal noise retrieval
errors are quite straightforward, while the smoothing error term,
$\left(\AvrKrnMtr-\IdnMtr\right)\aCvrMtr{\SttVct}\left(\AvrKrnMtr-\IdnMtr\right)^T$,
is more problematic and frequenctly causes confusion. The aim of this section
is to clarify some aspects around smoothing errors and how this error term is
handled in the \smr\ L2 data. The averaging kernel matrix has a central
position for this discussion and it is defined by Eq.~\ref{eq:A}.


\subsection*{The case of a single species profile}
%
For pedagogical reasons, let us start with a simplified example, that the state
vector just contains the vertical profile of a single species. For such a
retrieval case, a row of \AvrKrnMtr\ shows how changes throughout the true
profile contribute to the retrieved value at the altitude of concern.
Accordingly the term ``averaging kernels'' refers to the rows of \AvrKrnMtr.
The columns \AvrKrnMtr\ give the impact of a delta-type perturbance in \SttVct\
\citep{rodgers:00}.

The characterisation makes no attempt to estimate this smoothing error (the one
internal for a retrieval quantity), for the simple reason that a sufficient
accurate \aCvrMtr{\SttVct}\ can not be obtained. To replace \aCvrMtr{\SttVct}\
with the corresponding matrix used as part of \OEM, \aCvrMtr{a}, is not a
viable option as it would give misleading results. The smoothing error should
in general be over-estimated, considering how \aCvrMtr{\SttVct}\ is constructed
(Sec.~\ref{sec:Sa}). A smoothing error is present when $\AvrKrnMtr\neq\IdnMtr$.
The averaging kernel matrix deviates from the identity matrix due to impact of
a priori information, that roughly can be separated into two main aspects:
limited resolution (see Sec.~\ref{sec:mresp}) and non-perfect measurement
response (see Sec.~\ref{sec:mresp}).

If the smoothing error is not included in the retrieval error, how should then
the two remaining error terms be interpreted? Some guidance to understand this
question can be obtained be a rearrangement of Eq.~\ref{eq:delta}:
\begin{equation}
  \RtrVct -\aSttVct{a} - \AvrKrnMtr\left( \SttVct-\aSttVct{a} \right) = 
    \CtrFncMtr\aWfnMtr{\FrwMdlVct}\left(\FrwMdlVct-\FrwMdlVctHat\right) +
    \CtrFncMtr\MsrErrVct_n.
\end{equation}
If $\AvrKrnMtr\aSttVct{a}\approx\aSttVct{a}$ we get
\begin{equation}
  \RtrVct - \AvrKrnMtr\SttVct \approx 
    \CtrFncMtr\aWfnMtr{\FrwMdlVct}\left(\FrwMdlVct-\FrwMdlVctHat\right) +
    \CtrFncMtr\MsrErrVct_n.
\end{equation}
That is, the forward model parameter and retrieval noise errors give the
difference between the retrieved state and a smoothed version of the true state
$(\AvrKrnMtr\SttVct)$. However, as \SttVct\ is unknown, and then also
$\AvrKrnMtr\SttVct$, a somewhat different view must be used in practice.
Retrieved profile shall be interpreted as a running averages of the true
profiles, and errors refer to these averages. The vertcial resolution gives the
approximate averaging length. The condition
$\AvrKrnMtr\aSttVct{a}\approx\aSttVct{a}$ is roughly fulfilled as long as the
measurement response is around 1.

For situations where the measurement provides no information at all, the values
in both \CtrFncMtr\ and \AvrKrnMtr\ become zero. This results in $\RtrVct -
\aSttVct{a} = 0$. That is, the solution equals the a priori state. A pitfall
emerges here, the errors reported also become zero, but this only reflects that
neither thermal noise or forward model parameters have had any impact on the
solution. The total retrieval error equals the smoothing error.

In summary, reported errors provide useful information only when the
measurement response is relatively high. However, retrieved data with low
measurement response shall anyhow not be used for a scientific analysis, see
further ?\todo{Where do we discuss this?}.


\subsection*{Multiple retrieval quantities}
%
As a more general example, if the state vector follows Eq.~\ref{eq:x:example},
the averaging kernel matrix has the structure:
\begin{equation}
  \label{eq:A:example}
  \AvrKrnMtr =
  \begin{bmatrix}
    \AvrKrnMtr_{\VctStl{v}_1} & \AvrKrnMtr_{\VctStl{v}_1,\VctStl{v}_2} &
    \AvrKrnMtr_{\VctStl{v}_1,\VctStl{t}} & \AvrKrnMtr_{\VctStl{v}_1,\VctStl{\Delta\VctStl{o}}} &
    \AvrKrnMtr_{\VctStl{v}_1,\Delta\theta} & \AvrKrnMtr_{\VctStl{v}_1,\Delta\nu} \\ 
    \AvrKrnMtr_{\VctStl{v}_2,\VctStl{v}_1} & \AvrKrnMtr_{\VctStl{v}_2} &
    \AvrKrnMtr_{\VctStl{v}_2,\VctStl{t}} & \AvrKrnMtr_{\VctStl{v}_2,\VctStl{\Delta\VctStl{o}}} &
    \AvrKrnMtr_{\VctStl{v}_2,\Delta\theta} & \AvrKrnMtr_{\VctStl{v}_2,\Delta\nu} \\ 
    \AvrKrnMtr_{\VctStl{t},\VctStl{v}_1} & \AvrKrnMtr_{\VctStl{t},\VctStl{v}_2} &
    \AvrKrnMtr_{\VctStl{t}} & \AvrKrnMtr_{\VctStl{t},\VctStl{\Delta\VctStl{o}}} &
    \AvrKrnMtr_{\VctStl{t},\Delta\theta} & \AvrKrnMtr_{\VctStl{t},\Delta\nu} \\ 
    \AvrKrnMtr_{\VctStl{\Delta\VctStl{o}},\VctStl{v}_1} & 
    \AvrKrnMtr_{\VctStl{\Delta\VctStl{o}},\VctStl{v}_2} &
    \AvrKrnMtr_{\VctStl{\Delta\VctStl{o}},\VctStl{t}} & \AvrKrnMtr_{\VctStl{\Delta\VctStl{o}}} &
    \AvrKrnMtr_{\VctStl{\Delta\VctStl{o}},\Delta\theta} &
    \AvrKrnMtr_{\VctStl{\Delta\VctStl{o}},\Delta\nu} \\ 
    \AvrKrnMtr_{\Delta\theta,\VctStl{v}_1} & \AvrKrnMtr_{\Delta\theta,\VctStl{v}_2} &
    \AvrKrnMtr_{\Delta\theta,\VctStl{t}} & \AvrKrnMtr_{\Delta\theta,\VctStl{\Delta\VctStl{o}}} &
    \AvrKrnMtr_{\Delta\theta} & \AvrKrnMtr_{\Delta\theta,\Delta\nu} \\ 
    \AvrKrnMtr_{\Delta\nu,\VctStl{v}_1} & \AvrKrnMtr_{\Delta\nu,\VctStl{v}_2} &
    \AvrKrnMtr_{\Delta\nu,\VctStl{t}} & \AvrKrnMtr_{\Delta\nu,\VctStl{\Delta\VctStl{o}}} &
    \AvrKrnMtr_{\Delta\nu,\Delta\theta} & \AvrKrnMtr_{\Delta\nu}
  \end{bmatrix}.
\end{equation}
The sub-matrices along the diagonal, such as $\AvrKrnMtr_{\VctStl{v}_2}$, can
be interpreted exactly as in the simplified example above. In case of scalar
quantities, such as $\AvrKrnMtr_{\Delta\nu}$, these sub-matrices have size 1\,x\,1.

The off-diagonal sub-matrices show how the different quantities affect each
other. For example, one row of the sub-matrix
$\AvrKrnMtr_{\VctStl{v}_1,\VctStl{t}}$ gives how the corresponding value in gas
species profile 1 is affected by variations in the temperature profile, while a
row in $\AvrKrnMtr_{\VctStl{t},\VctStl{v}_1}$ gives how one temperature profile
value is influenced by changes in of gas species 1.

The retrieval will in general not be able to make a perfect seperation between
changes in the different retrieval quantities, for example, variations in the
atmospheric temperature profile will cause errors in a ozone profile retrieval,
even if the temperature profile is part of \SttVct. This influence across the
retrieval quantities emerges as an additional smoothing error. To exemplify
this, let us continue to use a combined ozone and temperature retrieval:
\begin{displaymath}
  \AvrKrnMtr=  
  \begin{bmatrix}
    \CtrFncMtr_{\VctStl{v}}\\\CtrFncMtr_{\VctStl{t}}
  \end{bmatrix}
  \begin{bmatrix}
    \aWfnMtr{\VctStl{v}} & \aWfnMtr{\VctStl{t}}
  \end{bmatrix} =
  \begin{bmatrix}
    \aWfnMtr{\VctStl{v}} & \aWfnMtr{\VctStl{t}}
  \end{bmatrix} =
  \begin{bmatrix}
    \CtrFncMtr_{\VctStl{v}}\aWfnMtr{\VctStl{v}} &
    \CtrFncMtr_{\VctStl{v}}\aWfnMtr{\VctStl{t}} \\
    \CtrFncMtr_{\VctStl{t}}\aWfnMtr{\VctStl{v}} &
    \CtrFncMtr_{\VctStl{t}}\aWfnMtr{\VctStl{t}} \\
  \end{bmatrix} =
  \begin{bmatrix}
    \AvrKrnMtr_{\VctStl{v}} & \AvrKrnMtr_{\VctStl{v},\VctStl{t}} \\
    \AvrKrnMtr_{\VctStl{t},\VctStl{v}} & \AvrKrnMtr_{\VctStl{t}} 
  \end{bmatrix} 
\end{displaymath}
where the ozone profile is denoted as $\VctStl{v}$. The smoothing error
covariance matrix for this example becomes
\begin{displaymath}
  \left(\AvrKrnMtr-\IdnMtr\right)
  \begin{bmatrix}
    \aCvrMtr{\VctStl{v}} & 0 \\
    0 & \aCvrMtr{\VctStl{t}} 
  \end{bmatrix} 
  \left(\AvrKrnMtr-\IdnMtr\right)^T =
  \begin{bmatrix}
    (\AvrKrnMtr_{\VctStl{v}}-\IdnMtr)\aCvrMtr{\VctStl{v}}
    (\AvrKrnMtr_{\VctStl{v}}-\IdnMtr)^T + 
    \AvrKrnMtr_{\VctStl{v},\VctStl{t}}\aCvrMtr{\VctStl{t}}\AvrKrnMtr_{\VctStl{v},\VctStl{t}}^T
    & \hspace{5mm}\dots \\
    \hspace{-25mm}\dots & \hspace{-35mm}%
    (\AvrKrnMtr_{\VctStl{t}}-\IdnMtr)\aCvrMtr{\VctStl{t}}
    (\AvrKrnMtr_{\VctStl{t}}-\IdnMtr)^T + 
    \AvrKrnMtr_{\VctStl{t},\VctStl{v}}\aCvrMtr{\VctStl{v}}\AvrKrnMtr_{\VctStl{t},\VctStl{v}}^T
  \end{bmatrix} 
\end{displaymath}
where $\dots$ indicates matrix elements left out, for brevity reasons. If we
focus on the upper left sub-matrix, covering the ozone profile smoothing error,
the term $(\AvrKrnMtr_{\VctStl{v}}-\IdnMtr)\aCvrMtr{\VctStl{v}}
(\AvrKrnMtr_{\VctStl{v}}-\IdnMtr)^T$ covers the smoothing error that is
internal for the ozone profile, while
$\AvrKrnMtr_{\VctStl{v},\VctStl{t}}\aCvrMtr{\VctStl{t}}\AvrKrnMtr_{\VctStl{v},\VctStl{t}}^T$
is the smoothing error induced by atmospheric temperatures in the retrieval of
ozoze. The last term is new compared to the singel profile case. Interestingly,
this new term is identical the expression to be used if temperature had been a
forward model parameter, compare \dots




\section{Full characterisation}
%
\lcomment{PE}{How do we distribute the full characterisations?}


\subsection{Appraoch and reference cases}
\label{sec:fullchar:select}
%
The full characterisation does not involve any measurement data. Instead a
linear analysis is performed for a set of atmospheres selected from the gas
species a climatology. Temperature data are here taken solely from
?\todo{MSIS?}. A standard limb scan sequence is assumed. The integration times
?\todo{One, several, or mix?}. A retrieval is set-up according to the time and
position of the hypothetical measurement, and the characterisation is made for 
\aWfnMtr{\SttVct}\ and \CtrFncMtr\ following Eq.~\ref{eq:lindy}.
That is, a linear analysis around a priori is performed.
\\
\lcomment{PE}{How do we define the database of full characteristics?
  Combinations of monts (middle of every second?) and latitudes (-80 to 80 in
  steps of 20\degree?). The databse should cover all frequency bands.}


\subsection{Averaging kernel matrix, \AvrKrnMtr}
\label{sec:A}
%
The full characterisation reports the complete averaging kernel matrix,
together with the positions inside the state vector each retrieval quantity
occupies. Each sub-matrix of \AvrKrnMtr\ can be extracted if these positions
are known.
\\
\lcomment{PE}{Or do we just report the sub-matrices along the diagonal?}


\subsection{Retrieval noise covariance matrix, \aCvrMtr{n}}
\label{sec:Sreterr}
%
This error covariance matrix equals the last term of Eq.~\ref{eq:Sdelta1}:
\begin{equation}
  \label{eq:Sn}
  \aCvrMtr{n} = \CtrFncMtr\aCvrMtr{\MsrErrVct_n}\CtrFncMtr^T.
\end{equation}
This covariance matrix can be divided into sub-matrices exactly as the
averaging kernel matrix, where off-diagonal sub-matrices give information on
the error correlation between the different retrieval quantities. As any
covariance matrix, \aCvrMtr{n}\ is symmetric (which is not the case for
\AvrKrnMtr!), and corresponding sub-matrices above and below the diagonal provide
redundant information.

At a first glance, Eq.~\ref{eq:Sn}\ could give the impression that \aCvrMtr{n}\
only covers the error caused by thermal noise, but in the set-up selected,
\aCvrMtr{n}\ should cover all main random error sources. This is the
case as \dots\todo{Explain this}.
\\
\lcomment{PE}{Do we make use of the symmetry of \aCvrMtr{n}\ and only store the
  upper part of the matrix?}


\subsection{Random errors}
\label{sec:erand}
%
\lcomment{PE}{The standard deviation of the different random errors separately.
  How best calculated?}



\subsection{Systematic errors}
\label{sec:esyst}
%
\lcomment{PE}{Write this section later as we hopefully can make use of some
  ongoing \ARTS\ development here. The most important systematic errors should
  be due to spectrospcopic data, such as pressure broading parameters. These
  errors are normally determined by perturbing the spectroscopic data, but this
  is slow and requires a rather intricate set-up. \ARTS\ is right now extended
  to provide weighting functions
  for spectroscopic varaibles, which makes this process both faster and simpler.\\
  Any other systematic error source to consider?}



\section{L2 data fields}
%

\subsection{Measurement response}
\label{sec:mresp}
%
The averaging kernel matrix is in the L2 data summerised by two measures, where
the measurement response is one. The concept 


\subsection{Vertical resolution}
\label{sec:mresp}
%
\dots


\subsection{Retrieval noise}
\label{sec:reterr}
%
\dots






%%% Local Variables: 
%%% mode: latex
%%% TeX-master: "L2_ATBD"
%%% End: 
