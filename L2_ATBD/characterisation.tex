\chapter{Retrieval characterisation}
\label{chapter:characterisation}
%
A full retrieval characterisation is only given for a set of representative
cases. This despite that such a characterisation anyhow is largely performed
internally for each retrieved scan, but to store the full averaging kernel and
error covariance matrices would increase the size of the L2 data files with
about two orders of magnitude and such detailed information is anyhow normally
not required. Instead, only some all-embracing quantities are included in the
L2 data, and full characterisations are provided separately. The background
theory of this chapter is found in Sec.~\ref{sec:formalism}.
 


\section{Interpretation of errors}
\label{sec:char:interpret}
%
The full retrieval error is given by Eq.~\ref{eq:Sdelta1}, where the total
retrieval error is presented as the sum of three terms. The calculation and
interpretation of the forward model parameter and the thermal noise retrieval
errors are quite straightforward, while the smoothing error term,
$\left(\AvrKrnMtr-\IdnMtr\right)\aCvrMtr{\SttVct}\left(\AvrKrnMtr-\IdnMtr\right)^T$,
is more problematic and frequently causes confusion. The aim of this section
is to clarify some aspects of the smoothing error and how this error term is
handled in the \smr\ L2 data. The averaging kernel matrix has a central
role for this discussion and it is defined by Eq.~\ref{eq:A}.


\subsection*{The case of a single species profile}
%
Let us start with a simplified example; that the state vector just contains the
vertical profile of a single species. For such a retrieval case, a row of
\AvrKrnMtr\ shows how changes throughout the true profile contribute to the
retrieved value at the altitude of concern. Accordingly the term ``averaging
kernels'' refers to the rows of \AvrKrnMtr. The columns \AvrKrnMtr\ give the
impact of a delta-type perturbance in \SttVct\ \citep{rodgers:00}.

The characterisation makes no attempt to estimate this smoothing error (the one
internal for a retrieval quantity), for the simple reason that a sufficient
accurate \aCvrMtr{\SttVct}\ cannot be obtained. To replace \aCvrMtr{\SttVct}\
with the corresponding matrix used as part of \OEM, \aCvrMtr{a}, is not a
viable option as it would give misleading results. The smoothing error should
in general be over-estimated, considering how \aCvrMtr{\SttVct}\ is constructed
(Sec.~\ref{sec:Sa}). A smoothing error is present when $\AvrKrnMtr\neq\IdnMtr$.
The averaging kernel matrix deviates from the identity matrix due to the
impact regularisation, that roughly can be separated into two main aspects:
limited resolution (see Sec.~\ref{sec:vertres}) and non-perfect measurement
response (see Sec.~\ref{sec:mresp}).

If the smoothing error is not included in the retrieval error, how should then
the two remaining error terms be interpreted? Some guidance to understand this
question can be obtained by a rearrangement of Eq.~\ref{eq:delta}:
\begin{equation}
  \RtrVct -\aSttVct{a} - \AvrKrnMtr\left( \SttVct-\aSttVct{a} \right) = 
    \CtrFncMtr\aWfnMtr{\FrwMdlVct}\left(\FrwMdlVct-\FrwMdlVctHat\right) +
    \CtrFncMtr\MsrErrVct_n.
\end{equation}
If $\AvrKrnMtr\aSttVct{a}\approx\aSttVct{a}$ we get
\begin{equation}
  \RtrVct - \AvrKrnMtr\SttVct \approx 
    \CtrFncMtr\aWfnMtr{\FrwMdlVct}\left(\FrwMdlVct-\FrwMdlVctHat\right) +
    \CtrFncMtr\MsrErrVct_n.
\end{equation}
That is, the forward model parameter and retrieval noise errors describe the
difference between the retrieved state and a smoothed version of the true state
$(\AvrKrnMtr\SttVct)$. However, as \SttVct\ is unknown, and then also
$\AvrKrnMtr\SttVct$, a somewhat different view must be used in practice.
Retrieved profiles are running averages of the true profiles, and errors refer
to the accuracy of these averages. The vertical resolution gives the
approximate averaging length. The condition
$\AvrKrnMtr\aSttVct{a}\approx\aSttVct{a}$ is roughly fulfilled as long as the
measurement response is around 1.

For situations where the measurement provides no information at all, a zero
measurement response, the values in both \CtrFncMtr\ and \AvrKrnMtr\ become
zero. This results in $\RtrVct - \aSttVct{a} = 0$. That is, the solution equals
the a priori state. A pitfall emerges here, the errors reported also become
zero, but this only reflects that neither thermal noise or forward model
parameters have had any impact on the solution. The total retrieval error
equals the neglected smoothing error.



\subsection*{Multiple retrieval quantities}
%
As a more general example, if the state vector follows Eq.~\ref{eq:x:example},
the averaging kernel matrix has the structure:
\begin{equation}
  \label{eq:A:example}
  \AvrKrnMtr =
  \begin{bmatrix}
    \AvrKrnMtr_{\VctStl{v}_1} & \AvrKrnMtr_{\VctStl{v}_1,\VctStl{v}_2} &
    \AvrKrnMtr_{\VctStl{v}_1,\VctStl{t}} & \AvrKrnMtr_{\VctStl{v}_1,\VctStl{\Delta\VctStl{o}}} &
    \AvrKrnMtr_{\VctStl{v}_1,\Delta\theta} & \AvrKrnMtr_{\VctStl{v}_1,\Delta\nu} \\ 
    \AvrKrnMtr_{\VctStl{v}_2,\VctStl{v}_1} & \AvrKrnMtr_{\VctStl{v}_2} &
    \AvrKrnMtr_{\VctStl{v}_2,\VctStl{t}} & \AvrKrnMtr_{\VctStl{v}_2,\VctStl{\Delta\VctStl{o}}} &
    \AvrKrnMtr_{\VctStl{v}_2,\Delta\theta} & \AvrKrnMtr_{\VctStl{v}_2,\Delta\nu} \\ 
    \AvrKrnMtr_{\VctStl{t},\VctStl{v}_1} & \AvrKrnMtr_{\VctStl{t},\VctStl{v}_2} &
    \AvrKrnMtr_{\VctStl{t}} & \AvrKrnMtr_{\VctStl{t},\VctStl{\Delta\VctStl{o}}} &
    \AvrKrnMtr_{\VctStl{t},\Delta\theta} & \AvrKrnMtr_{\VctStl{t},\Delta\nu} \\ 
    \AvrKrnMtr_{\VctStl{\Delta\VctStl{o}},\VctStl{v}_1} & 
    \AvrKrnMtr_{\VctStl{\Delta\VctStl{o}},\VctStl{v}_2} &
    \AvrKrnMtr_{\VctStl{\Delta\VctStl{o}},\VctStl{t}} & \AvrKrnMtr_{\VctStl{\Delta\VctStl{o}}} &
    \AvrKrnMtr_{\VctStl{\Delta\VctStl{o}},\Delta\theta} &
    \AvrKrnMtr_{\VctStl{\Delta\VctStl{o}},\Delta\nu} \\ 
    \AvrKrnMtr_{\Delta\theta,\VctStl{v}_1} & \AvrKrnMtr_{\Delta\theta,\VctStl{v}_2} &
    \AvrKrnMtr_{\Delta\theta,\VctStl{t}} & \AvrKrnMtr_{\Delta\theta,\VctStl{\Delta\VctStl{o}}} &
    \AvrKrnMtr_{\Delta\theta} & \AvrKrnMtr_{\Delta\theta,\Delta\nu} \\ 
    \AvrKrnMtr_{\Delta\nu,\VctStl{v}_1} & \AvrKrnMtr_{\Delta\nu,\VctStl{v}_2} &
    \AvrKrnMtr_{\Delta\nu,\VctStl{t}} & \AvrKrnMtr_{\Delta\nu,\VctStl{\Delta\VctStl{o}}} &
    \AvrKrnMtr_{\Delta\nu,\Delta\theta} & \AvrKrnMtr_{\Delta\nu}
  \end{bmatrix}.
\end{equation}
The sub-matrices along the diagonal, such as $\AvrKrnMtr_{\VctStl{v}_2}$, can
be interpreted exactly as in the simplified example above. In the case of scalar
quantities, such as $\AvrKrnMtr_{\Delta\nu}$, these sub-matrices have size 1\,x\,1.

The off-diagonal sub-matrices show how the different quantities interfere with
each other \citep{baron:studi:02}. For example, one row of the sub-matrix
$\AvrKrnMtr_{\VctStl{v}_1,\VctStl{t}}$ gives how the corresponding value in gas
species profile 1 is affected by variations in the temperature profile, while a
row in $\AvrKrnMtr_{\VctStl{t},\VctStl{v}_1}$ gives how one temperature profile
value is influenced by changes in gas species 1.

The retrieval will in general not be able to make a perfect separation between
changes in the different retrieval quantities, for example, variations in the
atmospheric temperature profile will cause errors in a ozone profile retrieval,
even if the temperature profile is part of \SttVct. This influence across the
retrieval quantities causes an additional smoothing error. To exemplify
this, let us continue to use a combined ozone and temperature retrieval:
\begin{displaymath}
  \AvrKrnMtr=  
  \begin{bmatrix}
    \CtrFncMtr_{\VctStl{v}}\\\CtrFncMtr_{\VctStl{t}}
  \end{bmatrix}
  \begin{bmatrix}
    \aWfnMtr{\VctStl{v}} & \aWfnMtr{\VctStl{t}}
  \end{bmatrix} =
  \begin{bmatrix}
    \aWfnMtr{\VctStl{v}} & \aWfnMtr{\VctStl{t}}
  \end{bmatrix} =
  \begin{bmatrix}
    \CtrFncMtr_{\VctStl{v}}\aWfnMtr{\VctStl{v}} &
    \CtrFncMtr_{\VctStl{v}}\aWfnMtr{\VctStl{t}} \\
    \CtrFncMtr_{\VctStl{t}}\aWfnMtr{\VctStl{v}} &
    \CtrFncMtr_{\VctStl{t}}\aWfnMtr{\VctStl{t}} \\
  \end{bmatrix} =
  \begin{bmatrix}
    \AvrKrnMtr_{\VctStl{v}} & \AvrKrnMtr_{\VctStl{v},\VctStl{t}} \\
    \AvrKrnMtr_{\VctStl{t},\VctStl{v}} & \AvrKrnMtr_{\VctStl{t}} 
  \end{bmatrix} 
\end{displaymath}
where the ozone profile is denoted as $\VctStl{v}$. The smoothing error
covariance matrix for this example becomes
\begin{displaymath}
  \left(\AvrKrnMtr-\IdnMtr\right)
  \begin{bmatrix}
    \aCvrMtr{\VctStl{v}} & 0 \\
    0 & \aCvrMtr{\VctStl{t}} 
  \end{bmatrix} 
  \left(\AvrKrnMtr-\IdnMtr\right)^T =
  \begin{bmatrix}
    (\AvrKrnMtr_{\VctStl{v}}-\IdnMtr)\aCvrMtr{\VctStl{v}}
    (\AvrKrnMtr_{\VctStl{v}}-\IdnMtr)^T + 
    \AvrKrnMtr_{\VctStl{v},\VctStl{t}}\aCvrMtr{\VctStl{t}}\AvrKrnMtr_{\VctStl{v},\VctStl{t}}^T
    & \hspace{5mm}\dots \\
    \hspace{-25mm}\dots & \hspace{-35mm}%
    (\AvrKrnMtr_{\VctStl{t}}-\IdnMtr)\aCvrMtr{\VctStl{t}}
    (\AvrKrnMtr_{\VctStl{t}}-\IdnMtr)^T + 
    \AvrKrnMtr_{\VctStl{t},\VctStl{v}}\aCvrMtr{\VctStl{v}}\AvrKrnMtr_{\VctStl{t},\VctStl{v}}^T
  \end{bmatrix} 
\end{displaymath}
where $\dots$ indicates matrix elements left out, for brevity reasons. If we
focus on the upper left sub-matrix, covering the ozone profile smoothing error,
the term $(\AvrKrnMtr_{\VctStl{v}}-\IdnMtr)\aCvrMtr{\VctStl{v}}
(\AvrKrnMtr_{\VctStl{v}}-\IdnMtr)^T$ covers the smoothing error that is
internal for the ozone profile, while the second term,
$\AvrKrnMtr_{\VctStl{v},\VctStl{t}}\aCvrMtr{\VctStl{t}}\AvrKrnMtr_{\VctStl{v},\VctStl{t}}^T$
is the smoothing error induced by atmospheric temperatures in the retrieved
ozone profile. The last term is new compared to an ozone-only retrieval.

Not surprisingly, this new term is identical to the expression to be used if
temperature had been a forward model parameter. If
$\AvrKrnMtr_{\VctStl{v},\VctStl{t}}$ is expanded, we get that the ozone retrieval
error due to temperature interference is \citep{rodgers2003intercomparison}
\begin{displaymath}
  \RtrErr_{\VctStl{v}}^{\VctStl{t}} =
  \CtrFncMtr_{\VctStl{v}}\aWfnMtr{\VctStl{t}}\aCvrMtr{\VctStl{t}}
   \aWfnMtr{\VctStl{t}}^T\CtrFncMtr_{\VctStl{v}}^T.
\end{displaymath}
This expression gives exactly the same result as treating temperature as a
forward model uncertainty and following Eq.~\ref{eq:So} for the error
calculation. This is consistent with the discussion in
Sec.~\ref{sec:setup:inverse:xb}; that for a linear case exactly the same result
is obtained when moving an interfering quantity from \FrwMdlVct\ to \SttVct.
This implies that the total error remains constant, but the corresponding error
term is at the same time moved from being a forward model parameter error to
being a smoothing error. 

(It should be noted that the last paragraphs make the silent assumption that
there is no correlation between the retrieval quantities; off-diagonal
sub-matrices of \aCvrMtr{\VctStl{\SttVct}}\ are zero. This is throughout
assumed for the \SMR\ processing and the general case is not discussed.)

The interference between retrieval quantities is included in the error
provided. The calculation of these errors also requires \aCvrMtr{\SttVct}, but
the accuracy of this matrix is here less important, compared to the estimation
of the internal smoothing error. The largest uncertainties of
\aCvrMtr{\SttVct}\ are found for gas species and temperature, but interference
between these quantities is limited and, hence, it does not give rise to any
dominating retrieval error. The interference of instrumental quantities on
\VMR\ retrievals can be substantial, but \aCvrMtr{\SttVct}\ can be
set quite accurately for the parts corresponding to instrumental uncertainties.\\

\noindent
In summary, the smoothing error internal to each retrieval quantity is not part
of the error given. Accordingly, the error refers to the accuracy of the
retrieved profile considered as a running average of the true profile. The
interference between different retrieval quantities is included. The errors are
valid on the constraint that the measurement response is relatively high. If
the measurement response is low, there is a systematic bias towards the a
priori state, that is not captured by the errors provided, but data with
low measurement response shall anyhow not be used for a scientific analysis
without special treatment (see \citet{rodgers2003intercomparison} for general
guidelines).



\section{Full characterisation}
%
\lcomment{PE}{How do we distribute the full characterisations? Take the text
  below just as some brainstorming around how we shall do this.}


\subsection{Approach and reference cases}
\label{sec:fullchar:select}
%
The full characterisation does not involve any measurement data. Instead a
linear analysis is performed for a set of atmospheres selected from the gas
species a climatology. Temperature data are here taken solely from
?\todo{MSIS?}. A standard limb scan sequence is assumed. The integration times
?\todo{One, several, or mix?}. A retrieval is set-up according to the time and
position of the hypothetical measurement, and the characterisation is made for
\aWfnMtr{\SttVct}\ and \CtrFncMtr\ following Eq.~\ref{eq:lindy}. That is, a
linear analysis around a priori is performed.
\\
\lcomment{PE}{How do we define the database of full characteristics?
  Combinations of months (middle of every second?) and latitudes (-80 to 80 in
  steps of 20\degree?). The database should cover all frequency bands. There
  should also be a discussion on what database case should be selected to match
  a particular measurement. Should we recommend scaling following the
  measurement response, where possible?}


\subsection{Averaging kernel matrix, \AvrKrnMtr}
\label{sec:A}
%
The full characterisation reports the complete averaging kernel matrix,
together with the positions inside the state vector each retrieval quantity
occupies. Each sub-matrix of \AvrKrnMtr\ can be extracted if these positions
are known. The a priori data for which the averaging kernel matrix are also
stored \dots 
\\
\lcomment{PE}{Exemplify how these data can be used.}


\subsection{Error covariance matrices}
\label{sec:Sreterr}
%
According to the discussion above, the random retrieval error is defined as the
total retrieval error (Eq.~\ref{eq:Sdelta1}) with the internal smoothing error
subtracted. The covariance matrix of this error is schematically calculated as
\begin{equation}
  \label{eq:Sdeltar}
  \aCvrMtr{\RtrErr_r} = \left(\AvrKrnMtr-\IdnMtr\right)\aCvrMtr{\SttVct}
  \left(\AvrKrnMtr-\IdnMtr\right)^T -
  \begin{bmatrix}
  \CvrMtr_\mathrm{is}^1 & 0 & \dots & 0 \\
  0 & \CvrMtr_\mathrm{is}^2 & \dots & 0 \\
  0 & \vdots & \ddots & 0 \\
  0 & 0 & 0 &  \CvrMtr_\mathrm{is}^q
  \end{bmatrix} 
  + \CtrFncMtr\aCvrMtr{\MsrErrVct_n}\CtrFncMtr^T,
\end{equation}
where $q$ is the number of retrieval quantities and $\CvrMtr_\mathrm{is}^i$ is
the internal smoothing error of quantity $i$, that using the nomenclature of
Sec.~\ref{sec:char:interpret} is
\begin{displaymath}
  \CvrMtr_\mathrm{is}^i = (\AvrKrnMtr_{\VctStl{i}}-\IdnMtr)\aCvrMtr{\VctStl{i}}
(\AvrKrnMtr_{\VctStl{i}}-\IdnMtr)^T.
\end{displaymath}
The \aCvrMtr{\RtrErr_r}\ covariance matrix can be divided into sub-matrices
exactly as the averaging kernel matrix, where off-diagonal sub-matrices give
information on the error correlation between the different retrieval
quantities. As any covariance matrix, \aCvrMtr{\RtrErr_r}\ is symmetric (which
is not the case for \AvrKrnMtr!), and corresponding sub-matrices above and
below the diagonal provide redundant information.

For completeness, the covariance matrix of the retrieval error due to thermal
noise is also given
\begin{equation}
  \label{eq:Sdeltan}
  \aCvrMtr{\RtrErr_n} = \CtrFncMtr\aCvrMtr{\MsrErrVct_n}\CtrFncMtr^T.
\end{equation}
\\
\lcomment{PE}{Do we make use of the symmetry and only
  store the upper part of the matrices?}


\subsection{Random errors}
\label{sec:erand}
%
These errors show how the different sources contribute to the total random
error. That is, the error due to thermal noise and all interfering smoothing
errors are reported individually. Only standard deviations $(\sigma)$ are
given. The total error $\sigma_\mathrm{tot}$, is related to the individual
error components as
\begin{equation}
  \label{eq:randerr:sigma}
  \sigma_\mathrm{tot}^2 = \sigma_\mathrm{n}^2 + \sum_{i=1}^{q-1} \sigma_{i}^2
\end{equation}
where $\sigma_\mathrm{n}$ is the error due to thermal noise, and $\sigma_{i}$
is an interference error. The matching value along the diagonal of
$\aCvrMtr{\RtrErr_o}$ equals $\sigma_\mathrm{tot}^2$.


\subsection{Systematic errors}
\label{sec:esyst}
%
\lcomment{PE}{Write this section later as we hopefully can make use of some
  ongoing \ARTS\ development here. The most important systematic errors should
  be due to spectroscopic data, such as pressure broadening parameters. These
  errors are normally determined by perturbing the spectroscopic data, but this
  is slow and requires a rather intricate set-up. \ARTS\ is right now extended
  to provide weighting functions
  for spectroscopic variables, which makes this process both faster and simpler.\\
  Any other systematic error source to consider?}



\section{L2 data fields}
%
The L2 data contain only some key characteristics, but these are calculated
specifically for each retrieval\todo{Polish section when L2 format set.}.
\\
\lcomment{PE}{How is \CtrFncMtr\ set? By pure linear expression, or can we get
  an effective \CtrFncMtr\ out of \LM?} \lcomment{PE}{When things are clearer,
  consider to add the diagonal sub-matrices of averaging kernel and error
  covariance matrices. That should be very useful, but would use much more disk
  space.}


\subsection{Measurement response}
\label{sec:mresp}
%
The averaging kernel matrix is in the L2 data summarised by three measures,
where the measurement response is one. The measurement response is defined as
the row sum of the averaging kernel matrix \citep{baron:studi:02}, but
following the treatment of smoothing errors, the summing is here only made
inside each sub-matrix along the diagonal. That is, only the measurement
response inside each retrieval quantity is considered.


\subsection{Vertical placement and resolution}
\label{sec:vertres}
%
Also these measures is based on the diagonal sub-matrices, and not the
complete \AvrKrnMtr. The vertical resolution is reported as the full width at
half maximum (\FWHM) of the averaging kernels. An averaging kernel is not
necessarily centred around the nominal retrieval altitude. The vertical
placement of the averaging kernel is given as the mean of altitude:
\begin{equation}
  \label{eq:zmean}
  z_m = \frac{\int z a(z_0,z)\, \DiffD z}{\int a(z)\,\DiffD z}
\end{equation}
where $z$ is altitude and $a(z)$ is the averaging kernel for the nominal
altitude of $z_0$\todo{Or give $z_m-z_0$?}.
\\
\lcomment{PE}{\FWHM\ is a quite optimistic value, especially for averaging
  kernels of the shape given by \OEM. Also include spread?}


\subsection{Random retrieval error}
\label{sec:reterr}
%
The total random error, $\sigma_\mathrm{tot}$, is given following
Eq.~\ref{eq:randerr:sigma}.






%%% Local Variables: 
%%% mode: latex
%%% TeX-master: "L2_ATBD"
%%% End: 
