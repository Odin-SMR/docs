\chapter{Format of L2 data}
\label{app:l2format}


\lcomment{PE}{Who will define the L2 format?}







%%% Local Variables: 
%%% mode: latex
%%% TeX-master: "L2_ATBD"
%%% End: 
