\chapter{Overview}
\label{chapter:overview}


\section{Terminology and data products}
\label{sec:terminology}
%
As mentioned in the Introduction, the topic of this document is the extraction
of geophysical data from observed spectra. This calculation step is normally
denoted as the retrieval or the inversion, where mainly the first term is used
in this document. The objective of \smr\ is to estimate various atmospheric
quantities, mainly the concentration of different gases, and these variables
are accordingly the main target of the retrieval. These data, together with
auxiliary information, form the L2 data. That is, the data that are made
available for the general scientific community. The format of the L2 data
treated by this document is described in Appendix~\ref{app:l2format}. The L2
data are public available, visit \url{odin.rss.chalmers.se} for registration
and download instructions.

Besides geophysical variables, the retrieval must consider a number of
instrumental variables to avoid that e.g.\ pointing errors cause unnecessary
large retrieval errors. The result of this part of the retrieval can be seen as
diagnostics data. Some of the quality fields of the L2 data are based on the
diagnostics data, and these data are stored internally for further analysis, to
e.g.\ detect instrumental drifts. 

The retrieval can also include atmospheric state variables that influence the
measurement, but the accuracy of the retrieval is not sufficiently high to
justify inclusion in the L2 data. Regarding atmospheric gases, the ones
included in the L2 data are denoted as target species, while other retrieved
gases are called secondary species.
\lcomment{PE}{Revise later to check if all fit with L2 format.}



\section{Theoretical formalism}
\label{sec:formalism}
%
The retrievals are presented and discussed using the formalism presented in
\citet[][Ch.~3]{rodgers:00}. The data to be inverted are appended to form the
measurement vector, \MsrVct. This vector is related to other variables as
\begin{equation}
  \label{eq:fmodel1}
  \MsrVct = \FrwMdl(\SttVct,\FrwMdlVct) + \MsrErrVct_n,
\end{equation}
where \FrwMdl\ is denoted as the forward model, \SttVct\ is the state vector,
\FrwMdlVct\ is the vector of forward model parameters and \MsrErrVct\
represents measurement noise. The distinction between the two arguments of the
forward model is that all variables that we want to retrieve form \SttVct,
while all other quantities that is required of the forward model are found in
\FrwMdlVct. 

The forward model treated explicitly in this document is the software used to
model atmospheric radiative transfer and sensor responses. This is a model
operating with discrete quantities, while the ``true'', hypothetical, forward
model, \trueFrwMdl, consisting of the actual physical mechanisms in the
atmosphere and the instrument, must be seen as a continuous function. If
\FrwMdl\ is constructed and used carefully, the discrete representation should
not cause any fundamental problems, and it is here assumed that all deviations
between \trueFrwMdl\ and \FrwMdl\ can be treated as imperfect values in
\FrwMdlVct. That is, $\trueFrwMdl=\FrwMdl(\SttVct,\FrwMdlVct)$, but the exact
values of \FrwMdlVct\ are unknown and all we can do is to use best possible
estimate, \FrwMdlVctHat. That is:
\begin{equation}
  \label{eq:fmodel2}
  \MsrVct = \FrwMdl(\SttVct,\FrwMdlVctHat) + \MsrErrVct_b + \MsrErrVct_n,
\end{equation}
where 
\begin{equation}
  \label{eq:buncert}
  \MsrErrVct_b = \trueFrwMdl - \FrwMdl(\SttVct,\FrwMdlVctHat).
\end{equation}
This difference ($\MsrErrVct_b$), is below denoted as the forward model
uncertainty. 

The retrieval process is also expressed as a general inverse model, \InvMdl:
\begin{equation}
  \label{eq:imodel}
   \RtrVct = \InvMdl(\MsrVct,\aSttVct{a},\FrwMdlVctHat,\InvMdlVct),
\end{equation}
where \RtrVct\ is the retrieved state vector and \InvMdlVct\ covers all
variables additional variables used only by the inverse model. The exact nature
of the vector \aSttVct{a}\ varies between retrieval approaches, but it 
represents in general the a priori estimate of \SttVct.

From this point it is assumed that the retrieval problem is not strongly
non-linear, and a local linear analysis is possible. Or expressed differently,
that derivatives of the forward and inverse models are approximately valid over a
significant range. A first example is the partial derivative of \FrwMdl\ with
respect to \SttVct:
\begin{equation}
  \label{eq:kx}
  \aWfnMtr{\SttVct} = \frac{\PartD\FrwMdl}{\PartD\SttVct}.
\end{equation}
This quantity is called the Jacobian, or the weighting function matrix. In the
same manner, we define \aWfnMtr{\FrwMdlVct} as
\begin{equation}
  \label{eq:kb}
  \aWfnMtr{\FrwMdlVct} = \frac{\PartD\FrwMdl}{\PartD\FrwMdlVct}.
\end{equation}
The contribution function matrix, \CtrFncMtr, is defined as the partial
derivative of the inverse model with respect to the measurement vector:
\begin{equation}
  \label{eq:dy}
  \CtrFncMtr = \frac{\PartD\InvMdl}{\PartD\MsrVct}.
\end{equation}
Having these partial derivatives at hand, the retrieval error can be related
to the fundamental uncertainties. As a first step, the forward model is
linearised around (\aSttVct{a},\FrwMdlVctHat):
\begin{equation}
  \label{eq:fmodel3}
  \MsrVct = \FrwMdl(\aSttVct{a},\FrwMdlVctHat) + 
  \aWfnMtr{\SttVct}\left(\SttVct-\aSttVct{a}\right) +
  \aWfnMtr{\FrwMdlVct}\left(\FrwMdlVct-\FrwMdlVctHat\right) +
  \MsrErrVct_n
\end{equation}
On the conditions given, we have now a second way to calculate the forward
model uncertainty:
\begin{equation}
  \MsrErrVct_b = \aWfnMtr{\FrwMdlVct}\left(\FrwMdlVct-\FrwMdlVctHat\right).
\end{equation}
By combining the equations above and rearranging the terms, we can finally
derive an expression for the total retrieval error
\begin{equation}
  \label{eq:delta}
  \RtrErr = \RtrVct - \SttVct =  \left(\AvrKrnMtr-\IdnMtr\right)
    \left(\SttVct-\aSttVct{a}\right) + 
    \CtrFncMtr\aWfnMtr{\FrwMdlVct}\left(\FrwMdlVct-\FrwMdlVctHat\right) +
    \CtrFncMtr\MsrErrVct_n
\end{equation}
The terms on the right hand side are denoted as smoothing error, forward model
error and thermal noise error, respectively. These error components are
discussed further in Chapter~\ref{chapter:characterisation}.


\section{Selected set-up}
\label{sec:setup}

\subsection{Retrieval method}
\label{sec:setup:inverse}
%





\section{Non-standard retrievals}
\label{sec:nonstandard}
%
\dots









%%% Local Variables: 
%%% mode: latex
%%% TeX-master: "L2_ATBD"
%%% End: 
