\chapter{Theory and overview}
\label{chapter:overview}


\section{Terminology and data products}
\label{sec:terminology}
%
As mentioned in the Introduction, the overall topic of this document is the
extraction of geophysical data from observed spectra. This calculation step is
normally denoted as the retrieval or the inversion, where mainly the first term
is used in this document. The objective of \smr\ is to estimate atmospheric
quantities, mainly the concentration of different gases, and these variables
are accordingly the main target of the retrieval, and, together with auxiliary
information, they form the official L2 data. The format of the \smr\ L2 data is
described in Appendix~\ref{app:l2format}. The L2 data are publicly available,
visit \url{odin.rss.chalmers.se} for registration and downloading instructions.

Besides geophysical variables, the retrieval must consider a number of
instrumental aspects to avoid that e.g.\ pointing errors cause unnecessary
large retrieval errors. The result of this part of the retrieval can be seen as
diagnostics data. Some of the quality fields of the L2 data are based on the
diagnostics data, and these data are stored internally for further analysis, to
e.g.\ detect instrumental drifts. 

The retrieval can also include atmospheric state variables that influence the
measurement, but the accuracy of the retrieval is not sufficiently high to
justify inclusion in the L2 data. Regarding atmospheric gases, the ones
included in the L2 data are denoted as target species, while other retrieved
gases are called secondary species.
\lcomment{PE}{Revise later to check if all fit with L2 format.}



\section{Radiative transfer basics}
\label{sec:rt}
%
There is no \ATBD\ dedicated to radiative transfer, as a general software is used
to handle this task (Sec.~\ref{sec:setup:forward}). Instead the main
considerations around radiative transfer are commnted inside this \ATBD. For a
more detailed discussion, targeting \smr, the reader is referred to
\citet{eriksson:studi:02}.

A strong advantage of using microwaves is that local thermodynamic equilibrium
(LTE) can be assumed, even for radiative transfer in the mesosphere and lower
thermosphere. Further, as long as the lowest altitude of the line-of-sight, the
tangent point, is some kilometres above the tropopause, scattering can safely
be neglected as \smr\ operates at wavelengths $>$\,0.5\,mm. This restriction in
tangent altitudes is made for the operational \smr\ retrievals. 

The combination of LTE and no scattering makes the basic simulation task
relatively simple. In fact, when it comes to atmospheric radiative transfer,
the required simulations boil down to summing up emission along the propagation
path, weighted with the transmission between \smr\ and the point of emission.
That is, the expression for radiative transfer to handle is \citep[see
e.g.][Eq.~50]{chandrasekhar:60}:
\begin{equation}
  \label{eq:rteq}
     I(\nu) = I_0(\nu)e^{-\int^l_0{k(s,\nu)ds}} + 
     \int^l_0{k(s,\nu)B(T(s),\nu) e^{-\int^l_0{k(s',\nu)ds'}} ds},
\end{equation}
where $I$ is radiance, $I_0$ is the radiance along the line-of-sight (LOS) at
the top of the atmosphere, where the propagation path starts, $l$ is the length
of this propagation path, $k$ is the absorption coefficient, $s$ is the
distance along the propagation path, and $B$ is the Planck function giving the
emission of a blackbody at temperature $T$ and frequency $\nu$.
Eq.~\ref{eq:rteq} assumes that the gaseous absorption is unpolarised. This is a
valid assumption in the microwave region for all species, besides oxygen that
is affected by the Zeeman effect \todo{Should this not be "for all species, except for oxygen which is affected by the Zeeman effect"?}. That is, Earth's magnetic field induces
polarised absorption and emission, and a matrix-vector equivalent to
Eq.~\ref{eq:rteq} is required to handle oxygen \citep[see
e.g.][]{larsson:zeema:14}.

Evaluation of Eq.~\ref{eq:rteq} gives the monochromatic intensity (radiance)
along pencil-beam propagation paths, matching infinite frequency and angular
resolution. Accordingly, Eq.~\ref{eq:rteq} must be solved for a set of
frequencies and propagation paths, to make it possible to incorporate the
antenna pattern, channel frequency response and other characteristics of the
sensor in the simulations \citep[see e.g.][]{eriksson:studi:02,read2006clear}.


\section{Theoretical formalism}
\label{sec:formalism}

\subsection{Forward and inverse models}
%
The retrievals are presented and discussed using the formalism presented in
Chapter~3 of \citet{rodgers:00}. The data to be inverted are appended to create
the measurement vector, \MsrVct. This vector is related to other variables as
\begin{equation}
  \label{eq:fmodel1}
  \MsrVct = \FrwMdl(\SttVct,\FrwMdlVct) + \MsrErrVct_n,
\end{equation}
where \FrwMdl\ is denoted as the forward model, \SttVct\ as the state vector,
\FrwMdlVct\ as the vector of forward model parameters and $\MsrErrVct_n$
represents measurement noise. The distinction between the two arguments of the
forward model is that all variables that we want to retrieve form \SttVct,
while all other quantities required of the forward model are placed in
\FrwMdlVct. 

The forward model treated explicitly in this document is the software used to
model atmospheric radiative transfer and sensor responses. This is a model
operating with discrete quantities, while the ``true'', hypothetical, forward
model, \trueFrwMdl, representing the actual physical mechanisms in the atmosphere
and the instrument, must be seen as a continuous function. If \FrwMdl\ is
constructed and used carefully, the discrete representation should not cause
any fundamental problems, and it is here assumed that all deviations between
\trueFrwMdl\ and \FrwMdl\ can be treated as imperfect values in \FrwMdlVct.
That is, $\trueFrwMdl=\FrwMdl(\SttVct,\FrwMdlVct)$, but the exact values of
\FrwMdlVct\ are unknown and all we can do is to use the best possible estimate,
\FrwMdlVctHat. That is:
\begin{equation}
  \label{eq:fmodel2}
  \MsrVct = \FrwMdl(\SttVct,\FrwMdlVctHat) + \MsrErrVct_b + \MsrErrVct_n,
\end{equation}
where 
\begin{equation}
  \label{eq:eps_b}
  \MsrErrVct_b = \trueFrwMdl - \FrwMdl(\SttVct,\FrwMdlVctHat).
\end{equation}
This difference ($\MsrErrVct_b$) is below denoted as the forward model
uncertainty. 

The retrieval process is formalised in the inverse model, \InvMdl:
\begin{equation}
  \label{eq:imodel}
   \RtrVct = \InvMdl(\MsrVct,\aSttVct{a},\FrwMdlVctHat,\InvMdlVct),
\end{equation}
where \RtrVct\ is the retrieved state vector and \InvMdlVct\ covers all
additional variables introduced by the inverse model. The exact nature
of the vector \aSttVct{a}\ varies between retrieval approaches, but in general it 
represents an a priori estimate of \SttVct.


\subsection{Linearisation}
%
From this point it is assumed that the retrieval problem is not strongly
non-linear, and a local linear analysis is possible. Or expressed differently,
that derivatives of the forward and inverse models are approximately valid over
a significant range. The Jacobian, or the weighting function matrix, is the
partial derivative of \FrwMdl\ with respect to \SttVct:
\begin{equation}
  \label{eq:kx}
  \aWfnMtr{\SttVct} = \frac{\PartD\FrwMdl}{\PartD\SttVct}.
\end{equation}
In the same manner, we define \aWfnMtr{\FrwMdlVct} as
\begin{equation}
  \label{eq:kb}
  \aWfnMtr{\FrwMdlVct} = \frac{\PartD\FrwMdl}{\PartD\FrwMdlVct}.
\end{equation}
The contribution function matrix, \CtrFncMtr, is defined as the partial
derivative of the inverse model with respect to the measurement vector:
\begin{equation}
  \label{eq:g}
  \CtrFncMtr = \frac{\PartD\InvMdl}{\PartD\MsrVct}.
\end{equation}
Having these partial derivatives at hand, the retrieval error can be related
to the fundamental uncertainties. As a first step, the forward model is
linearised around (\aSttVct{a},\FrwMdlVctHat):
\begin{equation}
  \label{eq:fmodel3}
  \MsrVct = \FrwMdl(\aSttVct{a},\FrwMdlVctHat) + 
  \aWfnMtr{\SttVct}\left(\SttVct-\aSttVct{a}\right) +
  \aWfnMtr{\FrwMdlVct}\left(\FrwMdlVct-\FrwMdlVctHat\right) +
  \MsrErrVct_n
\end{equation}
On the conditions given, we have now a second way to express the forward
model uncertainty \todo{Consider changing frasing to "Under these assumptions, we now have a second ..."}:
\begin{equation}
  \MsrErrVct_b = \aWfnMtr{\FrwMdlVct}\left(\FrwMdlVct-\FrwMdlVctHat\right).
\end{equation}
By combining the equations above and rearranging the terms, we can finally
derive an expression for the total retrieval error, $\RtrErr = \RtrVct - \SttVct$:
\begin{equation}
  \label{eq:delta}
  \RtrErr =  \left(\AvrKrnMtr-\IdnMtr\right)
    \left(\SttVct-\aSttVct{a}\right) + 
    \CtrFncMtr\aWfnMtr{\FrwMdlVct}\left(\FrwMdlVct-\FrwMdlVctHat\right) +
    \CtrFncMtr\MsrErrVct_n,
\end{equation}
where \IdnMtr\ is the identity matrix and
\begin{equation}
  \label{eq:A}
  \AvrKrnMtr = \CtrFncMtr\aWfnMtr{\SttVct},
\end{equation}
is the averaging kernel matrix. The terms on the right hand side of
Eq.~\ref{eq:delta} are denoted as smoothing error, forward model retrieval
error and measurement noise retrieval error, respectively. These error
components are discussed further in Chapter~\ref{chapter:characterisation}.


\subsection{Calculation of retrieval error}
%
Despite that Eq.~\ref{eq:delta} gives an expression for the retrieval error it cannot be used in practice, for the simple reason that \SttVct\ and \FrwMdlVct\
are not known. The retrieval error can only evaluated in a statistical sense,
and for this purpose uncertainties and errors are described by
covariance matrices (\CvrMtr). See e.g.\
\url{en.wikipedia.org/wiki/Covariance_matrix} for a description of this type of
matrices, as well as the basic calculation rules needed to understand the error
propagation. The statistical correspondence to Eq.~\ref{eq:delta} is
\begin{equation}
  \label{eq:Sdelta1}
  \aCvrMtr{\RtrErr} = \left(\AvrKrnMtr-\IdnMtr\right)\aCvrMtr{\SttVct}
  \left(\AvrKrnMtr-\IdnMtr\right)^T + 
  \CtrFncMtr\aWfnMtr{\FrwMdlVct}\aCvrMtr{\FrwMdlVct}
  \aWfnMtr{\FrwMdlVct}^T\CtrFncMtr^T + 
  \CtrFncMtr\aCvrMtr{\MsrErrVct_n}\CtrFncMtr^T.
\end{equation}
The combination of the instrument and the forward model can be seen as
the ``observation system'', then having an uncertainty of
\begin{equation}
  \label{eq:eps_o}
  \MsrErrVct_o = \MsrErrVct_b + \MsrErrVct_n. 
\end{equation}
The covariance matrix of the observation system uncertainty is 
\begin{equation}
  \label{eq:So}
  \aCvrMtr{\MsrErrVct_o} = \aWfnMtr{\FrwMdlVct}\aCvrMtr{\FrwMdlVct}
  \aWfnMtr{\FrwMdlVct}^T + \aCvrMtr{\MsrErrVct_n}.
\end{equation}
Using the definition of $\aCvrMtr{\MsrErrVct_o}$, Eq.~\ref{eq:Sdelta1}
can be written in a somewhat more compact form:
\begin{equation}
  \label{eq:Sdelta2}
  \aCvrMtr{\RtrErr} = \left(\AvrKrnMtr-\IdnMtr\right)\aCvrMtr{\SttVct}
  \left(\AvrKrnMtr-\IdnMtr\right)^T + \CtrFncMtr\aCvrMtr{\MsrErrVct_o}\CtrFncMtr^T.
\end{equation}



\section{Selected set-up}
\label{sec:setup}

\subsection{Retrieval method}
\label{sec:setup:inverse}
%

\subsubsection{The need for regularisation}
\label{sec:setup:inverse:reg}
%
In short, the task of the retrieval method is to derive \RtrVct\ from \MsrVct.
It turns out that the mapping from \MsrVct\ to \RtrVct\ is not unique. That is,
there exists an infinite set of $\SttVct$-vectors that result in a fit with the
measurement, considering the measurement noise
($\MsrVct-\FrwMdl(\SttVct,\FrwMdlVctHat)=O(\MsrErrVct_n)$\todo{Is this correct
  nomenclature?}). Instead, some ``optimal'' state must be selected, among this
set of possible solutions.

If we for a moment assume that the retrieval problem can be treated as fully
linear around (\aSttVct{a},\FrwMdlVctHat), we
have that $\aWfnMtr{\SttVct}\left(\SttVct-\aSttVct{a}\right)\approx \MsrVct
-\FrwMdl(\aSttVct{a},\FrwMdlVctHat)$ and the solution can be written as
\begin{equation}
  \label{eq:linretrieval}
  \RtrVct = \aSttVct{a} + \CtrFncMtr(\MsrVct-\FrwMdl(\aSttVct{a},\FrwMdlVctHat)).
\end{equation}
It is clear that \CtrFncMtr, at least roughly, represents an inverse of
\aWfnMtr{\SttVct}, but to use the standard inverse of \aWfnMtr{\SttVct} (only
possible if \aWfnMtr{\SttVct} is square) or to apply the least squares method
($\CtrFncMtr=(\aWfnMtrTrp{\SttVct}\aWfnMtr{\SttVct})^{-1}\aWfnMtr{\SttVct}$)
will result in an extremely high sensitivity to noise. The retrieval problem at
hand is said to be ill-posed, which is the standard situation for passive
atmospheric sounding.


\subsubsection{The optimal estimation method}
\label{sec:setup:inverse:oem}
%
The standard way to tackle ill-posed problems of this type is
``regularisation'', where some constrain on the solution is introduced. In
passive atmospheric sounding, using statistical information as basis for the
regularisation has become the standard approach, and it is also the approach
selected for \smr. The method is probably most widely known as the optimal
estimation method (OEM), and is the name selected for this document. The method
can be seen as an application of Bayes theorem, under certain conditions, and
another possible name of the method the maximum a posteriori solution, see
further \citet{rodgers:00}.

The aspect of ``statistical regularisation'' is probably most clearly
identified by noting that OEM minimises the following ``cost function'':
\begin{equation}
  \label{eq:costfun}
  \CstFnc(\SttVct) = (\MsrVct-\FrwMdl(\SttVct,\FrwMdlVctHat))^T\aCvrMtr{o}^{-1}
  (\MsrVct-\FrwMdl(\SttVct,\FrwMdlVctHat)) +
  (\SttVct-\aSttVct{a})^T\aCvrMtr{a}^{-1}(\SttVct-\aSttVct{a}),
\end{equation}
that is,
\begin{equation}
  \label{eq:mincost}
  \RtrVct = \min\limits_{\SttVct}(\CstFnc(\SttVct)).
\end{equation}
The cost function \CstFnc\ can be seen as the sum of the two penalty terms. The
first term weights how well the solution corresponds to the measurement. If
\CstFnc\ would be set to only contain this term, the standard least squares
solution would be obtained. The regularisation is introduced by the second
term, that evaluates the probability of \SttVct\ based on a priori
information.

If the retrieval problem is linear around (\aSttVct{a},\FrwMdlVctHat), \RtrVct\
can be calculated using Eq.~\ref{eq:linretrieval}, with
\citep[][Eq.~4.5]{rodgers:00}
\begin{equation}
  \label{eq:lindy}
  \CtrFncMtr = (\aWfnMtrTrp{\SttVct}\aCvrMtr{o}^{-1}\aWfnMtr{\SttVct}
  +\aCvrMtr{a}^{-1})^{-1}\aWfnMtrTrp{\SttVct}\aCvrMtr{o}^{-1}.
\end{equation}
However, the inversion of \smr\ measurements result in a (moderately)
non-linear problem and an iterative process is required to find the minimum of
\CstFnc, see Sec.~\ref{sec:ml}.


\subsubsection{Considerations around \SttVct\ and \FrwMdlVct}
\label{sec:setup:inverse:xb}
%
Formally, \aCvrMtr{a}\ shall be set to \aCvrMtr{\SttVct}\ \citep{rodgers:00},
that is, a description of the natural variability (including correlations) of
the elements in \SttVct, but this rule is not always obeyed. First of all,
normally there is no manner in which an exact \aCvrMtr{\SttVct} can be derived,
and this matrix is set in some parametric way, loosely based on measurements
and general knowledge of atmospheric physics. It could also be the case that
\aCvrMtr{a}\ is partly used as a ``tuning parameter''. The most common example
is that the variances in \aCvrMtr{a}\ are set with some margin with respect to the true
values, to maintain measurement information as far as possible, or expressed
differently, not to constrain the solution more than necessary. Further, it is
likely that \aSttVct{a}\ differs from the true mean \todo{Should there be two "mean"?} mean value of \SttVct, and this systematic deviation should also be represented in \aCvrMtr{a}\
\citep{eriksson:analy:00}. For a continued discussion, and comments on the
relationship between \aCvrMtr{a}\ and the corresponding regularisation matrix
used in ``Tikhonov regularisation'', see, for example,
\citet{eriksson:analy:00} and \citet{ungermann2011tomographic} \todo{Something wrong with the last reference}. 

Finally regarding \aCvrMtr{a}, the point made in Sec.~2.6 of \citet{rodgers:00}
is stressed, for given values along the diagonal, setting \aCvrMtr{a}\ to be
purely diagonal means a stronger regularisation than including reasonable
off-diagonal elements (to reflect correlations). It is a popular mistake to
think that this relationship is reversed. Here it should be noted that using a
diagonal \aCvrMtr{a}\ equals the common choice in Tikhonov regularisation to
select the ``smallest norm'' ($\MtrStl{L}_0$) solution.

The formally correct choice for \aCvrMtr{o} is \aCvrMtr{\MsrErrVct_o}
\citep{eriksson:analy:00,rodgers:00}, but again this theoretical guideline is
seldomly followed exactly. The normal choice is to set
$\aCvrMtr{o}=\aCvrMtr{\MsrErrVct_n}$. One important practical consideration is
that the inverse of \aCvrMtr{\MsrErrVct_o}\ must be calculated and this is a
highly costly operation in the general case, but \aCvrMtr{\MsrErrVct_n}\ is in
general a pure diagonal matrix and the inverse can be determined with a minimal
calculation cost. 

Further, the setting of \aCvrMtr{o}\ is influenced by the exact choice of
variables included in \SttVct. As discussed in Sec.~4.1.2 of \citet{rodgers:00}
there is no strict division line between \SttVct\ and \FrwMdlVct. Obviously,
\SttVct\ must contain a representation of the quantities targeted by the
measurements, but for variables of interfering character (e.g.\ frequency
off-set) there is a choice to make. For a perfectly linear retrieval case, exactly
the same results will be obtained for the target quantities independently if
interfering factors are covered by \SttVct\ or \FrwMdlVct. However, inclusion
in \SttVct\ is to prefer as this will result in that values describing the
interfering effects are obtained, which can provide important diagnostic
information (e.g.\ to detect instrumental changes). In addition, in a
non-linear situation a more optimal solution is obtained by placing interfering
factors in \SttVct, as it is likely that \aWfnMtr{\FrwMdlVct}, and then also
\aCvrMtr{\MsrErrVct_o} (cf.\ Eq.~\ref{eq:So}), vary with \SttVct. That is,
the description of observation uncertainties is likely to deteriorate with
differences between $\SttVct$ and $\aSttVct{a}$. If interfering species are part
of \SttVct, such non-linear effects will be taken care of by the reevaluation of 
\aWfnMtr{\SttVct}.


\subsection{Forward model}
\label{sec:setup:forward}
%
Some basic demands \todo{Replace "demands" with "requirements"?} on the forward model were introduced in Sec.~\ref{sec:rt}.
The simulation problem to be tackled is basically the simplest possible,
Eq.~\ref{eq:rteq} should be handled by all forward models for passive microwave
measurements (but limb sounding requires that refraction and the spherical shape of the Earth is considered when determining the propagation path, which excludes all
``plane-parallel'' models). On the other hand, the analysis of some data
requires non-standard forward model features: the 118\,GHz oxygen measurements give 
rise to a demand for treatment of Zeeman effects, and Doppler effects due to winds
could have a significant impact on mesospheric spectra.

Beside the atmospheric radiative transfer, the second main task of the forward
model is to incorporate effects of sensor characteristics. For the \smr\
radiometer package the sensor features needing consideration are antenna
angular response, mixer, sideband filtering and frequency response of each
spectrometer channel. A first order compensation of the Doppler shift due to
the satellite's movement is part of the L1b processing.

A further requirement is that the forward model can deliver the Jacobian
(\aWfnMtr{\SttVct}), a demand that should be clear from Eq.~\ref{eq:lindy}.
This task must be handled for all quantities that can be part of the state
vector. That is, the Jacobian is needed not only for the relevant atmospheric
variables (VMRs, temperature and winds), but also fitting of various sensor
imperfections must be handled (pointing off-set, frequency off-set and baseline
fit), see further Sec.~\ref{sec:x}.

In contrast to many other satellite missions \citep[e.g.][]{read2006clear}, no
\smr\ specific forward model has been developed. Instead, the general forward
model \ARTS\ (the Atmospheric radiative transfer simulator) is used for \smr\
L2 processing. Earlier operational L2 processing \citep{urban:odins:05-b} made
use of the first \ARTS\ version (and also the Moliere forward model
\citep{urban2004moliere}), while the retrievals described in this document are
based on the latest version. At this point it is just noted that \ARTS\ comply
fully with the demands listed above, and the presentation of \ARTS\ is
continued in Chapter~\ref{chapter:arts}.



\section{Non-standard retrievals}
\label{sec:nonstandard}
%
As mentioned already in Sec.~\ref{sec:aim}, the operational retrievals do not
handle all \smr\ measurements. Most importantly, scattering is not activated in
the forward model simulations and special retrievals have been developed to
extract information from spectra containing tropospheric information.
This retrieval scheme was introduced by \citet{rydberg:nonga:09} and the latest
results are found in \citet{eriksson:overw:14}. However, the method of
\citet{rydberg:nonga:09} is only applicable for tangent altitudes below 9\,km
and latitudes between 30$^\circ$S and 30$^\circ$N. This leaves a gap between
the operational and the tropical-tropospheric retrievals, and there are
still \smr\ spectra that are not part of any inversion. 

The operational processing is of 1D character (Sec.~\ref{sec:arts:config}), as
the standard limb scanning pattern of \smr\ does not provide a basis for
resolving along-track features. However, Odin has during some shorter periods
made measurements dedicated to polar mesospheric clouds. During these
measurement campaigns, Odin has just scanned over a narrow altitude region in
order to enhance the along-track resolution. Accordingly, to maintain the
measurement information of these special observations, 2D (or tomographic)
retrievals were needed. These retrievals were also made by combing \OEM\ and
\ARTS\ but are presented separately, in \citet{christensen:tomog:15}.



%%% Local Variables: 
%%% mode: latex
%%% TeX-master: "L2_ATBD"
%%% End: 
