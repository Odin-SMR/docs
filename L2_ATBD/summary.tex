\chapter{Summary}
\label{chapter:summary}


\lcomment{PE}{Keep text short. Mainly make a ``FAQ'', as sketched below.}

\dots\\

\noindent
Some key issues that should be considered when examining or applying Odin-SMR
L2 data:
\begin{itemize}
\item The retrievals assume a horisontally homogenous (1D) atmosphere.
\item The main atmospheric vertical coordinate is pressure, i.e.\ \VMR and
  temperature profiles are retrieved as a function of pressure. The L2 data
  contain also an estimate of the geometrical altitudes\fixme{Correct?}\
  matching the profiles values, but this information shall be considered as
  secondary.
\item L2 data are the best estimate at each pressure level\fixme{Does L2
    contain anything else than profiles?}. To obtain the value at other
  pressure levels, apply a linear interpolation using the logarithm of pressure
  as independent variable.
\item A single geographical position is assigned to the L2 data, that matches
  \dots \lcomment{PE}{Geo-position set how?} However, the data are far from
  point estimates, they represent averages of several 100\,km along the
  viewing direction of Odin-SMR, and not even the centre position of the
  averages is necessarily not found at the stated position. 
\end{itemize}



%%% Local Variables: 
%%% mode: latex
%%% TeX-master: "L2_ATBD"
%%% End: 
