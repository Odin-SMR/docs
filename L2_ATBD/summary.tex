\chapter{Summary}
\label{chapter:summary}


\lcomment{PE}{Keep text short. Mainly make a ``FAQ'', as sketched below.}

\dots\\

\noindent
Some key issues that should be considered when examining or applying \smr\
L2 data:
\begin{itemize}
\item The retrievals assume a horizontally homogeneous (1D), non-scattering,
  atmosphere.
\item The main atmospheric vertical coordinate is pressure, i.e.\ \VMR\ and
  temperatures are retrieved as a function of pressure. The L2 data
  contain also an estimate of the geometrical altitudes\fixme{Correct?}\
  matching the profiles values, but this information shall be considered as
  secondary.
\item Retrieved profiles shall be considered as running average estimates of
  the true profile, with an averaging length roughly given by the vertical
  resolution.
\item Errors refer to the accuracy of running averages, and are only valid on
  the constrain of high measurement response. Using data with low measurement
  response requires consideration of smoothing effects.
\item The retrievals are best estimates at each pressure level\fixme{Does L2
    contain anything else than profiles?}. To obtain values at other
  pressure levels, apply a linear interpolation using the logarithm of pressure
  as independent variable.
\item A single geographical position is assigned to the L2 data, that matches
  \dots \lcomment{PE}{Geo-position set how?} However, the data are far from
  point estimates in the horizontal direction, they represent averages of
  several 100\,km along the viewing direction of \smr, and not even the
  centre position of the averages is necessarily not found at the stated
  position.
\end{itemize}



%%% Local Variables: 
%%% mode: latex
%%% TeX-master: "L2_ATBD"
%%% End: 
