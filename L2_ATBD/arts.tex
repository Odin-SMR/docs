\chapter{The ARTS forward model}
\label{chapter:arts}


\section{General features}
\label{sec:arts:features}

\subsection{Main features}
%
\ARTS, the Atmospheric radiative transfer simulator, is a publicly available
software that is both developed and maintained as an open source project. The
software was initiated 1999, and the first official version was released some
years later \citep{buehler:artst:05}. The main novelty of this version was the
fact that \ARTS\ behaves more or less as a scripting language, in contrast to
most other forward models that are controlled by fixed set of keywords. The
control file was introduced to meet the design goal of high modularity. When
defining a \ARTS\ control file, the user can select between a high number of
workspace methods and variables. Regarding physical mechanisms covered, \ARTS\
v1 did not stand out in any way, but despite this the software found a number
of applications, including operational \SMR\ processing
(Sec.~\ref{sec:setup:forward}).

Already before the first version was released, a development branch was started
with the aim of adding more advanced features to \ARTS. At the time, treatment
of scattering and going beyond horizontally homogeneous atmospheres (1D), were
the main aims. This development resulted in a second main version \ARTS\
\citep{eriksson:arts2:11}, and further extensions have been made more recently.
\ARTS\ can today handle several observation techniques, as well as simulations
for other planets, but only aspects relevant for \smr\ are discussed below.
\smr\ measures thermal emission in a limb sounding geometry and, in this
context, the most important features of \ARTS\ are:
\begin{itemize}
\item The full polarisation state of radiation is described, by the Stokes
  formalism.
\item Atmospheric fields can be defined to vary in one dimension (1D;
  pressure), two dimensions (2D; pressure and angle along the orbit) or three
  dimensions (3D; pressure, latitude and longitude).
\item The planet's overall shape is spherical (also spheroidal allowed for 2D
  and 3D), there are no assumptions on a ``flat Earth''.
\item Gaseous absorption can be calculated based on several spectroscopic
  databases and a high number of absorption parameterisations.
\item The impact of the magnetic field on oxygen absorption due to the Zeeman
  effect can be considered \citep{larsson:zeema:14}.
\item Refraction can be considered.
\item There is extensive support to incorporate sensor characteristics.
\item The Jacobian can be obtained for a high number of atmospheric and sensor
  variables, in general with limited additional calculation time (only valid
  for non-scattering calculations).
\item Scattering can be incorporated by two different algorithms, denoted as DOIT
  \citep{emde:apoli:04} and MC \citep{davisetal:04}.
\end{itemize}



\subsection{Access and documentation}
%
The \ARTS\ website has the address: \url{www.radiativetransfer.org}. Version 1
has been declared obsolete and is now limited to internal use, while download
instructions for recent public releases and the development branch are found
at \url{www.radiativetransfer.org/getarts}. Some external packages are needed,
generally or for special features, and these are listed at
\url{www.radiativetransfer.org/tools}.

To obtain an overview of \ARTS\ as a forward model, the primary reading should
be \citet{buehler:artst:05} and \citet{eriksson:arts2:11}. A more practically
inclined introduction to \ARTS\ is provided by the ``ARTS user guide''. Some
background theory is found in ``ARTS theory'' (but this shall not be taken as a
complete treatment of atmospheric radiative transfer), and some parts of \ARTS\
are even described by dedicated journal articles (see references throughout
this chapter). However, the core documentation of \ARTS\ is the built-in
description of workspace methods and variables. This is the main information
source for more experienced users and this documentation for the present stable
version can be browsed here: \url{www.radiativetransfer.org/docserver-stable}.
Finally, the best way to get started with practical calculations is to copy and
modify some of the test and demonstration control files distributed with the
software. For further details and links to the ARTS user guide and other
documents, see \url{www.radiativetransfer.org/docs}.



\subsection{Validation and quality assurance}
%
\ARTS\ has been compared to other forward models by
\citet{melsheimer:inter:05}, \citet{buehler2006radiative} and
\citet{saunders07:_atmos_infrar_sound_airs_jgr}, as well as in other ways
lacking official documentation. A number of test cases has been defined for
\ARTS, targeting either a special part of \ARTS\ or a complete calculation of
some type. These tests are run at each new commit of code to the software
repository, in order to catch bugs and unexpected side-effects of code changes
as early as possible. As \ARTS\ is a very flexible tool, it is impossible to
design a complete set of tests, but standard applications of \ARTS\ should be
fairly well covered by the tests. Some of the tests include comparison to a
stored set of reference values, to test that not only the code runs but also
that the actual simulation results have not changed above some specified
tolerance. 

\lcomment{PE}{One of the present tests mimic \smr, but it should be updated and
  make use of reference values, to ensure safety when updating to a new \ARTS\
  version for the operational processing. Also the L2 processing chain should
  have an internal quality check. How?}


\section{Configuration used}
\label{sec:arts:config}

\subsection{\ARTS\ version and compilation options}
%
The processing chain contains a check that a specific version of \ARTS\ is
used. The presently used version is ARTS 2.3.?\todo{Add number later.}. \ARTS\
is written in C++, and a compilation for each platform is needed. The
\texttt{cmake} build system is used to set up the compilation of \ARTS. A
minimal version of \ARTS\ is created, using the following options:
\begin{verbatim}
  cmake -DCMAKE_BUILD_TYPE=Release -DNO_NETCDF=1 -DNO_DOCSERVER=1 ..
\end{verbatim}
\lcomment{PE}{Finish this section later. During the development phase we should
just use latest version of the development branch. Later we should select a
specific version and add check in the code that exactly this version is used.
The executable should be minimal ARTS (no DISORT, T-Matrix ...). What level
optimisation should be used? Shall asserts be turned off?}


\subsection{Basic settings\todo{Add all missing section refs.}}
%
The atmosphere is set to be 1D. That is, atmospheric fields vary only as a
function of pressure, and are treated as constant in the latitude and longitude
dimensions. A 1D atmosphere implies a speherical reference geoid, and the
radius of the this geiod is set ?\todo{Fixed or as curvature radius?}. The
upper limit of the model atmosphere is in \ARTS\ defined by the lowest value in
the pressure grid. The range of this grid varies, see Sec.~?. The lower limit
of the atmosphere is determined by the planet's surface, and \ARTS\ demands
that a surface is always defined. The altitude of the surface is for these
simulations set to ?\,km\todo{Add number}, and regarding radiative properties
it is set to behave as a blackbody.

It is possible to use a ``dummy'' surface for the following reason. Even if
data with (bore-sight) tangent altitudes below the tropopause are rejected, the
outer parts of the antenna pattern can result in propgation paths that
theoretically intersect with the surface, but the net impact of the surface is
still zero. This is the case due to the very high optical thickness of the
middle and lower troposphere at frequencies around 500\,GHz
\citep{ekstrom:first:07}. Even for tangent altitudes around 0\,km\ only
altitudes above $\sim$8\,km effectively contributes to measured radiance.

The atmosphere is assumed to not cause scattering, which is a valid assumption
for the wavelengths of \smr\ and tangent altitudes some kilometres above the
tropopause. 


\subsection{Gaseous absorption}
%
In lack of scattering and assuming LTE, the local radiative properties of the
atmosphere is solely determined by the absorption coefficient. Molecular
absorption cross-sections are pre-calculated and stored in so called absorption
look-up tables. One table is created for each frequency band. The tables
contain absorption cross-sections for combinations aof pressure and
temperature, and final values are obtained by interpolation, all following

\citet{buehler:absor:11}.


\subsection{Radiative transfer}

Eq.~\ref{eq:rteq}  stokes

\subsection{Sensor characteristics and calibration}
%






\section{Important aspects}
\label{sec:arts:aspects}
%
\dots (Pressure as vertical coordinate, basis functions)





%%% Local Variables: 
%%% mode: latex
%%% TeX-master: "L2_ATBD"
%%% End: 
