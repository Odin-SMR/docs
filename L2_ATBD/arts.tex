\chapter{The ARTS forward model}
\label{chapter:arts}


\section{General features}
\label{sec:arts:features}

\subsection{Main features}
%
\ARTS, the Atmospheric radiative transfer simulator, is a publicly available
software that is both developed and maintained as an open source project. The
programming language used is C++. The software was initiated 1999, and the
first official version was released some years later \citep{buehler:artst:05}.
The main novelty of this first version was the fact that \ARTS\ behaves more or
less as a scripting language, in contrast to most other forward models that are
controlled by fixed set of keywords. The control file was introduced to meet
the design goal of high modularity, that was judged needed to provide users
with a flexible simulation platform and to facilitate for developers to add new
features. When defining a \ARTS\ control file, the user can operate with a high
number of workspace methods and variables. Regarding physical mechanisms
covered, \ARTS\ v1 did not stand out in any way, but despite this the software
found a number of applications, including operational \SMR\ processing
(Sec.~\ref{sec:setup:forward}).

Already before the first version was released, a development branch was started
with the aim of adding more advanced features to \ARTS. At the time, treatment
of scattering and going beyond horizontally homogeneous atmospheres (1D) were
the main aims. This development resulted in a second main version \ARTS\
\citep{eriksson:arts2:11}, and further additions have also been made more
recently. Today \ARTS\ can handle several observation techniques, and also
simulations for other planets are possible, but only aspects relevant for
\smr\ are discussed below. \smr\ measures thermal emission in a limb sounding
geometry and, in this context, the most important features of \ARTS\ are:
\begin{itemize}
\item The full polarisation state of radiation is described, by the Stokes
  formalism.
\item Atmospheric fields can be defined to vary in one dimension (1D,
  pressure), two dimensions (2D, pressure and angle along the orbit) or three
  dimensions (3D, pressure, latitude and longitude).
\item The planet's overall shape is spherical (also spheroidal allowed for 2D
  and 3D), there are no assumptions on a ``flat Earth''.
\item Gaseous absorption can be calculated based on several spectroscopic
  databases and a high number of absorption parameterisations.
\item There is extensive support to incorporate sensor characteristics.
\item Scattering can be incorporated by two different algorithms, called DOIT
  \citep{emde:apoli:04} and MC \citep{davisetal:04}.
\item The impact of the magnetic field on oxygen absorption due to the Zeeman
  effect can be considered \citep{larsson:zeema:14}.
\end{itemize}



\subsection{Access and documentation}
%
The \ARTS\ website has the address: \url{www.radiativetransfer.org}. Version 1
is declared obsolete and is now not public available, while how most recent
public releases and the development branch can be obtained is described at
\url{www.radiativetransfer.org/getarts}. Some external packages are needed,
generally or for special features, and these are listed at
\url{www.radiativetransfer.org/tools}.

To obtain an overview of \ARTS, the primary reading should be the articles
\citet{buehler:artst:05} and \citet{eriksson:arts2:11}. The \ARTS\ software is
described more in detail by the ``ARTS user guide''. Some background theory is
found in ``ARTS theory'' (but this shall not be taken as a complete treatment
of atmospheric radiative transfer), and some parts of \ARTS\ are even described
by dedicated journal articles (see references throughout this chapter).

The best introduction to the actual practical usage of \ARTS\ is probably the
test and demonstration control files distributed with the software. However,
the core documentation of \ARTS\ is the built-in description of workspace
methods and variables. This is the main information source for more experienced
users and this documentation for the present stable version can be browsed
here: \url{www.radiativetransfer.org/docserver-stable}. For further details and
links to the ARTS user guide and other documents, see
\url{www.radiativetransfer.org/docs}.



\subsection{Validation and quality assurance}
%
\ARTS\ has been compared to other forward models by
\citet{melsheimer:inter:05}, \citet{buehler2006radiative} and
\citet{saunders07:_atmos_infrar_sound_airs_jgr}, as well as by other less well
documented efforts. A number of test cases has been defined for \ARTS, targeting
either a special part of \ARTS\ or a complete calculation of some type. These
tests are run at each new commit of code to the software repository, in order
to catch bugs and unexpected side-effects of code changes as early as possible.
As \ARTS\ is a very flexible tool, it is impossible to design a complete set of
tests, but standard applications of \ARTS\ should be fairly well covered by the
tests. Some of the tests include comparison to a stored set of reference
values, to test that not only the code runs but also that the actual simulation
results have not changed above some specified tolerance. 

\lcomment{PE}{One of the present tests mimic \smr, but it should be updated and
  make use of reference values, to ensure safety when updating to a new \ARTS\
  version for the operational processing. Also the L2 processing chain should
  have an internal quality check. How?}


\section{Configuration used}
\label{sec:arts:config}


\subsection{\ARTS\ version and compilation options}
%
The processing chain contains a check that a specific version of \ARTS\ is used.
The presently used version is ARTS 2.3.?\todo{Add number}. \ARTS\ is configured
for compilation using \texttt{cmake}, in the following manner:
\begin{verbatim}
  cmake ???
\end{verbatim}
\lcomment{PE}{Finish this section later. During the developemnt phase we should
just use latest version of the development branch. Later we should select a
specific version and add check in the code that exactly this version is used.
The executable should be minimal ARTS (no DISORT, T-Matrix ...). What level
optimisation should be used? Shall asserts be turned off?}


\subsection{Atmospheric radiative transfer}
%

$I_0$

\subsection{Gaseous absorption}
%


\subsection{Sensor characteristics and calibration}
%






\section{Important aspects}
\label{sec:arts:aspects}
%
\dots (Pressure as vertical coordinate, basis functions)





%%% Local Variables: 
%%% mode: latex
%%% TeX-master: "L2_ATBD"
%%% End: 
