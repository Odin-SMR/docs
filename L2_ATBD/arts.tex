\chapter{The ARTS forward model}
\label{chapter:arts}


\section{General features}
\label{sec:arts:features}

\subsection{Main features}
%
\ARTS, the Atmospheric radiative transfer simulator, is a publicly available
software that is both developed and maintained as an open source project. The
software was initiated 1999, and the first official version was released some
years later \citep{buehler:artst:05}. The main novelty of this version was the
fact that \ARTS\ behaves more or less as a scripting language, in contrast to
most (or all?) other forward models that are controlled by a fixed set of
keywords. The control file was introduced to meet the design goal of high
modularity. When defining an \ARTS\ control file, the user can select between a
high number of workspace methods and variables. Regarding physical mechanisms
covered, \ARTS\ v1 did not stand out in any way, but despite this the software
found a number of applications, including operational \SMR\ processing
(Sec.~\ref{sec:setup:forward}).

Already before the first version was released, a development branch was started
with the aim of adding more advanced features to \ARTS. At the time, treatment
of scattering and going beyond horizontally homogeneous atmospheres (1D), were
the main aims. This development resulted in a second main version \ARTS\
\citep{eriksson:arts2:11}, and further extensions have been made more recently.
\ARTS\ can today handle several observation techniques, as well as simulations
for other planets, but only aspects relevant for \smr\ are discussed below.
\smr\ measures thermal emission in a limb sounding geometry and, in this
context, the most important features of \ARTS\ are:
\begin{itemize}
\item The full polarisation state of radiation can by described, by the Stokes
  formalism.
\item Atmospheric fields can be defined to vary in one dimension (1D;
  pressure), two dimensions (2D; pressure and angle along the orbit) or three
  dimensions (3D; pressure, latitude and longitude).
\item The planet's overall shape is spherical (also spheroidal allowed for 2D
  and 3D), there are no assumptions on a ``flat Earth''.
\item Gaseous absorption can be calculated based on several spectroscopic
  databases and a high number of absorption parameterisations.
\item The impact of the magnetic field on oxygen absorption due to the Zeeman
  effect can be considered \citep{larsson:zeema:14}.
\item Refraction can be considered.
\item There is extensive support to incorporate sensor characteristics.
\item The Jacobian can be obtained for a high number of atmospheric and sensor
  variables, in general with limited additional calculation time (but only
  for non-scattering calculations).
\item Scattering can be incorporated by two different algorithms, denoted as DOIT
  \citep{emde:apoli:04} and MC \citep{davisetal:04}.
\end{itemize}



\subsection{Access and documentation}
%
The \ARTS\ website is \url{www.radiativetransfer.org}. Version 1
has been declared obsolete and is now limited to internal use, while downloading
instructions for recent public releases and the development branch are found
at \url{www.radiativetransfer.org/getarts}. Some external packages are needed,
generally or for special features, and these are listed at
\url{www.radiativetransfer.org/tools}.

To obtain an overview of \ARTS\ as a forward model, the primary reading should
be \citet{buehler:artst:05} and \citet{eriksson:arts2:11}. A more practically
inclined introduction to \ARTS\ is provided by the ``ARTS user guide'' (AUG).
Some background theory is found in ``ARTS theory'', and some parts of \ARTS\
are even described by dedicated journal articles (see references throughout
this chapter). However, the core documentation of \ARTS\ is the built-in
description of workspace methods and variables. This is the main information
source for more experienced users and this documentation for the present stable
version can be browsed at: \url{www.radiativetransfer.org/docserver-stable}.
Finally, the best way to get started with practical calculations is to copy and
modify some of the test and demonstration control files distributed with the
software. For further details and links to the ARTS user guide and other
documents, see \url{www.radiativetransfer.org/docs}.



\subsection{Validation and quality assurance}
%
\ARTS\ has been compared to other forward models by
\citet{melsheimer:inter:05}, \citet{buehler2006radiative} and
\citet{saunders07:_atmos_infrar_sound_airs_jgr}, as well as in more informal
manners. A number of test cases has been defined for \ARTS, targeting either a
special part of \ARTS\ or a complete calculation of some type. These tests are
run at each new commit of code to the software repository, in order to catch
bugs and unexpected side-effects of code changes as early as possible. As
\ARTS\ is a very flexible tool, it is impossible to design a comprehensive set
of tests, but standard applications of \ARTS\ should be fairly well covered by
the tests. Some of the tests include comparison to a stored set of reference
values, to test that not only the code runs but also that the actual simulation
results have not changed above some specified tolerance.

\lcomment{PE}{One of the present tests mimic \smr, but it should be updated and
  make use of reference values, to ensure safety when updating to a new \ARTS\
  version for the operational processing. Also the L2 processing chain should
  have an internal quality check. How?}


\section{Configuration used}
\label{sec:arts:config}

\subsection{\ARTS\ version and compilation options}
%
The processing chain contains a check that a specific version of \ARTS\ is
used. The version used presently is ARTS 2.3.?\todo{Add number later.}. \ARTS\
is written in C++, and a compilation for each platform architecture is needed. The
\texttt{cmake} build system is used to set up the compilation of \ARTS. A
minimal version of \ARTS\ is created, using the following options:
\begin{verbatim}
  cmake -DCMAKE_BUILD_TYPE=Release -DNO_NETCDF=1 -DNO_DOCSERVER=1 ..
\end{verbatim}
\lcomment{PE}{Finish this section later. During the development phase we should
just use latest version of the development branch. Later we should select a
specific version and add check in the code that exactly this version is used.
The executable should be minimal ARTS (no DISORT, T-Matrix ...). What level
optimisation should be used? Shall asserts be turned off?}


\subsection{Basic settings}
%
The atmosphere is set to be 1D. That is, atmospheric fields vary only as a
function of pressure, and are implicitly treated as constant in the latitude
and longitude dimensions. A 1D atmosphere implies further a spherical reference
geoid, and the radius of the this geoid is set to ?\todo{Fixed or as curvature
  radius?}. The upper limit of the model atmosphere is in \ARTS\ defined by the
lowest value in the user defined pressure grid. The range of this grid varies,
see Sec.~\ref{sec:b:pgrid}. The lower limit of the atmosphere is determined by
the planet's surface, and \ARTS\ demands that a surface is always defined. The
altitude of the surface is for these simulations set to ?\,km\todo{Add number},
and regarding radiative properties it is set to behave as a blackbody.

It is possible to use a ``dummy'' surface for the following reason. Even if
data with (bore-sight) tangent altitudes below the tropopause are rejected, the
outer parts of the antenna pattern can result in propagation paths that
theoretically intersect with the surface, but the net impact of the surface is
still zero. This is the case due to the very high optical thickness of the
middle and lower troposphere at frequencies around 500\,GHz
\citep{ekstrom:first:07}. Even for tangent altitudes around 0\,km\ only
altitudes above $\sim$8\,km effectively contributes to measured radiance.

The atmosphere is assumed to not cause scattering, which is a valid assumption
for the wavelengths of \smr\ and tangent altitudes some kilometres above the
tropopause. 


\subsection{Gaseous absorption}
%
In lack of scattering and assuming LTE, the local radiative properties of the
atmosphere is solely determined by the absorption coefficient. \ARTS\ has two
main ways of calculating absorption, ``on-the-fly'' or by look-up tables, where
the latter option is by far the most efficient for simulations of \smr. That
is, molecular absorption cross-sections are pre-calculated and stored in
tables. One table is created for each frequency band. The tables contain
absorption cross-sections for combinations of pressure and temperature, and
final values are obtained by interpolation. It is throughout assumed that the
absorption cross-sections do not vary with the mixing ratio of the gas, and that
there are no ``non-linear'' species in the absorption table (this option is
required for water vapour in the lower troposphere).
\ARTS' absorption look-up table system is described in detail by
\citet{buehler:absor:11}. The spectroscopic data used to produce the look-up
tables are discussed in Sec.~\ref{sec:b:absdata}).


\subsection{Radiative transfer}
%
Besides for 118\,GHz (see below), the simulations follow Eq.~\ref{eq:rteq}. Only
the first element of the Stokes vector is calculated (remaining three elements
are zero for assumed conditions).

Solving equation Eq.~\ref{eq:rteq} requires two main steps, to determine the
propagation path and to perform the integration along the found path.
Refraction must be considered for \smr\ \citep{eriksson:studi:02}, which is
achieved by a simple ray tracing scheme (see AUG). The maximum step length of
the ray tracing is set to ?\,km\todo{Add value}, which is estimated to give a
tangent point accuracy better than ?\,m\todo{Add value}\ for tangent altitudes
above 15\,km\todo{Perform test}.

The radiance at the top of the atmosphere, $I_0$ is set to mimic cosmic
background radiation. There is no attempt to model situations when the moon is
found in the LOS, the retrieval reject such spectra. The combination of Odin
orbit and keeping the LOS close to the orbit plane results in that there is
never a need to consider the Sun.



\subsection{118\,GHz}
%
The standard absorption and radiative transfer settings cannot be used for
simulations targeting the 118\,GHz oxygen transition, as the Zeeman effect must
be considered. First of all, a matrix-vector version Eq.~\ref{eq:rteq} must be
applied \citep{larsson:zeema:14}, where the full Stokes vector is calculated.
Further, absorption must be calculated ``on-the-fly'' as the look-up table
approach does not handle polarised absorption. However, even with an extension
of the lookup-tables, the ``on-the-fly'' option would still be to prefer as in
this case only the 118\,GHz transition itself must be considered (in contrast
to the 0.6\,mm frequency bands where hundreds of transitions must be summed
up) and a full calculation turns out to be faster than performing a
multi-dimensional interpolation.


\subsection{Sensor characteristics and calibration}
%
The impact of the various sensor responses is incorporated following
\citet{eriksson:06}. Also in this approach calculation time is saved by
performing pre-calculations, but, in contrast to the absorption tables, no
post-processing is needed. In short, the impact of the different sensor parts
is all linear operations and a ``response matrix'' representing the complete
receiver system can be determined. This calculation can be done before the
atmospheric radiative transfer has been performed, as long as the calculation
grids are set and the description of the sensor responses are at hand. The
response matrix obtained, \SnsMtr, is applied as:
\begin{equation}
  \label{eq:H}
  \aMsrVct{f} = \SnsMtr\MpiVct
\end{equation}
where \aMsrVct{f}\ is the measurement vector produced by the forward
model and \MpiVct\ is all monochromatic pencil beam radiances compiled into a
vector. The vector \MpiVct\ must contain a certain number of values to
represent the radiation field, as a function of frequency and zenith angle,
with sufficient accuracy. The methods setting up \SnsMtr\ assumes that both
radiances and sensor responses vary linearly between the points were data are is available.\todo{Will the FillGrid method be used?}\ The variables controlling
\SnsMtr\ and \MpiVct\ are discussed in Sec.~?\todo{Add value}.

The conversion of the \smr's L1b data to brightness temperatures ($T_b$, K)
must also be mimicked by the forward model. The radiances ($I$,
W/(m$^2\cdot$Hz$\cdot$sr)) given by the atmospheric radiative transfer part are
converted to follow the Rayleigh-Jeans approximation of the Planck function:
\begin{equation}
  \label{eq:i2tb}
  T_b = \frac{c^2}{2\nu k_B} I,
\end{equation}
where $c$ is the speed of light and $k_B$ is the Boltzmann constant. This
conversion is made before the sensor responses are applied.



\section{\ARTS\ and weighting functions}
\label{sec:arts:wfuns}
%
\lcomment{PE}{The Jacobian probably deserves special attention, as it is the
  least well established part among forward models. Discuss how ARTS calculates
  weighting functions. First of all, this is done quite efficiently, less than
  doubling of the calculation time compared to a just calculating \MsrVct. This
  due to analytic expressions and \SnsMtr. With absorption look-up table, the
  derivative of absorption with respect to temperature is done by perturbation.
  And temperature Jacobian does not consider non-local effects due to
  refraction and hydrostatic equilibrium. The pointing Jacobian could be used
  to make a first order correction for these simplifications, but this not yet
  tested.}



\section{Important aspects}
\label{sec:arts:aspects}
%
Some of the definitions of \ARTS\ have noticeable implications for the
L2 processing:
\begin{itemize}
\item The basic vertical coordinate in \ARTS\ is pressure. This has the
  consequence that the retrieval grid for atmospheric quantities is a set of
  pressures, and that retrieved profiles are functions of pressure.
\item With some exceptions of no concern here, \ARTS\ assumes data to vary
  linearly between grid points. This applies to everything from sensor responses
  (such as the input antenna pattern file) to how retrieved data shall be
  interpreted. However, for the pressure dimension the variation is assumed to
  be linear in the logarithm of the pressure. That is, if a retrieved
  atmospheric profile shall be interpolated to another set of pressures, it
  should be done as (using Matlab notation):
\begin{verbatim}
  x_new = interp1( log(p_grid), x, log(p_grid_new) )
\end{verbatim}
where \verb|p_grid| is the original pressure grid, \verb|x| is the atmospheric
profile to be interpolated, and \verb|p_grid_new| is the pressure grid.
\item \lcomment{PE}{Should be more to add here}
\end{itemize}



%%% Local Variables: 
%%% mode: latex
%%% TeX-master: "L2_ATBD"
%%% End: 
