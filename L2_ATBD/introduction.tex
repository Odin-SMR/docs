\chapter{Introduction}
\label{chapter:introduction}

% The page numbering must be reset here inside the file
\pagenumbering{arabic}
\setcounter{page}{1}


\section{Aim and scope of this document}
\label{sec:aim}
%
\dots


\section{Odin-SMR}
\label{sec:odin}
%
The Odin satellite was launched on 20 February 2001, into a sun-synchronous 18:00 hour
ascending node orbit, carrying two co-aligned limb sounding instruments: OSIRIS
(Optical Spectrograph and Infrared Imaging System) and SMR (Sub-Millimetre
Radiometer). Originally, Odin was used for both atmospheric and astronomical
observations, while since 2007 only its aeronomy mission
\citep{murtagh:anove:02} is active. Odin is a Swedish-led project, in
cooperation with Canada, France and Finland. Both of Odin's instruments are
still functional, and the present operation of the satellite is partly
performed as a ESA third party mission. 

The SMR package is highly tunable and flexible. In short, the four main
receiver chains can be tuned to cover frequencies between 486\,GHz -- 504\,GHz and
541\,GHz -- 581\,GHz, but the maximum total instantaneous bandwidth is only
1.6\,GHz. This bandwidth is determined by the two autocorrelation (AC)
spectrometers used for atmospheric observations. The two ACs can be coupled to
any of the four front-ends, but only two or three front-ends are used
simultaneously. The ACs cover 400 or 800\,MHz per front-end, depending on
configuration, and are used with a frequency resolution of 2\,MHz. To cover all
molecular transitions of interest, a number of ``observation modes'' has been
defined.

Single sideband operation is obtained by tunable Martin-Pupplet
interferometers. The nominal sideband suppression is better than 19 dB across
the image band. The main reflector of SMR has a diameter of 1.1\,m, giving a
vertical resolution at the tangent point of about 2\,km. The vertical scanning
of the two instruments' line-of-sigh is achieved by a rotation of the satellite
platform, with a rate matching a vertical speed of the tangent altitude of
750\,m/s. Measurements are performed during both upward and downward scanning.
The lower end of the scan is typically at about 7\,km, the upper end varies
between 70\,km and 110\,km, depending on observation mode. In correspondence
the horizontal sampling ranges from 1 scan per 600 km to 1 scan per 1000 km.
Measurements are in general performed along the orbit plane, providing a
latitude coverage between 82.5$^{\circ}$S and 82.5$^{\circ}$N. Since the end of
2004 Odin is also pointing off-track during certain periods, e.g.\ during the
austral summer season, allowing the latitudinal coverage to be extended towards
the poles. 


\section{Further reading}
\label{sec:reading}
%
\dots For further basic technical information about SMR, see
\citet{eriksson:studi:02}, \citet{merino:studi:02} and
\citet{murtagh:anove:02}.






%%% Local Variables: 
%%% mode: latex
%%% TeX-master: "L2_ATBD"
%%% End: 
