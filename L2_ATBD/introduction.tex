\chapter{Introduction}
\label{chapter:introduction}

% The page numbering must be reset here inside the file
\pagenumbering{arabic}
\setcounter{page}{1}


\section{Aim and scope of this document}
\label{sec:aim}
%
\smr\ performs passive measurements of the atmosphere, mainly at wavelengths
around 0.6\,mm. The basic output of \smr\ is spectra in different frequency
bands. After calibration the spectra are denoted as L1b data, and are grouped
into limb scans for further processing. The overall aim of this document is to
describe the software and algorithms used for the L2 processing, that is, the
step of extracting geophysical data from L1b data. The aim includes also to
motivate the different decisions involved, such as what instrumental parameters
that shall be retrieved in parallel to the geophysical data to provide best
possible L2 data product. The document describes the retrieval process and
related questions in a general manner, while details and information that
varies between frequency bands are found elsewhere (see
Sec.~\ref{sec:reading}).

The document covers only the ``operational'' \smr\ L2 products. These products
are based on measurements confined to tangent altitudes above the tropopause,
with a margin of a few kilometres\todo{How will we set this limit?}, to
allow that scattering can be neglected in the simulations of atmospheric
radiative transfer. Observations that can be affected by scattering (due to
tropospheric clouds) are treated separately, and the same applies to data
measured when Odin has been performed a special scanning sequence (see further
Sec.~\ref{sec:nonstandard}). The result of these retrievals are formally also
L2 products, but inside this document the term ``L2'' refers only to the
operational products.

An overview of the \smr\ L2 processing is found Chapter~2. The software tool
that handles the tasks directly associated with atmospheric radiative transfer
and sensor modelling is described in Chapter~3, while the actual data
extraction step is covered by Chapter~4. The L2 data are not complete without a
careful characterisation of spatial resolution and errors, which is the topic
of Chapter~5. Finally, Chapter 6 gives a summary, with focus on the most
important points to correctly understand the \smr\ L2 data. A list of used
acronyms is found directly after the table of contents.




\section{\smr}
\label{sec:odin}
%
\subsection{The Odin satellite}
%
The Odin satellite was launched on 20 February 2001, into a sun-synchronous
18:00 hour ascending node orbit, carrying two co-aligned limb sounding
instruments: OSIRIS (optical spectrograph and infrared imaging system) and
\SMR\ (sub-millimetre radiometer). Originally, Odin was used for both
atmospheric and astronomical observations, while since 2007 only its aeronomy
mission is active. Odin is a Swedish-led project, in cooperation with Canada,
France and Finland. Both of Odin's instruments are still functional, and the
present operation of the satellite is partly performed as a ESA third party
mission.


\subsection{The \SMR\ instrument}
%
The \smr\ package is highly tunable and flexible. In short, the four main
receiver chains can be tuned to cover frequencies in the ranges
486\,--\,504\,GHz and 541\,-- \,581\,GHz, but the maximum total instantaneous
bandwidth is only 1.6\,GHz. This bandwidth is determined by the two
auto-correlation spectrometers (ACs) used for atmospheric observations. The two
ACs can be coupled to any of the four front-ends, but only two or three
front-ends are used simultaneously. The ACs cover 400 or 800\,MHz per
front-end, depending on configuration. In the configuration applied for
atmospheric sounding, the channels of the ACs have a spacing of 1\,MHz, while
the frequency resolution is only 2\,MHz. To cover all molecular transitions of
interest, a number of ``observation modes'' has been defined. Each observation
mode makes use of two or three frequency bands. Single sideband operation is
obtained by tunable Martin-Pupplet interferometers. The nominal sideband
suppression is better than 19\,dB across the image band.

\smr\ has also a receiver chain around the 118\,GHz oxygen transition, that was
heavily used during Odin's astronomy mission. For the atmospheric mission, this
front-end was planned to be used for retrieving temperature profiles, but a
technical problem (drifting LO frequency) and the fact that the analyses
requires treatment of Zeeman effects have given these data low priority and are
not yet processed properly.

The main reflector of \smr\ has a diameter of 1.1\,m, giving a
vertical resolution at the tangent point of about 2\,km. The vertical scanning
of the two instruments' line-of-sigh is achieved by a rotation of the satellite
platform, with a rate matching a vertical speed of the tangent altitude of
750\,m/s. Measurements are performed during both upward and downward scanning.
The lower end of the scan is typically at about 7\,km, the upper end varies
between 70 and 110\,km, depending on observation mode. In correspondence,
the horizontal sampling ranges from 1 scan per 600 km to 1 scan per 1000 km.
Measurements are in general performed along the orbit plane, providing a
latitude coverage between 82.5$^{\circ}$S and 82.5$^{\circ}$N. Since the end of
2004 Odin is also pointing off-track during certain periods, e.g.\ during the
austral summer season, allowing the latitudinal coverage to be extended towards
the poles. 


\subsection{Main instrumental error sources}
%
The receiver noise temperature of \smr\ differs between the front-ends, varying
between ? and ?\,K\todo{Add values}. This makes thermal noise an important
error source. The pointing of Odin can be reconstructed to an accuracy matching
?\,m in tangent altitude, and observed frequencies can be controlled with an
accuracy better than 2\,MHz\todo{Or will we skip Hanning?}. The retrieval
process includes both a pointing and frequency correction, but there will
be some remaining errors.

The three error sources discussed above basically follow nominal performance
and are well characterised, but there are also features that are not yet
understood. First of all, it is clear that the sideband suppression does not
reach the nominal value of 19\,dB in all frequency bands. Ongoing tests and
investigations aim to obtain better knowledge on the actual suppression achieved.
Further, the sub-bands of the ACs are not always overlapping perfectly, and
some degree of non-linear response can not be ruled out. Both these issues are
most problematic when measuring spectral features giving a high range of
brightness temperatures over the frequency band.

There are also non-nominal features that are handled, or characterised, fairly
well. Reflections inside the receiver give rise to ``baseline ripple'', but
these ripples are largely removed as part of the L1b processing\addref. The
calibration uncertainty increases linearly with measured brightness
temperature, and is presently estimated to not exceed 2\,K\todo{Correct?}\ at
220\,K. \smr\ applies Dicke switching with respect to ``cold sky'' but there is
some remaining impact of gain variations. These variations give rise to a
constant shift of the brightness temperatures across the band (that is, it can
be seen as a flat baseline ripple), where the shift has a standard deviation of
about 2\,K and is uncorrelated between tangent altitudes (and front-ends). The
retrieval can remove these shifts with marginal impact on the retrievals as
long as some part of the measured spectrum corresponds to cosmic background
radiation (that is, high tangent altitudes), but this is a main error source
for other situations.



\section{Further reading}
\label{sec:reading}
%
\citet{murtagh:anove:02} give an overview of the Odin aeronomy mission, as well
as the general technical details of \smr. Somew further technical information
is found in e.g.\ \citet{eriksson:studi:02}, and a more detailed description of
the observation modes is provided by \citet{merino:studi:02}.\todo{Missed
  something?}

The input to this L2 processing is the L1b data described in ?\addref. The
exact settings applied and the general characteristics of the L2 data, that
both vary between the frequency bands, are described in ?\addref.
\lcomment{PE}{Probably more to say here ...}




%%% Local Variables: 
%%% mode: latex
%%% TeX-master: "L2_ATBD"
%%% End: 
