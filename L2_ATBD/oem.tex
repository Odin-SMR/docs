\chapter{OEM implementation}
\label{chapter:oem}

The optimal estimation method (\OEM) is introduced generally in
Sec.~\ref{sec:setup:inverse}, while details more specific for the \smr\ L2
processing are treated in this section.
\\
\lcomment{PE}{It is for the moment assumed that the general \OEM\ function of
  Atmlab will be used. If the \OEM-part is found to be a bottle-neck in the
  calculations, some optimisation could be achieved by making a stand-alone
  function and e.g.\ use the conjugate gradient scheme during the iterations.
  Or maybe this option will be added to \texttt{oem.m}. However, the first main
  concern regarding calculation time will be to find optimised settings for
  \ARTS.}


\section{Software}
\label{sec:oem.m}
%
The \smr\ retrievals are performed by a Matlab implementation of \OEM,, that is
part of the Atmlab package. This package is available through the \ARTS\ site,
at \url{www.radiativetransfer.org/tools}. The implementation of \OEM\ in Atmlab,
\texttt{oem.m}, aims at being generic, that is, it shall be possible to couple
the function to different forward models. The Atmlab package contains the
required functions to interface \texttt{oem.m} with \ARTS. This functionality
has been used to create the second version of the Qpack
\citep{eriksson:qpack:05} inversion system. This version, Qpack2, is used by
several groups operating ground-based microwave radiometers
\citep[e.g.][]{acp-15-5099-2015}. The function \texttt{oem.m} was coupled to a lidar
forward model by \citet{sica2015retrieval}.

To be clear, earlier L2 processing made direct use of the Qpack system (version
1), but this is not longer the case, to allow a higher degree of optimisation
with respect to \smr. Instead, the core functionality, on which Qpack2 is
based, is used to set up a retrieval system targeting the needs raised by \smr.
The main components taken from Atmlab are \texttt{oem.m}, functions to
read/write \ARTS\ output/input files, functions to interpolate data of
climatology character (denoted as ``atmdata'' inside Atmlab) and functions to
create parametric variance-covariance matrices. All these functions are also
used by Qpack2 and, hence, are well tested.


\section{Iteration scheme}
\label{sec:ml}
%
The \smr\ processing offers a non-linear retrieval problem and an iterative
procedure is required to determine the solution (Eq.~\ref{eq:mincost}). A
number of iteration schemes has been suggested. The simplest version is
Gauss-Newton \citep[see][Sec.~5.3]{rodgers:00}. This iteration scheme assumes a
robust decrease of the cost function (\CstFnc, Eq.~\ref{eq:costfun}) during the
iterations, but this is seldom the case in practical retrievals and
Gauss-Newton can result in that \CstFnc\ instead is increasing while iterating.
The standard choice among iteration approaches including a manner to enforce a
decrease of the cost function at each step, is the Levenberg\,-\,Marquardt
method (LM), and it is also selected for these retrievals.

\LM\ operates with a parameter $\gamma$. With $\gamma=0$, \LM\ becomes equal to
Gauss-Newton iteration. For large values $\gamma$, \LM\ instead behaves as a
steepest descent method. The later results in a relatively small change in
\SttVct, and then also a slow convergence rate, but \SttVct\ is updated in a
direction that is close to optimal for ensuring a decrease of \CstFnc. See
\citet[][Sec.~5.7]{rodgers:00} for a more detailed presentation of \LM.
There exists different versions of \LM. The version applied here is
\citep[][Eq.~5.36]{rodgers:00}
\begin{equation}
  \label{eq:ml}
  \RtrVct_c = \RtrVct_i + 
  \left[ (1+\gamma)\aCvrMtr{a}^{-1} + 
          \aWfnMtrTrp{i}\aCvrMtr{o}^{-1}\aWfnMtr{i} \right]^{-1}
  \left[ \aWfnMtrTrp{i}\aCvrMtr{o}^{-1}(\MsrVct-\FrwMdl(\RtrVct_i,\FrwMdlVctHat)) -
         \aCvrMtr{a}^{-1}(\RtrVct_i-\aSttVct{a})\right]
\end{equation}
where $\RtrVct_c$ is the candidate solution for iteration $i+1$, $\RtrVct_i$ is
the (final) solution after iteration $i$, and \aWfnMtr{i}\ is the Jacobian
(\aWfnMtr{\SttVct}) using $(\RtrVct_i,\FrwMdlVctHat)$ as linearisation point.
Other variables are introduced in Sec.~\ref{sec:formalism}. The iteration is
started by setting $\RtrVct_0 = \aSttVct{a}$, and $\RtrVct_c$ becomes
$\RtrVct_{i+1}$ if $\CstFnc(\RtrVct_c) < \CstFnc(\RtrVct_i)$.

The scheme for updating $\gamma$, that must consider both successful
$(\CstFnc(\RtrVct_c) < \CstFnc(\RtrVct_i))$ and
unsuccessful $(\CstFnc(\RtrVct_c) \geq \CstFnc(\RtrVct_i))$ iterations. It also
needed to handle problematic cases by e.g.\ setting a limit for the number of
iterations to perform. The overall iteration scheme applied is:
\begin{itemize}
\item[0] Set start values: $\RtrVct_0 = \aSttVct{a}$, $i=0$ and $\gamma=\gamma_0$.
\item[1] If max iterations reached, $i=i_\mathrm{max}$, jump to 5. 
\item[2] Calculate $\RtrVct_c$ by Eq.~\ref{eq:ml}.
\item[3] If $\CstFnc(\RtrVct_c) < \CstFnc(\RtrVct_i)$:
  \begin{itemize}
  \item[3a] Set $\RtrVct_{i+1}=\RtrVct_c$.
  \item[3b] If convergence has reached (see Sec.~\ref{sec:conv}), set
    $\RtrVct=\RtrVct_{i+1}$ and jump to 5.
  \item[3c] Decrease $\gamma$ with a factor $f_s$. That is, $\gamma$ is updated
    as $\gamma\leftarrow\gamma/f_s$. 
  \item[3d] If $\gamma$ becomes smaller than a 
    threshold value, $\gamma<\gamma_\mathrm{min}$, set $\gamma=0$.
  \item[3e] Continue iterations by jumping to 1.
  \end{itemize}
\item[4] If $\CstFnc(\RtrVct_c) \geq \CstFnc(\RtrVct_i)$:
  \begin{itemize}
  \item[4a] Increase $\gamma$. If $\gamma=0$, set $\gamma=\gamma_\mathrm{min}$.
    Otherwise, updated as $\gamma\leftarrow f_u\cdot\gamma$.
  \item[4b] If $\gamma$ becomes larger than a threshold value,
    $\gamma>\gamma_\mathrm{max}$, set $\RtrVct=\RtrVct_i$ and jump to 5.
  \item[4c] Re-do the iteration with new $\gamma$ by moving to 1.
  \end{itemize}
\item[5] Perform retrieval characterisation
  (Chapter~\ref{chapter:characterisation}), using $\aWfnMtr{\SttVct}=\aWfnMtr{i}$.
\end{itemize}
The values of $\gamma_0$, $f_s$, $f_u$, $\gamma_\mathrm{min}$,
$\gamma_\mathrm{max}$ and $i_\mathrm{max}$ vary between the retrieval of the
different frequency bands. Actual values are listed in ?\todo{Where?}.
\\
\lcomment{PE}{Shall optical thickness be used to stabilise convergence? That
  is, only channels with low optical thickness are considered in the first
  iteration, and gradually higher optical thicknesses are introduced during the
  iterations.}


\section{Convergence tests and values}
\label{sec:conv}
%
\subsection{Stop criterion}
There are different aspects related to convergence. A first question is the
criterion for halting the iterations. This issue is discussed carefully in
Sec.~5.6.3 of \citet{rodgers:00} and the details are not repeated here. The
stop criterion applied is obtained by combining Eqs.~5.29 and 5.30 in the
book section cited:
\begin{equation}
  \label{eq:stopcrit}
  \left( \RtrVct_{i+1} - \RtrVct_i \right)^T
  \left( \aCvrMtr{a}^{-1} + 
         \aWfnMtrTrp{i}\aCvrMtr{o}^{-1}\aWfnMtr{i} \right)
  \left( \RtrVct_{i+1} - \RtrVct_i \right) < \Delta\SttVct_\mathrm{stop} \cdot n
\end{equation}
where $n$ is the length of the vector \SttVct\ and
$\Delta\SttVct_\mathrm{stop}$ is the variable controlling the wanted
strictness. The value $\Delta\SttVct_\mathrm{stop}=?$\todo{Select value}\ is
applied for all frequency bands. 

If Eq.~\ref{eq:stopcrit} becomes fulfilled, it is considered that basic
convergence has been achieved. There are two exceptions, resulting in
non-convergence status, that either the iteration number limit
$(i_\mathrm{max})$ has been reached or that $\gamma$ has reached its upper
threshold $(\gamma_\mathrm{max})$.
\\
\lcomment{PE}{Describe here how the overall convergence flag is set. We should
  also consider the history of $\gamma$, but how? }


\subsection{Normalised cost values}
%
For these \smr\ retrievals, the length of \MsrVct, $m$, is always much higher
than $n$ (length of \SttVct). In this situation, the final cost value, defined
according to Eq.~\ref{eq:costfun}, should be approximately $m$ at correct
convergence. As $m$ can vary from retrieval to retrieval, even for a single
frequency band, a normalised cost, $\CstFnc'$, is defined to make it simpler to
compare cost value from different retrievals. The normalised cost is
defined as:
\begin{eqnarray}
  \label{eq:scaledc}
  \CstFnc' &=& \CstFnc_y' + \CstFnc_x', \\
  \CstFnc_y' &=& \frac{(\MsrVct-\FrwMdl(\SttVct,\FrwMdlVctHat))^T\aCvrMtr{o}^{-1}
  (\MsrVct-\FrwMdl(\SttVct,\FrwMdlVctHat))}{m},\\
  \CstFnc_x' &=& \frac{(\SttVct-\aSttVct{a})^T\aCvrMtr{a}^{-1}(\SttVct-\aSttVct{a})}{m}.\\
\end{eqnarray}
The L2 data contain $\CstFnc'$, as well as $\CstFnc_y'$ to provide information
on how the final cost is distributed between the two penalty terms.


\subsection{Correct convergence}
%
Eq.~\ref{eq:stopcrit} evaluates only if the iterations has reached a stable end
point. If the found solution actually corresponds to a global minimum of the
cost function is another question, and a problematic one. To ensure a globally
optimal solution a more advanced approach than \LM\ is required, but all
such methods are computationally very demanding. A simplistic approach is to
restart the iteration with other start conditions, where $\RtrVct_0$ and/or
$\gamma=\gamma_0$ are/is given other initial value. Such tests have been
performed on a number of cases for each frequency bands, to validate the
iteration settings, but is not performed on a regular basis.\todo{Do this!}

The final cost value can be used for rough testing of incorrect convergence,
but it is hard to give a general critical value. Theoretically, it is expected
that $\CstFnc'\approx1$, but this is only true on the condition that
\aCvrMtr{o}\ gives a correct estimate of the observation uncertainties. For
simplicity, we only consider thermal noise, that is
$\aCvrMtr{0}=\aCvrMtr{\MsrErrVct_n}$. If then the magnitude of thermal noise is
over-estimated in \aCvrMtr{0}, on purpose or by mistake, obtained values of
$\CstFnc'$  on average be below 1 (assuming good convergence). However, for
practical retrievals, $\CstFnc'$ tend instead to be $>1$ due to artefacts in
the spectra, simplifications in the retrieval set-up and other reasons
resulting in that the residual:
\begin{equation}
  \label{eq:dy}
  \delta\MsrVct = \MsrVct-\FrwMdl(\RtrVct,\FrwMdlVctHat),
\end{equation}
contains features besides thermal noise. So called baseline ripple is a common
reason to an excess size of the residual.

For these reasons, a suitable upper limit on $\CstFnc'$ varies between the
frequency bands. Suggested limits for quality filtering for each band are found
?\todo{Where?}. 

The discussion above hints that the value of $\CstFnc'$ depends on the data
quality, and this aspect should not be forgotten. A high final
cost can be caused by both bad input data and that the iteration procedure
ended up at a local minimum. Without a detailed analysis it is impossible to say
which of the two potential problems that is the cause to suspiciously high end
value of $\CstFnc'$.  
\\
\lcomment{PE}{Is there anything smart to say about $\CstFnc_y'$? Can this value
  help to disentangle things?}


\section{Some notes on the matrix operations}
\label{sec:matrixops}
%
As mentioned above, $m>n$ for these \smr\ retrievals. In the terminology of 
\citet{rodgers:00}, the $n$-form of equations should then be used for lowest
calculation burden, and this guideline is strictly followed.

The matrix algebra is throughout performed by Matlab's internal functions. To
explicitly calculate matrix inverses is avoided as far as possible. For
example, Eq.~\ref{eq:ml} is evaluated, schematically, as
\begin{displaymath}
  \RtrVct_c = \RtrVct_i + 
  \left[ (1+\gamma)\aCvrMtr{a}^{-1} + 
          \aWfnMtrTrp{i}\aCvrMtr{o}^{-1}\aWfnMtr{i} \right] \backslash
  \left[ \aWfnMtrTrp{i}\aCvrMtr{o}^{-1}(\MsrVct-\FrwMdl(\RtrVct_i,\FrwMdlVctHat)) -
         \aCvrMtr{a}^{-1}(\RtrVct_i-\aSttVct{a})\right]  
\end{displaymath}
where $\backslash$ represents Matlab's left matrix divide operator (for
details, see Matlab documentation). By using this operator, the inverse
of $\left[ (1+\gamma)\aCvrMtr{a}^{-1} +
  \aWfnMtrTrp{i}\aCvrMtr{o}^{-1}\aWfnMtr{i} \right]$ is not calculated, instead
$\RtrVct_c - \RtrVct_i$ is found by solving an equation system (by QR
decomposition). This option is more numerically efficient, than calculating the
actual inverse.

However, a direct calculation of the inverse of \aCvrMtr{a} and \aCvrMtr{o} can
not be avoided. These calculations take advantage of the block-diagonal
structure (see Sec.~?), where each sub-matrix is inverted separately. Further,
in some situations the matrix invers can be obtained in an analytic manner.
This is possible for purely diagonal matrices, and for the combination of
constant variance and exponentially declining correlation
\citep[][Sec.~10.3.2.2]{rodgers:00}.

The size of the elements in \SttVct\ can differ hugely. For example, VMR values
go down to at least ppb level ($10^{-9}$), while frequency off-sets can reach
1\,MHz. Variances follow the magnitude of elements squared and the span of
values in \aCvrMtr{a}\ is accordingly even higher. Without any action, the
matrix algebra will be pushed towards, or beyond, the numerical accuracy. To
avoid numerical problems (at least due to the reason considered), an internal
re-scaling of \SttVct\ is applied. The re-scaling is done in such way that the
variances of \aCvrMtr{a}, matching the transformed \SttVct, all are unity. That
is, the elements in \SttVct\ are scaled with the square root of the
corresponding variance in \aCvrMtr{a}.


%%% Local Variables: 
%%% mode: latex
%%% TeX-master: "L2_ATBD"
%%% End: 
