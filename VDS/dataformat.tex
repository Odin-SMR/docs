\chapter{Data formats for the instruments}
\label{sec:dataformats}
\section{Aura/MLS}
The MLS data collocated with Odin/SMR is accessible through the Odin REST
API, see section~\ref{sec:api}. The data is returned as a JSON object with the
following attributes:
\begin{itemize}
    \item data\_fields \emph{(Object)}: Object containing the data under the
        following keys:
        \begin{itemize}
            \item AscDescMode \emph{(Int)}: 0 for ascending, 1 for descending
                measurement mode
            \item <Species> \emph{(Array of doubles)}: Concentration profile
                for <Species>
            \item <Species>Precision \emph{(Array of doubles)}: Precision of
                the concentration profile for <Species>
            \item L2gpValue \emph{(Array of doubles)}: Concentration profile
                for <Species>
            \item L2gpPrecision \emph{(Array of doubles)}: Precision of the
                concentration profile fore <Species>
            \item Quality \emph{(Double)}: Quality; larger is generally better
            \item Status \emph{(Int)}: Status code; use with caution if
                non-zero, don't use of odd
            \item Convergence \emph{(Double)}: Convergence of retrieval
                algorithms; values near unity indicate good convergence
        \end{itemize}
    \item geolocation\_fields \emph{(Object)}: Object containing the
        geolocation of the data under the following keys:
        \begin{itemize}
            \item ChunkNumber \emph{(Int)}: Number of chunks used in retrieval
            \item Latitude \emph{(Double)}: Latitude for the observation
            \item Longitude \emph{(Double)}: Longitude for the observation
            \item LineOfSightAngle \emph{(Double)}: Line of sight angle for the
                observation
            \item LocalSolarTime \emph{(Double)}: Local solar time for the
                observation
            \item MJD \emph{(Double)}: Time for the observation in MJD
            \item Time \emph{(Double)}: Time for the observation in TAI units
            \item OrbitGeodeticAngle \emph{(Double)}: The geodetic angle of the
                orbit at the time observation
            \item SolarZenithAngle \emph{(Double)}: The solar zenith angle for
                the observation
            \item Pressure \emph({Array of doubles}): Pressure profile for the
                observation
        \end{itemize}
\end{itemize}

\section{ENVISAT/MIPAS}
The MIPAS data collocated with Odin/SMR is accessible through the Odin REST
API, see section~\ref{sec:api}. The data is returned as a JSON object with the
following attributes:
\begin{itemize}
    \item MJD \emph{(Double)}: Time in MJD for the observation
    \item time \emph{(Double)}: Time of the observation in days since
        1970-01-01T00:00:00
    \item altitude \emph{(Array of doubles)}: Altitudes for the observation
    \item latitude \emph{(Double)}: Latitude for the observation
    \item longitude \emph{(Double)}: Longitude for the observation
    \item los \emph{(Array of doubles)}: Line of sight angles for the
        observation
    \item sza \emph{(Double)}: Sun zenith angle for the observation
    \item geo\_id \emph{(Array of strings)}: Geolocation identifier
    \item sub\_id \emph{(Array of strings)}: Sub-project identifier
    \item pressure \emph{(Array of doubles)}: Pressure profile for the
        observation
    \item temperature \emph{(Array of doubles)}: Temperature profile for the
        observation
    \item chi2 \emph{(Double)}: $\chi^2$ of retrieval
    \item dof \emph{(Double)}: Degrees of freedom of target retrieval
    \item eta \emph{(Array of doubles)}: Engineering tangent altitude
    \item eta\_indices \emph{(Array of ints)}: Indices of used engineering
        tangent altitude
    \item rms \emph{(Double)}: Root mean square of residual
    \item target \emph{(Array of doubles)}: Target profiles
    \item target\_noise\_error \emph{(Array of doubles)}: Noise error of target
        profiles
    \item visibility \emph{(Array of ints)}: Visibility of altitude;
        0 indicates obusfacted, 1 indicates visible
    \item akm\_diagonal \emph{(Array of doubles)}: Diagonal elements of
        averaging kernel
    \item vr\_akdiag \emph{(Array of doubles)}: Vertical resolution (altitude
        grid spacing divided by averaging kernel diagonal)
    \item vr\_col \emph{(Array of doubles)}: Vertical resolution (FWHM of
        averaging kernel columns)
    \item vr\_row \emph{(Array of doubles)}: Vertical resolution (FWHM of
        averaging kernel rows)
\end{itemize}

\section{ISS/JEM/SMILES}
The SMILES data collocated with Odin/SMR is accessible through the Odin REST
API, see section~\ref{sec:api}. The data is returned as a JSON object with the
following attributes:
\begin{itemize}
    \item data\_fields \emph{(Object)}: Object containing the data under the
        following keys:
        \begin{itemize}
            \item Apriori \emph{(Array of doubles)}: A priori profile
            \item AprioriError \emph{(Array of doubles)}: A priori error
            \item L2Value \emph{(Array of doubles)}: Retrieved profile
            \item L2Precision \emph{(Array of doubles)}: Calculation error;
                negative values indicate unuseful data
            \item AveragingKernel \emph{(Two-dimensional array of doubles)}:
                Averaging kernel
            \item Baseline<N> \emph{(Array of doubles)}: Coefficients of
                continuum
            \item Baseline<n>Precision \emph{(Array of doubles)}: Baseline
                errors of coefficients of continuum
            \item Convergence \emph{(Double)}: Convergence status
            \item CorrLength \emph{(Double)}: Correlative length of a priori
            \item CostFunctionY \emph{(Array of doubles)}: Cost function of
                spectra for each altitude
            \item CostFunctionYAll \emph{(Double)}: Cost function of spectra
            \item DifferenceY \emph{(Array of doubles)}: Normalised \chem{HCl}
                difference between scan and zonal mean profile
            \item DifferenceYAll \emph{(Double)}: Maximum normalised \chem{HCl}
                difference between scan and zonal mean profile
            \item FOVInterference \emph{(Int)}: Field-of-view interference
                flag; -1 indicates no information, 0 indicates no interference,
                larger than 0 indicates source of interferance
            \item InformationValue \emph{(Array of doubles)}: Information value
            \item MaxNumIteration \emph{(Int)}: Maximum convergence number
            \item NumIterPerform \emph{(Int)}: Convergence loop number and
                result
            \item MeasurementError \emph{(Array of doubles)}: Measurement
                error
            \item PrecisionWOSignal \emph{(Array of doubles)}: Calculation
                error without signal information
            \item Pressure \emph({Array of doubles}): Pressure profile for the
                observation
            \item Temperature \emph({Array of doubles}): Temperature profile
                for the observation
            \item RadianceResidualMax \emph{(Double)}: Maximum radience
                residual
            \item RadianceResidualMean \emph{(Double)}: Mean radience residual
            \item RadianceResidualRMS \emph{(Double)}: Root mean square
                radiance residual
            \item SeqCount \emph{(Int)}: Sequence counter
            \item AOSUnitNum \emph{(Double)}: Number of observed AOS units
            \item SmoothingError \emph({Array of doubles}): Smoothing error
            \item VerticalResolution \emph({Array of doubles}): Vertical
                resolution of the profiles
            \item WaterVapor \emph({Array of doubles}): Using water vapour of
                retrieval
            \item Status \emph{(Int)}: Status information as a bit mask error
                code; 0 indicates useful information
        \end{itemize}
    \item geolocation\_fields \emph{(Object)}: Object containing the
        geolocation of the data under the following keys:
        \begin{itemize}
            \item AscendingDescending \emph{(Int)}: 0 for ascending, 1 for
                descending measurement mode
            \item Latitude \emph{(Double)}: Latitude for the observation
            \item Longitude \emph{(Double)}: Longitude for the observation
            \item LineOfSightAngle \emph{(Double)}: Azimuth view
            \item MJD \emph{(Double)}: Time for the observation in MJD
            \item Time \emph{(Double)}: Time for the observation in seconds
                from 1958-01-01
            \item LocalTime \emph{(Double)}: Local time (decimal hour of day)
                for the observation
            \item TimeUTC \emph{(String)}: Date and time for the observation in
                UTC as a string in ISO format
            \item SolarZenithAngle \emph{(Double)}: The solar zenith angle for
                the observation
            \item Altitude \emph({Array of doubles}): Represenetative
                altitudes~(km) for the measured profiles
            \item Reserved \emph{(Int)}: Reserved field
        \end{itemize}
\end{itemize}


%\section{Odin/OSIRIS}
%...


\section{Meteor3M/SAGE III}
The SAGE III data collocated with Odin/SMR is accessible through the Odin REST
API, see section~\ref{sec:api}. The data is returned as a JSON object with the
following attributes:
\begin{itemize}
    \item FileName    \emph{(String)}: Filename is the same as in the original
        data set
    \item Instrument  \emph{(String)}: Name of the instrument
    \item EventType   \emph{(String)}: Solar or Lunar (only Solar included in
        VDS)
    \item MJDStart    \emph{(Double)}: Start time in MJD for the observation
    \item MJDEnd      \emph{(Double)}: End time in MJD for the observation
    \item LatStart    \emph{(Double)}: Start latitude for the observation
    \item LatEnd      \emph{(Double)}: End latitude for the observation
    \item LongStart   \emph{(Double)}: Start longitude for the observation
    \item LongEnd     \emph{(Double)}: End longitude for the observation
    \item Pressure    \emph{(Array of doubles)}: Pressure profile for the
        observation
    \item Temperature \emph{(Array of doubles)}: Temperature profile for the
        observation
    \item <Species>   \emph{(Array of triplets of doubles)}: Profile for
        <Species> for the observation; each row is a triplet containing
        concentration, uncertainty, and a quality bit
\end{itemize}


\section{Odin/OSIRIS}
The OSIRIS data collocated with Odin/SMR is accessible through the Odin REST
API, see section~\ref{sec:api}. The data is returned as a JSON object with the
following attributes (see \url{http://odin-osiris.usask.ca/sites/default/files/media/pdf/l2dataformat.pdf}
for more details):
\begin{itemize}
    \item data\_fields \emph{(Object)}: Object containing the data under the
        following keys:
        \begin{itemize}
            \item O3                     \emph{(Array of doubles)}: Measured ozone profiles 
                expressed as volume mixing ratio
            \item O3NumberDensity        \emph{(Array of doubles)}: The measured ozone profiles 
                expressed as number density
            \item O3Precision            \emph{(Array of doubles)}: The error in the ozone volume 
                mixing ratio
            \item RTModel\_AirDensity     \emph{(Array of doubles)}: The atmospheric air density 
                profile used in the radiative transfer model
            \item RTModel\_Albedo         \emph{(Double)}: The ground albedo value used in the 
                radiative transfer code
            \item RTModel O3Density      \emph{(Array of doubles)}: The atmospheric air density 
                profile used in the radiative transfer model
            \item RTModel\_O3InitialGuess \emph{(Array of doubles)}: The initial guess of \chem{O_3} 
                used in the MART retrieval, expressed as a number density
            \item RTModel\_Temperature    \emph{(Array of doubles)}: The atmospheric temperature 
                profile used in the radiative transfer model in the inversion algorithms
        \end{itemize}
    \item geolocation\_fields \emph{(Object)}: Object containing the
        geolocation of the data under the following keys:
        \begin{itemize}
            \item Altitude              \emph{(Array of double)}: The geometric altitude of the 
                primary swath data products expressed in km. The altitudes are specified with 
                respect to the IAU 1976 reference geoid
            \item Latitude              \emph{(Double)}: The nominal latitude of the scan expressed 
                in degrees. This corresponds to the latitude of the tangent point at time Time.
            \item Longitude             \emph{(Double)}: The nominal longitude of the scan expressed 
                in degrees. This corresponds to the latitude of the tangent point at time Time
            \item LocalSolarTime        \emph{(Double)}: The local solar time at the nominal tangent 
                point expressed in hours
            \item RTModel\_Altitude      \emph{(Array of double)}: The geometric altitude of the model 
                inputs expressed in km. The altitudes are with respect to the IAU 1976 reference geoid 
            \item ScanEndLatitude       \emph{(Double)}: The latitude of the tangent point in degrees 
                at the end of the scan
            \item ScanEndLongitude      \emph{(Double)}: The longitude of the tangent point at the end 
                of the scan
            \item ScanEndTime           \emph{(Double)}: The end time of the scan expressed as TAI93 
                assuming no offset between UTC and TAI93
            \item ScanNo                \emph{(Int)}: The unique identification number of this scan
            \item ScanStartLatitude     \emph{(Double)}: The latitude of the tangent point in degrees 
                at the start of the scan
            \item ScanStartLongitude    \emph{(Double)}: The longitude of the tangent point in degrees 
                at the start of the scan
            \item ScanStartTime         \emph{(Double)}: The start time of the scan expressed as TAI93 
                assuming no offset between UTC and TAI93
            \item ScanUpFlag            \emph{(Int)}:Indicates if the scan was going up (1) or going down (0)
            \item SolarAzimuthAngle     \emph{(Double)}:  The solar azimuth angle expressed in degrees at 
                the nominal tangent point of the scan
            \item SolarScatteringAngle  \emph{(Double)}: The solar scattering angle expressed in degrees 
                at the nominal tangent point of the scan 
            \item SolarZenithAngle      \emph{(Double)}: The solar zenith angle expressed in degrees at 
                the nominal tangent point of the scan
            \item Time                  \emph{(Double)}: The nominal time of the scan in UTC expressed as 
                TAI93 assuming no offset between UTC and TAI93. This corresponds to the instant when the 
                tangent point of the OSIRIS look vector was at 25 km
            \item MJD \emph{(Double)}: Time for the observation in MJD
        \end{itemize}
\end{itemize}

