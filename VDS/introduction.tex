\chapter{Introduction}
\label{chapter:introduction}


% The page numbering must be reset here inside the file
\pagenumbering{arabic}
\setcounter{page}{1}


\section{Aim and scope of this document}
\label{sec:aim}
\smr\ performs passive limb measurements of the atmosphere,
mainly at wavelengths and frequencies around 0.6\,mm and 500\,GHz,
respectively.
From these measurements, profiles of 
\chem{O_3}, \chem{ClO}, \chem{N_{2}O}, \chem{HNO_{3}}, 
\chem{H_{2}O}, \chem{CO}, and isotopologues of \chem{H_{2}O}, and \chem{O_{3}},
that are species that are of interest for studying stratospheric and 
mesospheric chemistry and dynamics, can be derived. 
\smr\ has been in operation for approximately 14 years, and thus, the Level2
dataset can potentially be applied for scientifically interesting trend analysis.

A new \smr\ Level2 product dataset will be generated, and this dataset will be based
on updated/revised processing algorithms and input data.
A verification dataset (VDS) will be used as a tool to verify the new 
processing system/Level2 products.

The aim of this document is to describe this VDS, and the API used for
accessing the data.
This VDS is a representative subset of the \smr\ Level1B dataset
and collocated correlative measurements from similar
instruments, i.e. Level2 data from Odin/OSIRIS, Aura/MLS, ENVISAT/MIPAS,
ISS/JEM/SMILES, and Meteor3M/SAGEIII.   
Ground based measurements, i.e. ozone sonde and lidar observations,
from a number of different stations, will also be used
to verify the new \smr\ Level2 products.  
For convenience, the Odin/SMR Level2 data product produced with the older
2.0/2.1 versions of the processing chain is also accessible for the data in the
VDS through the same API.

\section{Document structure}

This document is organized as follows:
Chapter 2 describes the \smr\ Level2 data products.
Chapter 3 describes the verification/correlative datasets
included in the VDS, and how the VDS was constructed.
Chapter 4 describes an API to the VDS.

