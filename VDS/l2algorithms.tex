\chapter{\smr\ Level2 algorithms and input data}
\label{level2-algorithms}
\section{Overview}


\begin{figure}[t]
\centering
\includegraphics[width=14cm]{level2-overview.png}
\caption{Schematic of the Level2 processing and input data.}
\label{fig:level2}
\end{figure}

The basic target of the Level2 algorithms is to
convert observed and calibrated Level1B spectra
to atmospheric species profiles.
The retrieval problem is in general non-unique 
and non-linear for \smr\,  
%The mapping between an observed set of spectra (a scan)
%and the underlying atmospheric species profile is not unique.
and the \smr\ level2 data products will be derived by an
\OEMlong\ (\OEM) implementation, as described
in the \smr\ \ATBD\ for level2 processing \citep{atbdl2}.
The \OEM\ implementation combines measurement information with 
Ancillary--Auxiliary data and applies a forward 
model (Figure~\ref{fig:level2}). High level requirements for each 
component is described in the following sections.

\section{Forward model requirements}

Forward model requirements are described in Sect. 2.2
and 2.4.2 in the \ATBD\ for level2 processing.
In short, these requirements are:

\begin{itemize}

\item limb sounding requires that refraction and the spherical shape
of the Earth is considered when determining the propagation path,
which excludes all plane-parallel models,

\item the forward model must be able to deliver Jacobians,

\item the forward model must be able to incorporate effects of
sensor characteristics (e.g. antenna angular response, sideband filtering,
frequency response of each spectrometer channel).

\end{itemize}

\ARTS, the \ARTSlong,
is a publicly available \FOSSlong\ project.
% that is both developed and  maintained as free and open source project.
\ARTS\ is in compliance with the requirements for \smr\ Level2
data processing. Thus, no further refinements of \ARTS\ are necessary to perform.

\subsection{Spectroscopic data}

Spectroscopic data can be seen as basic input to the forward model,  
and it is of high importance that this data is as correct as possible. 
The positions and strengths of molecular transitions
are known with a relatively high accuracy. In contrast, effects of pressure
broadening and non-resonant absorption are less well known.
Spectroscopic parameters can for instance be found in the 
High-resolution transmission molecular absorption (HITRAN) database.
However, a requirement is that a literature review is performed,
in order to identify the best possible spectroscopic data to apply. 


\section{OEM implementation requirements}

The basis of the required \OEM\ (\OEMlong) implementation for \smr\ 
is described in Chapter 4 of the \ATBD\ for level2 processing.
In short, the basic considerations/requirements of the OEM implementation are:
\begin{itemize}
\item the \OEM\ implementation must contain an interface to
      the applied forward model,
\item the retrieval problem is in general non-unique, hence
      a constraint to the solution must be applied,
\item measurement information must be combined with \textit{a priori}
      information, and the \OEM\ implementation must contain functions to 
      create parametric covariance matrices,
\item the retrieval problem is non-linear, hence an iterative
      procedure is required to find the solution,
\item the iteration scheme needs a convergence test and stop 
      criteria.
\end{itemize}

A matlab implementation, that is part of the Atmlab package, 
of \OEM\ will be applied. This package is available through the \ARTS\ site,
at \url{www.radiativetransfer.org/tools}. It provides an interface
to \ARTS\ and other necessary functionality, and has
been used by several groups for retrieval processing.
However, the exact setup of the retrieval processing
has to be seen specific for \smr\ and tests must be performed
to identify best setup. Level2 data validation 
is covered in Chapter~\ref{validation}.


\section{Level1B data requirements}

The accuracy of the Level1B dataset is important for the quality of the
Level2 dataset for obvious reasons. 
Following various validation exercises a number of 
issues of \smr\ Level1 data have been identified, 
including strange baselines, glitches and jumps in the power level.
A requirement is therefore that a review of the calibration 
scheme and the processing algorithms is performed,
and that any errors that are found are corrected if possible.
A further requirement is that Level1 data is quality classified, 
and that corrupt data is weeded out.
The accuracy of Level1 data should be verified by comparison
to simulated measurements (see Chapter~\ref{validation}).
In short,
%\smr\ has been in operation
%for \(\sim\)14 years, and thus, the Level2 dataset can be applied for
%scientifically interesting trend analysis. It is therefore of
%high importance that the Level1B dataset is accurate and stable.
the best available \smr\ Level1B data should be applied,
and the Level1B data must not contain any errors that are known.


\section{Ancillary--Auxiliary data requirements}
\subsection{Climatology \textit{a priori} data}
\textit{A priori} data, or a climatology database covering all species of interest
is required, as the \OEM\ implementation needs a starting estimate for each profile
to be retrieved. 
The climatology must also cover the relevant vertical range for each species, and
latitudinal and seasonal variations.
%Climatlogy data must have an accuracy better than ? \(\%\).
A requirement is that a literature review is performed,
in order to identify the best possible climatology data.


\subsection{ZPT data}
External ZPT data (altitude, pressure, temperature) is given as input
to the Level2 processing. Data from the selected source must be available
for the complete Odin mission. Furthermore, the data should be as accurate
as possible %(typically temperature errors less than 2\,K?), 
and contain no known artificial trends.

A commonly used data source for this type of application, is
European centre for medium-range weather forecasts (ECMWF).
In principle, there are two possible ECMWF products to choose
between, i.e. data from the operational model or from   
ERA-Interim. ERA-Interim is a global atmospheric reanalysis from 1979,
and is based on the same version of assimilation system,
whereas the operational product has changed assimilation
version during the Odin mission. 
 
Thus, a requirement is that a temperature trend analysis is performed
for candidate data sources, in order to verify that the data contains
no spurious trend, that can propagate to an artificial trend in
\smr\ Level2 data.
 

\subsection{Sensor characteristic data}  
The forward model simulation must take into account sensor characteristic data,
such as the antenna angular response, sideband filtering, and frequency response 
of each spectrometer channel. A requirement is that the best possible
data is applied. 



