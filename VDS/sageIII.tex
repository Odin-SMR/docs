\section{SAGE III on Meteor-3M}

\subsection*{Introduction\footnote{The text in this section has been
adapted from \cite{NASA_SAGEIII/M3M, SAGEIII_DPUG}.}}
SAGE III on Meteor-3M (SAGE III/M3M) was a third generation, satellite-borne
instrument and a crucial element in NASA’s Earth Observing System (EOS). The
instrument was launched on the Russian Meteor-3M spacecraft on 10 December
2001 into into a Sun-synchronous orbit at an altitude of 1020 km and with an
approximate 9:00 a.m. equatorial crossing time.  The instrument was active from
2002-02-27 to 2005-11-12.

\subsection*{Level 2 products}
The SAGE III instrument measures the attenuation of solar radiation resulting
from the scattering and absorption by atmospheric constituents in the Earth’s
atmosphere as the spacecraft observes a sunrise or sunset event.  Due to the
orbital parameters, solar occultation measurement opportunities are limited to
mostly high latitudes in the Northern Hemisphere (between 50\degree~and
80\degree~N) and mid-latitudes in the Southern Hemisphere (between
30\degree~and 50°~S).  Level~2 products from these measurements include
profiles of ozone~(\chem{O_3}), water vapour~(\chem{H_2O}) and nitrogen
dioxide~(\chem{NO_2}).  Of these \chem{O_3} and \chem{H_2O} have been included
in the VDS; see Table~\ref{tab:sage3products} for a summary.

Similar measurements where made during the lunar moonrise and moonset. Due to
poor data coverage, none of these products have been included in the VDS, but
are listed here for completeness.  These measurements were made only during the
second and third quarter phases of the Moon and when the atmosphere along the
line-of-sight (LOS) was not directly illuminated by the Sun.  Level~2 products
from these measurements include profiles of ozone~(\chem{O_3}), nitrogen
dioxide~(\chem{NO_2}), nitrogen trioxide~(\chem{NO_3}) and chlorine
dioxide~(\chem{OClO}).

\begin{table}
    \caption{Products from SAGE III included in the VDS}
    \label{tab:sage3products}
    \begin{tabular}[llp{1in}l]
\hline\hline
Species & Altitude range & Expected precision & Corresponding Odin/SMR frequency mode \\
\hline
Ozone (\chem{O_3}) & 6--85 km & 10\% & FM \\
Water vapour (\chem{H_2O}) & 0--50 km & 5--15\% (<5\% below 33~km) & FM \\
\hline\hline
    \end{tabular}
\end{table}

\subsection*{Data format}
The SAGE III data collocated with Odin/SMR is accessible through the Odin REST
API trhough URIs such as:

EXAMPLE

The data is returned as a JSON object with the attributes described in
Table~\ref{tab:sage3data}.

\begin{table}
    \caption{Description of attributes in SAGE III JSON object}
    \label{tab:sage3data}
    \begin{tabular}[lp{1in}p{3in}]
\hline\hline
Attribute & Type & Comment
\hline
FileName & String & Filename is the same as in the original data set\\
Instrument & String & Name of the instrument \\
EventType & String & Solar or Lunar (only Solar included in VDS) \\
MJD & List of doubles & Start and end time in MJD for the measurement \\
Latitudes & List of doubles & Start and end latitudes for the measurement \\
Longitudes & List of doubles & Start and end longitudes for the measurement \\
Pressure & List of doubles & Pressure profile for the measurement \\
Temperature & List of doubles & Temperature profile for the measurement \\
<Species> & List list of doubles & Profile for <species> the measurement; each
row contains concentration, uncertainty, and a quality bit
flag\footnote{see \ref{SAGEII_DPUG} for details!} \\
\hline\hline
    \end{tabular}
\end{table}

