\chapter{API decription}
\label{sec:api}
This section describes the API calls used to get data from from the VDS.  The
data is accessed through a hierarcical REST API where deeper URIs return more
specific data.  All call URIs have a common root \url{rest_api/<version>}, which
has been omitted below for clarity.  All GET calls return JSON objects unless
otherwise noted. Key/value pairs are listed as name of the key
along with the type of the corresponding value within parantheses, followed
by a brief description of the contents.  See the sections on the different
data sources for specifications on the structure of their respective JSON
objects.


\section{\url{<version>/vds}}
Method: \emph{GET}

Returns object with the following attributes:
\begin{itemize}
    \item VDS:

        A list of objects containing information about collocated scans,
        grouped by backend and frequency mode.
        Each object contains the following keys:

        \begin{itemize}
            \item Backend \emph{(String)}: The backend for the data
            \item FreqMode \emph{(Int)}: The frequency mode for the data
            \item URL-allscans \emph{(URI)}: A URI for getting a list of all
                scans from the VDS for the Backend/FreqMode pair
            \item URL-collocations \emph{(URI)}: A URI for getting a more
                specific list of the data available for the Backend/FreqMode
                pair
        \end{itemize}
\end{itemize}


\section{\url{vds/<backend>/<freqmode>}}
Method: \emph{GET}

Returns object with the following attributes:
\begin{itemize}
    \item VDS:

        A list of objects containing information about collocated scans for
        the chosen backend and frequency, grouped by instrument and species.
        Each object contains the following keys:

        \begin{itemize}
            \item Backend \emph{(String)}: The backend for the data
            \item FreqMode \emph{(Int)}: The frequency mode for the data
            \item Instrument \emph{(String)}: The name of the instrument with
                which the VDS data is collocated
            \item NumScan \emph{(Int)}: Number of collocated scans
            \item Species \emph{(String)}: The species the collocated data
                considers
            \item URL \emph{(URI)}: A URI for getting a more specific list of
                the data available for the Instrument/Species pair
        \end{itemize}
\end{itemize}


\section{\url{vds/<backend>/<freqmode>/allscans}}
Method: \emph{GET}

Returns object with the following attributes:
\begin{itemize}
    \item VDS:

        A list of objects containing detailed information about all the scans
        in the VDS for the chosen backend and frequency.
        Each object contains the following keys:

        \begin{itemize}
            \item Info \emph{(Object)}: Object containing information
                about the Odin/SMR scan, such as time and geolocation
            \item URLS \emph{(Object)}: Object containing URIs for getting the
                Odin/SMR spectra and apriori data for the specific scan, as
                well as the PTZ data
        \end{itemize}
\end{itemize}


\section{\url{vds/<backend>/<freqmode>/<species>/<instrument>}}
Method: \emph{GET}

Returns object with the following attributes:
\begin{itemize}
    \item VDS:

        A list of objects containing information about collocated scans for
        the chosen backend, frequency, species and instrument, grouped by date.
        Each object contains the following keys:

        \begin{itemize}
            \item Backend \emph{(String)}: The backend for the data
            \item Date \emph{(String)}: The date (in ISO format) on which the
                data was collected
            \item FreqMode \emph{(Int)}: The frequency mode for the data
            \item Instrument \emph{(String)}: The name of the instrument with
                which the VDS data is collocated
            \item NumScan \emph{(Int)}: Number of collocated scans
            \item Species \emph{(String)}: The species the collocated data
                considers
            \item URL \emph{(URI)}: A URI for getting a more specific list of
                the data available for the particular date
        \end{itemize}
\end{itemize}


\section{\url{vds/<backend>/<freqmode>/<species>/<instrument>/<date>}}
Method: \emph{GET}

Returns object with the following attributes:
\begin{itemize}
    \item VDS:

        A list of objects containing detailed information about the scans for
        the chosen backend, frequency, instrument, species and date.
        Each object contains the following keys:

        \begin{itemize}
            \item CollocationInfo \emph{(Object)}: Object containing
                information about the collocated measurement, such as time and
                geolocation, as well as Delta time and angular distance between
                the \smr\ scan and the collocated data from the selected
                instrument
            \item OdinInfo \emph{(Object)}: Object containing information
                about the \smr\ scan, such as time and geolocation
            \item URLS \emph{(Object)}: Object containing URIs for getting the
            Odin/SMR spectra and apriori data for the specific scan, as
            well as the PTZ data, and the collocated data from the selected
            instrument
        \end{itemize}
\end{itemize}


\section{\url{vds_external/<instrument>/<species>/<date>/<file>/<file_index>}}
Method: \emph{GET}

Returns object containing the data for the specified instrument, species and
date. See Sect.~\ref{sec:dataformats} for details on their
respective data structures.

\section{Example usage}
\label{sec:api_usage}
\begin{lstlisting}[language=Python, basicstyle=\footnotesize]
# Setup the name space:
import requests

FMs = [1, 2, 8, 13, 14, 17, 19, 21]
instruments = ['mls', 'mipas', 'sageIII', 'smiles']

# Make a request to the root URI of the VDS API:
r0 = requests.get("http://malachite.rss.chalmers.se/rest_api/v4/vds/")

# Single out the frequency mode of interest, in this case 2:
FM2 = [x for x in r0.json()['VDS'] if x['FreqMode'] == 2][0]

# Make a new request using the URI provided in the JSON object, and
# single out the species O3 and the instrument MLS:
r1 = requests.get(FM2["URL-collocation"])
O3MLS = [x for x in r1.json()['VDS'] if x['Species'] == 'O3' and
         x['Instrument'] == 'mls'][0]

# Repeat for the new object and a date of interest:
r2 = requests.get(O3MLS['URL'])
day = [x for x in r2.json()['VDS'] if x['Date'] == '2012-09-15'][0]

# To get the detailed information about the collocated scans on that
# day, we make one more request:
r3 = requests.get(O3MLS['URL'])
data = r3.json()['VDS']

# data contains URIs for getting all the colloacted Odin/SMR scans
# for 2012-09-15, as well as the data from MLS. As a final step,
# let's request the latter for the first of the collocated scans for
# our chosen frequency mode, species, instrument and day:
r4 = requests.get(data[0]['URLS']['URL-mls-O3'])
mlsData = r4.json()

# Now we have the data at hand, and can proceed with crunching it!
\end{lstlisting}
