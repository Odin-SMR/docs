\section{API decription}
\label{sec:api}
This section describes the API calls used to get data from from the VDS.  The
data is accessed through a hierarcical REST API where deeper URIs return more
specific data.  All call URIs have a common root \url{rest_api/<version>}, which
has been omitted below for clarity.  All GET calls return JSON objects unless
otherwise noted. See the sections on the different data sources for
specifications on the structure of their respective JSON objects.


\subsection{\url{<version>/vds}}
Method: \emph{GET}

Returns object with the following attributes:
\begin{itemize}
    \item VDS:

        A list of objects containing information about collocated scans,
        grouped by backend and frequency mode.
        Each object contains the following keys:

        \begin{itemize}
            \item Backend \emph{(String)}: The backend for the data
            \item FreqMode \emph{(Int)}: The frequency mode for the data
            \item URL-allscans \emph{(URI)}: A URI for getting a list of all
                scans from the VDS for the Backend/FreqMode pair
            \item URL-collocations \emph{(URI)}: A URI for getting a more
                specific list of the data available for the Backend/FreqMode
                pair
        \end{itemize}
\end{itemize}


\subsection{\url{vds/<backend>/<freqmode>}}
Method: \emph{GET}

Returns object with the following attributes:
\begin{itemize}
    \item VDS:

        A list of objects containing information about collocated scans for
        the chosen backend and frequency, grouped by instrument and species.
        Each object contains the following keys:

        \begin{itemize}
            \item Backend \emph{(String)}: The backend for the data
            \item FreqMode \emph{(Int)}: The frequency mode for the data
            \item Instrument \emph{(String)}: The name of the instrument with
                which the VDS data is collocated
            \item NumScan \emph{(Int)}: Number of collocated scans
            \item Species \emph{(String)}: The species the collocated data
                considers
            \item URL \emph{(URI)}: A URI for getting a more specific list of
                the data available for the Instrument/Species pair
        \end{itemize}
\end{itemize}


\subsection{\url{vds/<backend>/<freqmode>/allscans}}
Method: \emph{GET}

Returns object with the following attributes:
\begin{itemize}
    \item VDS:

        A list of objects containing detailed information about all the scans
        in the VDS for the chosen backend and frequency.
        Each object contains the following keys:

        \begin{itemize}
            \item Info \emph{(Object)}: Object containing information
                about the Odin/SMR scan, such as time and geolocation
            \item URLS \emph{(Object)}: Object containing URIs for getting the
                Odin/SMR spectra and apriori data for the specific scan, as
                well as the PTZ data
        \end{itemize}
\end{itemize}


\subsection{\url{vds/<backend>/<freqmode>/<species>/<instrument>}}
Method: \emph{GET}

Returns object with the following attributes:
\begin{itemize}
    \item VDS:

        A list of objects containing information about collocated scans for
        the chosen backend, frequency, species and instrument, grouped by date.
        Each object contains the following keys:

        \begin{itemize}
            \item Backend \emph{(String)}: The backend for the data
            \item Date \emph{(String)}: The date (in ISO format) on which the
                data was collected
            \item FreqMode \emph{(Int)}: The frequency mode for the data
            \item Instrument \emph{(String)}: The name of the instrument with
                which the VDS data is collocated
            \item NumScan \emph{(Int)}: Number of collocated scans
            \item Species \emph{(String)}: The species the collocated data
                considers
            \item URL \emph{(URI)}: A URI for getting a more specific list of
                the data available for the particular date
        \end{itemize}
\end{itemize}


\subsection{\url{vds/<backend>/<freqmode>/<species>/<instrument>/<date>}}
Method: \emph{GET}

Returns object with the following attributes:
\begin{itemize}
    \item VDS:

        A list of objects containing detailed information about the scans for
        the chosen backend, frequency, instrument, species and date.
        Each object contains the following keys:

        \begin{itemize}
            \item CollocationInfo \emph{(Object)}: Object containing
                information about the collocation, such as time and
                geolocation, as well as Delta time and angular distance between
                the Odin/SMR scan and the collocated data from the selected
                instrument
            \item OdinInfo \emph{(Object)}: Object containing information
                about the Odin/SMR scan, such as time and geolocation
            \item URLS \emph{(Object)}: Object containing URIs for getting the
            Odin/SMR spectra and apriori data for the specific scan, as
            well as the PTZ data, and the collocated data from the selected
            instrument
        \end{itemize}
\end{itemize}


\subsection{\url{vds_external/<instrument>/<species>/<date>/<file>/<file_index>}}
Method: \emph{GET}

Returns object containing the data for the specified instrument, species and
date. See the sections describing the various instruments for details on their
respective data structures.

\subsection{Typical usage}
\label{sec:api_usage}

\TODO{Add typical usage}
