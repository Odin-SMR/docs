\chapter{\smr\ Level2 data products}

\section{The Odin mission}
\label{sec:odin}
%

The Odin satellite was launched on the 20th of February 2001, into a sun-synchronous
18:00 hour ascending node orbit, carrying two co-aligned limb sounding
instruments: \OSIRIS\ (\OSIRISlong) and
\SMR\ (\SMRlong) \citep{murtagh:anove:02}. 
Originally, Odin was used for both
atmospheric and astronomical observations, but since 2007 only its aeronomy
mission is active. Odin is a Swedish-led project, in cooperation with Canada,
France and Finland. Both of Odin's instruments are still functional, and the
present operation of the satellite is partly performed as an \ESA\ third party
mission.

\section{The SMR instrument}

The \smr\ package is highly flexible \citep{frisk:theod:03}. 
In short, the four main receiver chains
can be tuned to cover frequencies in the ranges 486--504\,GHz and
541--581\,GHz, but the maximum total instantaneous bandwidth is only
1.6\,GHz. This bandwidth is determined by the two auto-correlation
spectrometers (ACs) used for atmospheric observations. The two ACs can be
coupled to any of the four front-ends, but only two or three front-ends are
used simultaneously. The ACs cover 400 or 800\,MHz per front-end, depending on
configuration. In the configuration applied for atmospheric sounding, the
channels of the ACs have a spacing of 1\,MHz, while the frequency resolution is
only 2\,MHz.
To cover all molecular transitions
of interest a number of ``observation modes'' have been defined. Each
observation mode makes use of two or three frequency bands. Single sideband
operation is obtained by tunable Martin--Pupplet interferometers. The nominal
sideband suppression is better than 19\,dB across the image band.

\smr\ also has a receiver chain around the 118\,GHz oxygen transition that was
heavily used during Odin's astronomy mission. For the atmospheric mission, this
front-end was planned to be used for retrieving temperature profiles, but a
technical problem (drifting LO frequency) and the fact that the analysis
requires treatment of Zeeman splitting have given these data low priority. 

The main reflector of \smr\ has a diameter of 1.1\,m, giving a
vertical resolution at the tangent point of about 2\,km. The vertical scanning
of the two instruments' line-of-sight is achieved by a rotation of the satellite
platform, with a rate matching a vertical speed of the tangent altitude of
750\,m/s. Measurements are performed during both upward and downward scanning.
The lower end of the scan is typically at about 7\,km, the upper end varies
between 70 and 110\,km, depending on observation mode. In correspondence,
the horizontal sampling ranges from 1 scan per 600 km to 1 scan per 1000 km.
Measurements are in general performed along the orbit plane, providing a
latitude coverage between 82.5$^{\circ}$S and 82.5$^{\circ}$N. Since the end of
2004 Odin is also pointing off-track during certain periods, e.g.\ during the
austral summer season, allowing the latitudinal coverage to be extended towards
the poles. 

\begin{table}
\caption{ \smr\ Level2 data (version 3.0.0) products by frequency mode.}
\label{table:level2def}
\scalebox{0.965}{%
\begin{tabular}{|l|l|}
  \hline
  \textbf{Frequency mode}  & \textbf{Product} \\
  \hline
  \hline
  01                       & "ClO / 501 GHz / 20 to 55 km" \\
                           & "N2O / 502 GHz / 15 to 50 km" \\
                           & "O3 / 501 GHz / 20 to 50 km" \\

  \hline
  \hline
  02                       & "H2O / 545 GHz / 15 to 30 km" \\
                           & "HNO3 / 545 GHz / 20 to 50 km" \\
                           & "O3 / 545 GHz / 20 to 85 km" \\
                           & "O3-668 / 545 GHz / 25 to 45 km" \\
                           & "Temperature / 545 GHz / 15 to 65 km" \\
  \hline
  \hline
  08                       & "H2O / 488 GHz / 20 to 70 km" \\
                           & "H2O-181 / 488 GHz / 20 to 60 km" \\
                           & "O3 / 488 GHz / 20 to 60 km" \\
  \hline
  \hline
  13                       & "H2O - 557 GHz - 45 to 100 km" \\
                           & "O3 - 557 GHz - 45 to 90 km" \\
                           & "Temperature - 557 (Fmode 13) - 45 to 90 km" \\
  \hline
  \hline
  19                       & "H2O - 557 GHz - 45 to 100 km" \\
                           & "O3 - 557 GHz - 45 to 90 km" \\
                           & "Temperature - 557 (Fmode 19) - 45 to 90 km" \\
  \hline
  \hline
  21                       & "NO - 551 GHz - 45 to 115 km" \\
                           & "O3 - 551 GHz - 45 to 90 km" \\
                           & "Temperature - 551 GHz - 45 to 65 km" \\
  \hline
  14, 22, 24               & "CO - 576 GHz" \\
  \hline
\end{tabular}}
\end{table}



\section{\smr\ Level2 data products}

\smr\ data are categorized into main and science Level2 products.
The main products are retrieved from the so called ``stratospheric'' observation mode
of \smr, and this modes cover approximately 50\,\(\%\) of the \smr\ observation time.
In this mode spectra in frequency bands around 501 and 544\,GHz are collected.
The science data products are derived from less frequently applied observation modes
(typically applied a few days per month).
Ozone, \chem{ClO}, \chem{N_{2}O}, and \chem{HNO_{3}} profiles
are the main \smr\ Level2 products.
\chem{ClO} and \chem{N_{2}O} profiles are retrieved from spectra covering
transitions around 501\,GHz, and \chem{HNO_{3}} from spectra around 544\,GHz.
Ozone can be retrieved from both the 501 and the 544\,GHz band.
Profiles of \chem{H_{2}O}, \chem{CO}, \chem{NO} and
isotopologues of \chem{H_{2}O}, and \chem{O_{3}}
are considered as science data products for \smr.
Observations covering the science data products are performed on a less 
frequent basis than the main data products. 
Table~\ref{table:level2def} describes the \smr\ Level2 data products
by frequency mode. It should be noted that these are the products
that are covered by the present reprocessing project, and not all
products that potentially can be derived from \smr\ observations.
