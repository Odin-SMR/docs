\chapter*{Notations}
\addcontentsline{toc}{chapter}{Notations}
\label{chapter:notations}

\section*{Definition of common terms}
\addcontentsline{toc}{section}{Definition of Common Terms}

\begin{description}
\item[Sun-synchronous orbit]
A Sun-synchronous orbit (sometimes called a heliosynchronous orbit) is a geocentric orbit which combines altitude and inclination in such a way that an object on that orbit will appear to orbit in the same position, from the perspective of the Sun, during its orbit around the Earth
\end{description}


\lcomment{PE}{Don't follow, please, give one example.}
\lcomment{JR}{Not at all necessary, but may be a nice feature}



\section*{Abbreviations} 
\addcontentsline{toc}{section}{Abbreviations}

\begin{description}
\item[OEM] Optimal Estimation Method
\end{description}


\lcomment{PE}{Please, set up a table structure and give one example.}
\lcomment{JR}{I chose the "description" environment instead of a table}



%%% Local Variables: 
%%% mode: latex
%%% TeX-master: "L2_ATBD"
%%% End: 
