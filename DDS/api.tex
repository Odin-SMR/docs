\chapter{API decription}
\label{chapter:api}
This section describes the API calls used to get data from from the DDS.  The
data is accessed through a hierarcical REST API where deeper URIs return more
specific data.  All call URIs have a common root \url{rest_api/v5}, which
has been omitted below for clarity.  All GET calls return JSON objects unless
otherwise noted. Key/value pairs are listed as name of the key
along with the type of the corresponding value within parantheses, followed
by a brief description of the contents.  See the sections on the different
data sources for specifications on the structure of their respective JSON
objects. This description is only for the \smr~version~3 Level2 data. For
details on how to access the collocated measurements from the VDS,
see~\cite{VDS:2016}, for more details on the Level2 data format
see~\cite{iodd, atbdl2data}, and for a more general description of the \smr~API,
see \url{http://odin.rss.chalmers.se/apidocs/index.html}.


\section{API calls}
\subsection{\url{level2/DDS/}}
Method: \emph{GET}

Get project information.

\paragraph{Parameters:} None.

\paragraph{Data structure:}
Returns JSON object with the following structure:

\begin{lstlisting}[basicstyle=\footnotesize]
{
  "Data": [
    {
      "FreqMode": <int>,
      "URLS": {
        "URL-comment": <string: URL for comments for scans>,
        "URL-failed": <string: URL for failed scans>,
        "URL-scans": <string: URL for scans>
      }
    },
    ...
  ],
  "Count": <int: number of FreqModes in project>,
  "Type": "level2_project_freqmode"
}
\end{lstlisting}


\subsection{\url{level2/DDS/area}}
\label{api:area}
Method: \emph{GET}

Get data in provided area.

\paragraph{Parameters:}
\begin{itemize}
    \item product \emph{(array of strings)}: Return data only for these
        products
    \item min\_lat \emph{(float)}: Minimum latitude (-90 to 90)
    \item max\_lat \emph{(float)}: Maximum latitude (-90 to 90)
    \item min\_lon \emph{(float)}: Minimum longitude (0 to 360)
    \item max\_lon \emph{(float)}: Maximum longitude (0 to 360)
    \item min\_pressure \emph{(float)}: Minimum pressure (Pa)
    \item max\_pressure \emph{(float)}: Maximum pressure (Pa)
    \item min\_altitude \emph{(float)}: Minimum altitude (m)
    \item max\_altitude \emph{(float)}: Maximum altitude (m)
    \item start\_time \emph{(date)}: Return data after this date (inclusive)
    \item end\_time \emph{(date)}: Return data before this date (exclusive)
\end{itemize}

Provide latitude and/or longitude limits to get data for a certain area of the
earth. If no latitude or longitude limits are provided, data for the whole
earth is returned. Choose between min/max altitude and min/max pressure.

\paragraph{Data structure:}
Returns JSON object with the following structure:

\begin{lstlisting}[basicstyle=\footnotesize]
{
  "Data": <array of Level2 Data Structures as described below>,
  "Count": <int: number of matching data>,
  "Type": "L2"
}
\end{lstlisting}


\subsection{\url{level2/DDS/locations}}
Method: \emph{GET}

Get data close to provided location.

\paragraph{Parameters:}
\begin{itemize}
    \item product \emph{(array of strings)}: Return data only for these
        products
    \item location \emph{(array of strings; required)}: Return data close to
        these locations ('lat,lon').
    \item radius \emph{(float; required)}: Return data within this radius from
        the provided locations (km).
    \item min\_pressure \emph{(float)}: Minimum pressure (Pa)
    \item max\_pressure \emph{(float)}: Maximum pressure (Pa)
    \item min\_altitude \emph{(float)}: Minimum altitude (m)
    \item max\_altitude \emph{(float)}: Maximum altitude (m)
    \item start\_time \emph{(date)}: Return data after this date (inclusive)
    \item end\_time \emph{(date)}: Return data before this date (exclusive)
\end{itemize}

Provide one or more locations and a radius to get data within the resulting
circles on the earth surface. Choose between min/max altitude and min/max
pressure.

\paragraph{Data structure:}
Returns JSON object with the same structure as endpoint~\ref{api:area}.


\subsection{\url{level2/DDS/products/}}
\label{api:products}
Method: \emph{GET}

Get available products.

\paragraph{Parameters:} None.

\paragraph{Data structure:}
Returns JSON object with the following structure:

\begin{lstlisting}[basicstyle=\footnotesize]
{
  "Data": <array of strings: product names>,
  "Count": <int: number of available products>,
  "Type": "level2_product_name"
}
\end{lstlisting}


\subsection{\url{level2/DDS/<string:date>/}}
Method: \emph{GET}

Get data for the provided date.

\paragraph{Parameters:}
\begin{itemize}
    \item product \emph{(array of strings)}: Return data only for these
        products
    \item min\_pressure \emph{(float)}: Minimum pressure (Pa)
    \item max\_pressure \emph{(float)}: Maximum pressure (Pa)
    \item min\_altitude \emph{(float)}: Minimum altitude (m)
    \item max\_altitude \emph{(float)}: Maximum altitude (m)
    \item start\_time \emph{(date)}: Return data after this date (inclusive)
    \item end\_time \emph{(date)}: Return data before this date (exclusive)
\end{itemize}

Choose between min/max altitude and min/max pressure.

\paragraph{Data structure:}
Returns JSON object with the same structure as endpoint~\ref{api:area}.


\subsection{\url{level2/DDS/<int:frequency mode>/comments/}}
Method: \emph{GET}

Get list of comments for a frequency mode.

\paragraph{Parameters:}
\begin{itemize}
    \item offset \emph{(int)}: Skip scans before returning
    \item limit \emph{(int; default: 1000)}: Number of scans to return
\end{itemize}

\paragraph{Data structure:}
Returns JSON object with the following structure:

\begin{lstlisting}[basicstyle=\footnotesize]
{
  "Data": [
    {
      "Comment": <string: comment from processing>,
      "URLS": {
        "URL-failed": <string: URL for failed scans with comment>,
        "URL-scans": <string: URL for successful scans with comment>
      }
    },
    ...
  ],
  "Count": <int: number of unique comments>,
  "Type": "level2_scan_comment"
}
\end{lstlisting}


\subsection{\url{level2/DDS/<int:frequency mode>/failed/}}
Method: \emph{GET}

Get list of matching scans that failed the level2 processing.

\paragraph{Parameters:}
\begin{itemize}
    \item start\_time \emph{(date)}: Return data after this date (inclusive)
    \item end\_time \emph{(date)}: Return data before this date (exclusive)
    \item comment \emph{(string)}: Return scans with this comment
    \item offset \emph{(int)}: Skip scans before returning
    \item limit \emph{(int; default: 1000)}: Number of scans to return
\end{itemize}

\paragraph{Data structure:}
Returns JSON object with the following structure:

\begin{lstlisting}[basicstyle=\footnotesize]
{
  "Data": [
    {
      "Date": <string: date>,
      "Error": <string: error message>,
      "ScanID": <int: scan number>,
      "URLS": {
        "URL-ancillary": <string: URL for Level2 ancillary data>,
        "URL-level2": <string: URL for Level2 data>,
        "URL-log": <string: URL for Level1 log data>,
        "URL-spectra": <string: URL for Level1 spectra>
      }
    },
    ...
  ],
  "Count": <int: number of matching scans>,
  "Type": "level2_failed_scan_info"
}
\end{lstlisting}


\subsection{\url{level2/DDS/<int:frequency mode>/products/}}
Method: \emph{GET}

Get available products for a given project and frequency mode.

\paragraph{Parameters:} None.

\paragraph{Data structure:}
Returns JSON object with the same structure as endpoint~\ref{api:products}.


\subsection{\url{level2/DDS/<int:frequency mode>/scans/}}
Method: \emph{GET}

Get list of matching scans.

\paragraph{Parameters:}
\begin{itemize}
    \item start\_time \emph{(date)}: Return data after this date (inclusive)
    \item end\_time \emph{(date)}: Return data before this date (exclusive)
    \item comment \emph{(string)}: Return scans with this comment
    \item offset \emph{(int)}: Skip scans before returning
    \item limit \emph{(int; default: 1000)}: Number of scans to return
\end{itemize}

\paragraph{Data structure:}
Returns JSON object with the following structure:

\begin{lstlisting}[basicstyle=\footnotesize]
{
  "Data": [
    {
      "Date": <string: date>,
      "ScanID": <int: scan number>,
      "URLS": {
        "URL-ancillary": <string: URL for Level2 ancillary data>,
        "URL-level2": <string: URL for Level2 data>,
        "URL-log": <string: URL for Level1 log data>,
        "URL-spectra": <string: URL for Level1 spectra>
      }
    },
    ...
  ],
  "Count": <int: number of matching scans>,
  "Type": "level2_scan_info"
}
\end{lstlisting}


\subsection{\url{level2/DDS/<int:frequency mode>/<int:scan number>/}}
Method: \emph{GET}

Get level2 data, info, comments, and ancillary data for one scan and frequency
mode.

\paragraph{Parameters:} None.

\paragraph{Data structure:}
Returns JSON object with the following structure:

\begin{lstlisting}[basicstyle=\footnotesize]
{
  "Data": {
    "L2": <Level2 Data Structure as described below>,
    "L2anc": <Level2 Ancillary Data Structure as described below>,
    "L2c": <Level2 Comments Data Structure as described below>,
    "L2i": <Level2 Info Data Structure as described below>,
  },
  "Count": null,
  "Type": "mixed"
}
\end{lstlisting}


\subsection{\url{level2/DDS/<frequency mode>/<scan number>/L2/}}
Method: \emph{GET}

Get level2 data for one scan and frequency mode.

\paragraph{Parameters}:
\begin{itemize}
    \item product \emph{(array of strings)}: Return data only for these
        products
\end{itemize}

\paragraph{Data structure:}
Returns JSON object with the following structure:

\begin{lstlisting}[basicstyle=\footnotesize]
{
  "Data": [
    {
      "AVK": <array of arrays of floats: averaging kernels at altitudes>,
      "Altitude": <array of floats: altitudes>,
      "Apriori": <array of floats: apriori data at altitudes>,
      "ErrorNoise": <array of floats: noise error at altitudes>,
      "ErrorTotal": <array of floats: total error at altitudes>,
      "FreqMode": <int: frequency mode>,
      "InvMode": "string",
      "Lat1D": <float: approximate latitude of retrieval>,
      "Latitude": <array of floats: latitudes for retrieval at altitudes>,
      "Lon1D": <float: approximate latitude of retrieval>,
      "Longitude": <array of floats: longitude for retrieval at altitudes>,
      "MJD": <float: Modified Julian Date for retrieval>,
      "MeasResponse": <array of floats: measurement response at altitudes>,
      "Pressure": <array of floats: pressure at altitudes [Pa]>,
      "Product": <string: product name>,
      "Quality": <int: quality flag>,
      "ScanID": <int: scan number>,
      "Temperature": <array of float: retrieved temperature at altitudes [K]>,
      "VMR": <array of float: retrieved volumetric mixing ratio at altitudes>
    }
  ],
  "Count": null,
  "Type": "L2"
}
\end{lstlisting}


\subsection{\url{level2/DDS/<int:frequency mode>/<int:scan number>/L2anc/}}
Method: \emph{GET}

Get ancillary data for one scan and frequency mode.

\paragraph{Parameters:} None.

\paragraph{Data structure:}
Returns JSON object with the following structure:

\begin{lstlisting}[basicstyle=\footnotesize]
{
  "Data": [
    {
      "FreqMode": <int: freqcuency mode>,
      "InvMode": <string: inversion mode used in retrieval>,
      "LST": <float: local solar time>,
      "Lat1D": <float: approximate latitude of retrieval>,
      "Latitude": <array of floats: latitudes for retrieval at altitudes>,
      "Lon1D": <float: approximate latitude of retrieval>,
      "Longitude": <array of floats: longitude for retrieval at altitudes>,
      "MJD": <float: Modified Julian Date for retrieval>,
      "Orbit": <int: orbit number>,
      "Pressure": <array of floats: pressure at altitudes [Pa]>,
      "SZA": <array of floats: solar zenith angle for retrieval at altitudes>,
      "SZA1D": <float: approximate solar zenith angle for retrieval>,
      "ScanID": <int: scan number>,
      "Theta": <array of floats: potential temperature>
    }
  ],
  "Count": 1,
  "Type": "string"
}
\end{lstlisting}


\subsection{\url{level2/DDS/<int:frequency mode>/<int:scan number>/L2i/}}
Method: \emph{GET}

Get level2 auxiliary data for one scan and frequency mode.

\paragraph{Parameters:} None.

\paragraph{Data structure:}
Returns JSON object with the following structure:

\begin{lstlisting}[basicstyle=\footnotesize]
{
  "Data": {
    "BlineOffset": <array of arrays of floats: baseline offsets for spectra>,
    "ChannelsID": <array of floats: channel identifier>,
    "FitSpectrum": <array of arrays	 of floats: fitted spectra [K]>
    "FreqMode": <int: frequency mode>,
    "FreqOffset": <float: retrieved LO frequency offset [Hz]>,
    "InvMode": <string: inversion mode used in retrieval>,
    "LOFreq": <array of floats: LO frequency for each spectrum in scan [Hz]>,
    "MinLmFactor": <float: minimum Levenberg-Marquard factor of OEM>,
    "PointOffset": <float: retrieved pointing offset [degree],
    "Residual": <float: residual of retrieved and measured spectra>,
    "SBpath": <float: sideband path used for retrieving spectra [m]>,
    "STW": <array of floats: satellite time words for spectra>,
    "ScanID": <int: scan number>,
    "Tsat": <float: satellite onboard temperature [K]>,
    "URLS": {
      "URL-ancillary": <string: URL for Level2 ancillary data>,
      "URL-level2": <string: URL for Level2 data>,
      "URL-log": <string: URL for Level1 log data>,
      "URL-spectra": <string: URL for Level1 spectra>
    }
  },
  "Count": null,
  "Type": "L2i"
}
\end{lstlisting}


\subsection{\url{level2/DDS/<int:frequency mode>/<int:scan number>/L2c/}}
Method: \emph{GET}

Get level2 comments for one scan and frequency mode.

\paragraph{Parameters:} None.

\paragraph{Data structure:}
Returns JSON object with the following structure:

\begin{lstlisting}[basicstyle=\footnotesize]
{
  "Data": <array of strings: comments>
  "Count": <int: number of comments for scan>,
  "Type": L2c
}
\end{lstlisting}


\section{Example usage}
\label{sec:api_usage}
This is a brief example of how to use the \smr~API to access the DDS in Python.
The basic procedure for navigating the call hierarchy is the same in all major
programming languages and browsers.

\lstinputlisting[language=Python, basicstyle=\footnotesize]{example.py}
