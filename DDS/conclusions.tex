\chapter{Conclusions}
\label{chapter:conclusions}
\section{General comments}


Table~\ref{table:level2:v3} shows an overview of the characteristics of the
main data products from the updated \smr\ processing chain for the frequency
modes investigated in Chapter~\ref{chapter:comparisons}. The vertical coverage
has been defined as the altitude interval where the \emph{measurement response}
for the retrievals is >0.8. The vertical resolution and precision are
calculated from the $\pm 2\sigma$ percentiles of the \emph{full widths at half
maximum} of the averaging kernels and the \emph{total error} respectively.

The results presented in Table~\ref{table:level2:v3} should be compared with
the results for the old processing chain, presented in
Tables~\ref{table:level2} and~\ref{table:level2b}.

\begin{table}
    \caption{Characteristics of \smr\ Level2 main data products.}
    \label{table:level2:v3}
    \scalebox{0.965}{%
    \begin{tabular}{lrrrr}
        \toprule
        \textbf{Product} & \textbf{Frequency} & \textbf{Vertical}   & \textbf{Vertical}     & \textbf{Precision} \\
                         & \textbf{{[}GHz{]}} & \textbf{coverage}   & \textbf{resolution}   &                    \\
        \midrule  % FM 01:
        \chem{O_{3}}     & 501.5 (FM~01)      & \(\sim\)19--53\,km  & 5.3--5.5\,km          & 0.70--0.85\,ppmv \\
        \chem{ClO}       & 501.3 (FM~01)      & \(\sim\)18--58\,km  & 5.3--5.6\,km          & 0.07--0.08\,ppbv \\
        \chem{N_{2}O}    & 502.3 (FM~01)      & \(\sim\)15--62\,km  & 4.4--5.6\,km          & 5.00--6.10\,ppbv \\

        \midrule  % FM 02:
        \chem{O_{3}}     & 544.9 (FM~02)      & \(\sim\)17--77\,km  & 2.9--4.9\,km          & 0.18--0.26\,ppmv \\
        \chem{HNO_{3}}   & 544.4 (FM~02)      & \(\sim\)20--61\,km  & 5.0--7.2\,km          & 0.24--0.28\,ppbv \\
        Temperature      & 544.9 (FM~02)      & \(\sim\)21--64\,km  & 7.5--7.9\,km          & 1.13--2.18\,K \\

        \midrule  % FM 08:
        \chem{O_{3}}     & 489.2 (FM~08)      & \(\sim\)16--71\,km  & 4.1--6.0\,km          & 0.29--0.39\,ppmv \\
        \chem{H_{2}O}    & 489.5 (FM~08)      & \(\sim\)19--78\,km  & 4.0--4.7\,km          & 0.32--0.55\,ppmv \\

        \midrule  % FM 13:
        \chem{O_{3}}     & 556.9 (FM~13)      & \(\sim\)44--80\,km  & 6.1--6.7\,km          & 0.19--0.24\,ppmv \\
        \chem{H_{2}O}    & 556.7 (FM~13)      & \(\sim\)44--110\,km & 4.3--4.8\,km          & 0.06--0.33\,ppbv \\
        Temperature      & 556.9 (FM~13)      & \(\sim\)44--95\,km  & 5.3--5.6\,km          & 1.85--2.73\,K \\

        \midrule  % FM 19:
        \chem{O_{3}}     & 556.9 (FM~19)      & \(\sim\)43--81\,km  & 5.8--6.6\,km          & 0.19--0.24\,ppmv \\
        \chem{H_{2}O}    & 556.7 (FM~19)      & \(\sim\)43--109\,km & 4.1--4.4\,km          & 0.13--0.26\,ppbv \\
        Temperature      & 556.9 (FM~19)      & \(\sim\)43--93\,km  & 5.2--5.5\,km          & 1.72--2.65\,K \\
        \bottomrule
    \end{tabular}}
\end{table}

\section{Recomendations}
\subsection{Ozone}
The best ozone product for general use is that from FM2 .  The new product is in much better agreement with other instruments and does not seem to show any temporal drifts although a thorough analysis of drifts has not yet been attempted.   The other ozone products show some biases 
but could be used in conjunction with simultaneously measured products if account is taken of this. 

\subsection{Chlorine Monoxide}

 The ClO product should be reliable until 2009 and after jan 2018. Unfortunately a gradually deteriorating instrument malfunction has disturbed the intervening period.  We will attempt to correct for this at soon as possible
 
 \subsection{Water vapour}
 Water vapour continues to be a difficult product. the FM08 product has improved and is stable over time although shows a 15 % low bias compared to MLS.  For the upper stratosphere and mesosphere FM13 appears to be the best product but alsoe shows a low bias of around 15% and som variability above 70 km that needs further investigation.
 \subsection{Temperature}
 
 FM02 provides stratospheric temperatures with a possible increasing bias with altitude. There does not appear to be any systematic drift with tiime.  For higher altitudes FM13 can be used with knowledge of a possible cold bias of 3-5 K
 
 

