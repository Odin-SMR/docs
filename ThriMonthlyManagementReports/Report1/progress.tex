\chapter{Progress since Project Start}
\label{chapter:progress}

\section{Deliverables}

\begin{itemize}

\item{\bf D-1 - Project Management Plan}

The PMP was delivered to ESA on the 30th of September 2015.

\item{\bf D-2 - Tri-Monthly Management Report no.1}

This document constitutes the first tri-monthly management report. 

\item{\bf D-6 - ATBD v.1}

An ATBD covering the Level 2 algorithms has been delivered to ESA on the 16th of November 2015.

\item{\bf D-13 - Project Web Site}

A new web site for the project and all relevant data has been launched. It can be found at odin.rrs.chalmers.se.

\end{itemize}

\section{Work Packages}

\begin{itemize}

\item{\bf WP 1.1.1.1 - Characterisation of Data}

The Odin/SMR package is flexible and tunable and is equipped 
with four sub-mm receivers, and observations are performed
in a number of modes.

A review of Level0 and Level1 data has been performed.
Time-series plots of relevant Level0/1 data has been produced
for the various modes, with the aim of identifying and characterising 
trends and drifts in the data.
Some trends and sudden changes in basic Level1 data,
such as receiver noise temperature and antenna spillover 
contribution, has been found, and these were further analysed in W.P. 1.1.1.2.

\item{\bf WP 1.1.1.2 - Identify Corrupt Data}

Trends in basic Level1 data (receiver noise temperature (Trec),
and antenna spillover contribution (Tspil) ) has been analysed.
Some of the variation in these parameters is correlated to the
variation in the ambient temperature of the satellite.
The ambient temperature has decreased by about
5 K during the mission, and during northern hemispheric summer
the temperature is about 10 K colder than during the remainder of the year. 
A sudden change (april 2007 ) in Trec was found in data from
the 549 GHz receiver, which seems to be related to the tuning
of the receiver.
Small trends in the calibrated Level1B spectra were found,
but if these are artefacts or not is difficult 
to judge from only looking at Level1 data. 
This issue is investigated further in W.P 1.1.1.3 

\item{\bf WP 1.1.1.4 - Analysis of Algorithm }

The calibration algorithm has been analysed and documented.

\item{\bf WP 1.1.2.1 - Adjustment of Algorithm}

The calibration algorithm has been modified based on work already done in the fall of 2014, resulting in v.8 described more in depth in WP 1.1.2.3 in the next section.

\item{\bf WP 1.1.2.2 - Parameter Adjustments}

The parameters in the calibration algorithm have been adjusted based on work already done in the fall of 2014, resulting in v.8 described more in depth in WP 1.1.2.3 in the next section.

\item{\bf WP 2.1.1 - Review of State of the Art SMR L2 Algorithms}

A critical review of the state-of-the-art SMR Level 2 algorithms
(Forward model algorithm ARTS, Inversion algorithm OEM) has been made based on peer-reviewed scientific and technical publications. The results of this review are incorporated in the Level 2 ATBD.

\item{\bf WP 2.1.2 - Review of end-to-end Uncertainty Analysis of L2 Data}

 A review based on peer-reviewed scientific and technical publications covering uncertainty analysis of SMR L2 products has been made. The results of this review are incorporated in the Level 2 ATBD.

\item{\bf WP 2.1.4 - Write Algorithm Theoretical Basis Document}

An ATBD for the Level 2 processing has been produced which provides the theoretical basis, the physical theory, the mathematical procedures and the possible assumptions being applied.

\item {\bf WP 4.1.1 - Update Web Service for Publication of Data and Documentation}

A new web site has been launched, it is located at the adress odin.rss.chalmers.se. It contains information about the Odin satellite, the project and the involved entities and personnel, as well as an interface for downloading and viewing L1 and L2 data.   

\item {\bf WP 4.1.2 - Approval of Web Site Layout and Content by the Agency}

The layout and contents of the webpage was reviewed 2015-10-18 by Bojan Boykov at ESA. 

\end{itemize}


