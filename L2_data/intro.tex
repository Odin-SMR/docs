\chapter{Aim and Scope of this Document}
\label{chapter:intro}

% The page numbering must be reset here inside the file
\pagenumbering{arabic}
\setcounter{page}{1}

The Sub-Millimetre Radiometer (SMR), aboard the Odin satellite, has been in operation since 2001 and performs passive limb measurements of the atmosphere, mainly at frequencies around 500\,GHz. From these measurements, profiles of species that are of interest for studying stratospheric and mesospheric chemistry and dynamics can be derived, such as \chem{O_3}, \chem{ClO}, \chem{N_{2}O}, \chem{NO}, \chem{HNO_{3}}, \chem{H_{2}O}, \chem{CO}, and isotopologues of \chem{H_{2}O}, and \chem{O_{3}}. These profiles are referred to as Level2 (L2) products of \smr. They are generated by a Level2 processor. The input to the Level2 processor is geolocated and calibrated measurements (\smr\ Level1B data, described in \citet{atbdl1b}) and dynamic and static auxilliary/ancilliary data.

The aim of this document is to provide an user-friendly description of the L2 data. More details about the L2 processing system and the algorithms applied by the L2 processor are available in the Input/Output Data Definition Document (IODD) \citep{iodd} and in the Algorithms Theoretical Basis Document (ATBD) - Level 2 processing \citep{atbdl2}, respectively. 
  
