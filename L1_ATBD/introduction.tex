\chapter{Introduction}
\label{chapter:introduction}

% The page numbering must be reset here inside the file
\pagenumbering{arabic}
\setcounter{page}{1}


\section{Aim and scope of this document}
\label{sec:aim}

The sub-millimetre radiometer (\SMR\ ) onboard the Odin satellite
performs limb sounding measurements of the atmosphere.
The basic output of \smr\ is spectra in different frequnency
bands, mainly within the 486\,--\,504\,GHz and 541\,--\,581\,GHz region.
After calibration, a group of such such spectra, can be used
to retrieve profiles of e.g. \chem{O_3}, \chem{ClO}, \chem{N_{2}O}, \chem{HNO_{3}}, 
\chem{H_{2}O}, \chem{CO}, and isotopes of \chem{H_{2}O}, and \chem{O_{3}},
which are species that are of interest for studiying stratospheric and 
mesospheric chemistry and dynamics. 

The processing of data, from basic instrumental data to the
desired species concentration product, involves a number of steps.
It is standard practise to devide data products into different levels, as:

\begin{itemize}

\item Level0 data: time tagged and sorted science and house keeping data
and orbit and pointing data in separate files

\item Level1B data: groups of spectra, calibrated both in intensity and frequency. 
Pointing and instruments settings included.

\item Level2 data: extracted geophysical data (i.e. geolocated profile
of ozone concentration) from observed spectra

\end{itemize}


The aim of this document is to review and describe basic
data (Level0) from \smr\, and the algorithms used to 
to processes basic Level0 data into gelocated and calibrated 
groups of spectra (scans), which is refered to as Level1B data. 
The aim is also to describe the Level1B data format and 
the quality of the data.

This document is organised as following:
\smr\ observations and measurement modes are described 
in Chapter~1. The calibration procedure is presented in
Chapter~2, and the Level1b data format, including description
of quality flags, is described in Chapter~3.
Finally,
Chapter 4 gives a summary, with focus on the most important points to correctly
understand the \smr\ Level1B data. A list of used acronyms is found directly after
the table of contents.
   

\section{Odin}

The Odin satellite was launched on 2001-02-20, into a sun-synchronous
18:00 hour ascending node orbit, carrying two co-aligned limb sounding
instruments: OSIRIS (optical spectrograph and infrared imaging system) and
\SMR\ (sub-millimetre radiometer). Originally, Odin was used for both
atmospheric and astronomical observations, while since 2007 only its aeronomy
mission is active. Odin is a Swedish-led project, in cooperation with Canada,
France and Finland. Both of Odin's instruments are still functional, and the
present operation of the satellite is partly performed as a ESA third party
mission.
Odin passes the ground station at Esrange about 14 times each
and raw science and house-keeping data are downloaded at
Esrange and transferred to a data archive housed by the Parallel
Data Centre (PDC) at the Royal Institute of Technology (KTH)
in Stockholm.


\subsection{The \SMR\ instrument}


\begin{table}
\caption{ \smr\ frontend and backend frequency specification}
\label{table:config}
\begin{tabular}{|l|l|l|l|l|}
  \hline
  \textbf{Frontend} & \textbf{Tuning range} & \textbf{Backend} & \textbf{Bandwidth } & \textbf{Channel spacing} \\
                    & {[}GHz{]}             &                  & {[}MHz{]}           & {[}MHz{]}\\
  \hline
  549 A1            & 541 - 558             & AOS              & 1050                & 0.61\\
  \hline
  495 A2            & 486 - 504             & AC1/AC2          & 800                 & 1.0
 \\
 \cline{1-1}
 \cline{2-2}
 \cline{4-4}
 \cline{5-5}
  555 B1           & 547 - 564              &                 & 400                  & 0.5 \\
 \cline{1-1}
 \cline{2-2}
 \cline{4-4}
 \cline{5-5}
 572 B2            & 563 - 581              &                 & 200                  & 0.25 \\
 \cline{1-1}
 \cline{2-2}
 \cline{4-4}
 \cline{5-5}
  119 C           &  118.75                 &                 & 100                 & 0.125 \\
\hline
\end{tabular}
\end{table}

\begin{table}
\caption{Operational main scanning modes}
\label{table:scanpattern}
\begin{tabular}{|l|l|}
  \hline
  \textbf{Scanning mode} & \textbf{Scanning range {[}km{]}} \\
  \hline
  Stratospheric scan     &  7 -72 \\
 \hline
 Stratospheric/mesospheric scan &  7-110  \\
 \hline
 Summer mesospheric scan & 60-100 \\
 \hline
\end{tabular}
\end{table}



The \smr\ package is highly tunable and flexible \citep{frisk:theod:03}.
\smr\ is equipped with one mm and four sub-mm receivers.
The four main sub-mm receiver chains can be tuned to cover
frequencies in the ranges 486\,--\,504\,GHz and 541\,--\,581\,GHz.
Table~\ref{table:config} lists the receivers single sideband (SSB) tuning ranges.


The receivers form two groups, A1+A2 and B1+B2+C, where each group shares
a common path through the beam optics. 
The main observing mode is by switching against the cold sky, between the
main beam and an unfocused sky beam. In switching mode one group will
receive its signal via the main beam when the other sees the reference signal,
and vice versa.
The two auto-correlators (ACs) of \smr\ can be coupled to any of the front-ends,
and uses each up to 8 x 112 MHz bands.
Eight correlator ships provides 96 lags each.
The center IF (intermediate frequency) of each sub-band is tuneable
within a 500\,MHz wide band in 1\,MHz step. The channel spacing
ranging from 125\,KHz in 1 band mode to 1\,MHz in 8 band mode.
\smr\ can be tuned to over a wide frequency range, but the maximum total instantaneous
bandwidth is only 1.6\,GHz. 

In the configuration applied for
atmospheric sounding, the channels of the ACs have a spacing of 1\,MHz,
(while the frequency resolution in Level1b spectra is only 2\,MHz 
due to the fact that a Hanning smoothing is applied in data processing).
To cover all molecular transitions of interest, a 
number of ``observation modes'' has been defined. Each observation mode makes
use of two or three frequency bands. 
Definition of frequency modes are found in Tables~\ref{table:config2}, 
~\ref{table:config3}, and~\ref{table:config4}, in Appendix A.
Single sideband operation is obtained by tunable Martin-Pupplet
interferometers. The nominal sideband suppression is better than 19\,dB across
the image band.
\smr\ has also a receiver chain around the 118\,GHz oxygen transition, that was
heavily used during Odin's astronomy mission. For the atmospheric mission, this
front-end was planned to be used for retrieving temperature profiles, but a
technical problem (drifting LO frequency) and the fact that the analyses
requires treatment of Zeeman splitting have given these data low priority. 

The main reflector of \smr\ has a diameter of 1.1\,m, giving a
vertical resolution at the tangent point of about 2\,km. 
The vertical scanning is achieved by a rotation of the satellite
platform. Measurements are in general performed along the orbit plane, providing a
latitude coverage between 82.5$^{\circ}$S and 82.5$^{\circ}$N. Since the end of
2004 Odin is also pointing off-track during certain periods, e.g.\ during the
austral summer season, allowing the latitudinal coverage to be extended towards
the poles. 




\subsection{Measurement sequence}


The vertical scanning of \smr\'s line of sight is achieved by a
rotation of the platform, with a rate matching a vertical speed of
the tangent altitude of 750\,m/s. Measurements are performed during
both upward and downward scanning. The lower end of the scan is typically
at about 7\,km, the upper end varies between 70 and 110\,km, depending on
observation mode.

For calibration purposes \smr\ performs areonomy observation
in a switching mode, i.e. switching between the main beam and
an unfocused sky beam of 4.4$^{\circ}$ FWHM at all wavelengths. This is done
by means of a chopper wheel. An internal selection mirror
can choose between two possible directions of the sky beam,
separated by 28$^{\circ}$. In aeronmoy mode the beam pointing
away from the Earth is used. The separation from the main beam
amounts to 42$^{\circ}$.
The selection mirror may also direct the beam towards an internal
load at ambient temperature.

In nominal operation every other recorded signal comes from an
unfocused cold sky beam, except around the lower and upper turning
points of the scan where the reference beam is directed towards
the internal load (typically three consecutive load spectra are recorded).
The load acts as a blackbody emitter at
an ambient temperature of around 285 K.



\section{Further reading}
\label{sec:reading}

\citet{murtagh:anove:02} give an overview of the Odin aeronomy mission, as well
as the general technical details of \smr. \citet{frisk:theod:03} give a description
of the technical details of \smr.


