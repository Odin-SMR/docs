\section{Frequency mode 08}
\label{sec:fm08}

\subsection{Overview}
\label{sec:fm08:overview}
Frequency mode~08 monitors two bands, $488.950$--$489.350\,\mathrm{GHz}$ and
$488.350$--$488.750\,\mathrm{GHz}$. Its main use is retrievals of water vapour
in the lower stratosphere. \chem{O_3} is also retrieved.  This mode showed
considerable sideband leakage and served as the model for a new sideband
correction that has been applied to all bands.  Spectra from this observation
mode are shown in Figure~\ref{fig:spectra:08}.  Due to some miscommunication two different
command macros were utilised for this mode resulting is slight different frequency coverage and 
different settings for the sideband filter.

\begin{figure}[ht]
    \centering
    \includegraphics[width=0.95\textwidth]{../DDS/figures/spectra/fm_08_spectra_winter_97u}
    \caption{Annual median spectra for FM~08 for altitude interval 40--50~km at
        equatorial latitudes for the $97\,\mathrm{\mu m}$ sideband path during
        the arctic winter.
    }\label{fig:spectra:08}
\end{figure}


\subsection{Comparison of retrieved profiles}
\label{sec:fm08:comparison}


%%%%%%
% O3 %
%%%%%%

\subsubsection{\chem{O_3}}
\label{sec:fm08:comparison:O3}
The retrievals for \chem{O_3} have been compared with data from the MIPAS, MLS,
OSIRIS and SAGE~III instruments. Annual average differences to these
instruments are shown in Figure~\ref{fig:fm08:O3:profiles}. In
Figure~\ref{fig:fm08:O3:scatter} individual retrievals for the instruments for
the entire period are plotted against the retrievals from the new and old
versions of the \smr\ processing chain. The results show a considerable
improvement with the updated version of the processing compared to all
considered instruments. However there are indications of a change in the last three years 
that requires further investigation. 
The over-all correlation and coherency is much better,
though a systematic under estimation remains.
Figure~\ref{fig:fm08:O3:mr_avk} suggests that the product is useful over the
range 16--70~km with a vertical resolution of around 5~km.

\begin{figure}[tbhp]
    \centering
    \begin{subfigure}[b]{0.49\textwidth}
        \includegraphics[width=\textwidth]{ALL-Strat-v3_0_0_fm8_O3_perdiff_mipas}
        \caption{average difference to MIPAS}
        \label{fig:fm08:O3:profiles:MIPAS}
    \end{subfigure}
    \,
    \begin{subfigure}[b]{0.49\textwidth}
        \includegraphics[width=\textwidth]{ALL-Strat-v3_0_0_fm8_O3_perdiff_mls}
        \caption{average difference to MLS}
        \label{fig:fm08:O3:profiles:MLS}
    \end{subfigure}

    \begin{subfigure}[b]{0.49\textwidth}
        \includegraphics[width=\textwidth]{ALL-Strat-v3_0_0_fm8_O3_perdiff_osiris}
        \caption{average difference to OSIRIS}
        \label{fig:fm08:O3:profiles:OSIRIS}
    \end{subfigure}
    \,
    \begin{subfigure}[b]{0.49\textwidth}
        \includegraphics[width=\textwidth]{ALL-Strat-v3_0_0_fm8_O3_perdiff_sage}
        \caption{average difference to SAGE~III}
        \label{fig:fm08:O3:profiles:SAGEIII}
    \end{subfigure}
    \caption{Average difference in percent between retrievals of \chem{O_3}
    from \smr~v3 and collocated measurements from various instruments at
    different altitudes for frequency mode~08.}

    \label{fig:fm08:O3:profiles}
\end{figure}

\begin{figure}[tbhp]
    \centering
    \begin{subfigure}[b]{0.49\textwidth}
        \includegraphics[width=\textwidth]{ALL-Strat-v3_0_0_fm8_O3_scatter_v2}
        \caption{correlation of collcated instruments with \smr~v2.X}
        \label{fig:fm08:O3:scatter:v2}
    \end{subfigure}
    \,
    \begin{subfigure}[b]{0.49\textwidth}
        \includegraphics[width=\textwidth]{ALL-Strat-v3_0_0_fm8_O3_scatter_v3}
        \caption{correlation of collcated instruments with \smr~v3}
        \label{fig:fm08:O3:scatter:v3}
    \end{subfigure}
    \caption{Correlation between retrievals of \chem{O_3} using \smr\
    versions~2.X and~3 and collocated measurements from various instruments
    for frequency mode~08.}
    \label{fig:fm08:O3:scatter}
\end{figure}

\begin{figure}[tbhp]
    \centering
    \begin{subfigure}[b]{0.49\textwidth}
        \includegraphics[width=\textwidth]{ALL-Strat-v3_0_0_fm8_O3_mr}
        \caption{median measurement response with $1\sigma$ and $2\sigma$
        percentiles}
        \label{fig:fm08:O3:mr}
    \end{subfigure}
    \,
    \begin{subfigure}[b]{0.49\textwidth}
        \includegraphics[width=\textwidth]{ALL-Strat-v3_0_0_fm8_O3_avk}
        \caption{median averaging kernels\newline~}
        \label{fig:fm08:O3:avk}
    \end{subfigure}
    \caption{Measurement response and averaging kernels for \chem{O_3}
    retrievals for \smr~v3 at different altitudes for frequency mode~08.}
    \label{fig:fm08:O3:mr_avk}
\end{figure}


%%%%%%%
% H2O %
%%%%%%%

\subsubsection{\chem{H_2O}}
\label{sec:fm08:comparison:H2O}
The retrievals for \chem{H_2O} have been compared with data from the MIPAS,
MLS and SAGE~III instruments. Annual average differences to these instruments
are shown in Figure~\ref{fig:fm08:H2O:profiles}. In
Figure~\ref{fig:fm08:H2O:scatter} individual retrievals for the instruments for
the entire period are plotted against the retrievals from the new and old
versions of the \smr\ processing chain. Though the results show a considerably
improved coherency with the updated version of the processing compared to all
considered instruments, the correlation is still poor, and the water content is
still systematically underestimated, in particular for higher concentrations. Again there
appears to have been a change in performance in the last few years.
Figure~\ref{fig:fm08:H2O:mr_avk} suggests that the product is useful over the
range 19--78~km with a vertical resolution of around 4.5~km.

\begin{figure}[tbhp]
    \centering
    \begin{subfigure}[b]{0.49\textwidth}
        \includegraphics[width=\textwidth]{ALL-Strat-v3_0_0_fm8_H2O_perdiff_mipas}
        \caption{average difference to MIPAS}
        \label{fig:fm08:H2O:profiles:MIPAS}
    \end{subfigure}
    \,
    \begin{subfigure}[b]{0.49\textwidth}
        \includegraphics[width=\textwidth]{ALL-Strat-v3_0_0_fm8_H2O_perdiff_mls}
        \caption{average difference to MLS}
        \label{fig:fm08:H2O:profiles:MLS}
    \end{subfigure}

    \begin{subfigure}[b]{0.49\textwidth}
        \includegraphics[width=\textwidth]{ALL-Strat-v3_0_0_fm8_H2O_perdiff_sage}
        \caption{average difference to SAGE~III}
        \label{fig:fm08:H2O:profiles:SAGEIII}
    \end{subfigure}
    \caption{Average difference in percent between retrievals of \chem{H_2O}
    from \smr~v3 and collocated measurements from various instruments at
    different altitudes for frequency mode~08.}
       \begin{subfigure}[b]{0.49\textwidth}
        \includegraphics[width=\textwidth]{ALL-Strat-v3_0_0_fm8_H2O_perdiff_mipas_esa}
        \caption{average difference to MIPAS - ESA}
        \label{fig:fm08:H2O:profiles:MIPAS-ESA}
    \end{subfigure}

    \label{fig:fm08:H2O:profiles}
\end{figure}

\begin{figure}[tbhp]
    \centering
    \begin{subfigure}[b]{0.49\textwidth}
        \includegraphics[width=\textwidth]{ALL-Strat-v3_0_0_fm8_H2O_scatter_v2}
        \caption{correlation of collcated instruments with \smr~v2.X}
        \label{fig:fm08:H2O:scatter:v2}
    \end{subfigure}
    \,
    \begin{subfigure}[b]{0.49\textwidth}
        \includegraphics[width=\textwidth]{ALL-Strat-v3_0_0_fm8_H2O_scatter_v3}
        \caption{correlation of collcated instruments with \smr~v3}
        \label{fig:fm08:H2O:scatter:v3}
    \end{subfigure}
    \caption{Correlation between retrievals of \chem{H_2O} using \smr\
    versions~2.X and~3 and collocated measurements from various instruments
    for frequency mode~08.}
    \label{fig:fm08:H2O:scatter}
\end{figure}

\begin{figure}[tbhp]
    \centering
    \begin{subfigure}[b]{0.49\textwidth}
        \includegraphics[width=\textwidth]{ALL-Strat-v3_0_0_fm8_H2O_mr}
        \caption{median measurement response with $1\sigma$ and $2\sigma$
        percentiles}
        \label{fig:fm08:H2O:mr}
    \end{subfigure}
    \,
    \begin{subfigure}[b]{0.49\textwidth}
        \includegraphics[width=\textwidth]{ALL-Strat-v3_0_0_fm8_H2O_avk}
        \caption{median averaging kernels\newline~}
        \label{fig:fm08:H2O:avk}
    \end{subfigure}
    \caption{Measurement response and averaging kernels for \chem{H_2O}
    retrievals for \smr~v3 at different altitudes for frequency mode~08.}
    \label{fig:fm08:H2O:mr_avk}
\end{figure}


\subsection{Discussion}
\label{sec:fm08:discussion}
The Pearson correlation between the \smr\ retrievals and the other instruments
was calculated for the entire period for both versions of the processing chain.
The results are summarised in Table~\ref{tab:fm08:stats}, and show that the
new algorithm is a improvement compared to all the instruments for all species
used in this investigation. The improvement is considerable for both \chem{O_3}
and \chem{H_2O}. However there still seems to be a systematic underestimation
of both products. It is unclear if this is due to spectroscopic problems or
some sort of underestimation of the intensity.  Spectroscopic data are taken
from the HITRAN~12 database updated with the latest laboratory results;
see~\cite{atbdl2}.



\begin{table}[tbhp]
\centering
\caption{Pearson correlation and fit parameters of the old and new \smr\
retrievals for frequency mode~08, compared with collocated data from other
instruments for the period 2003--2014.}
\label{tab:fm08:stats}
\begin{tabular}{lllrrrr}
    \toprule
    \textbf{Species} & \textbf{Instrument} & \textbf{SMR} & \textbf{corr.} & \textbf{slope} & \textbf{intercept} & \textbf{$\left|\left<\right.\right.$res.$\left.\left.\right>\right|$} \\
    \midrule
    \chem{O3}   & MIPAS         & v3    & 0.851 & 1.004 & -0.207\,ppm   & 1.582\,ppm \\
                &               & v2.x  & 0.911 & 0.872 & -0.059\,ppm   & 1.179\,ppm \\
    \cline{2-7}
                & MIPAS (ESA)   & v3    & 0.843 & 0.956 & -0.052\,ppm   & 1.639\,ppm \\
                &               & v2.x  & 0.906 & 0.831 &  0.085\,ppm   & 1.260\,ppm \\
    \cline{2-7}
                & MLS           & v3    & 0.879 & 1.001 & -1.130\,ppm   & 1.378\,ppm \\
                &               & v2.x  & 0.878 & 0.884 &  0.031\,ppm   & 1.238\,ppm \\
    \cline{2-7}
                & OSIRIS        & v3    & 0.853 & 0.983 &  0.132\,ppm   & 1.408\,ppm \\
                &               & v2.x  & 0.909 & 0.913 & -0.091\,ppm   & 1.089\,ppm \\
    \cline{2-7}
                & SAGE III      & v3    & 0.837 & 1.078 & -0.144\,ppm   & 0.566\,ppm \\
                &               & v2.x  & 0.737 & 0.975 & -0.324\,ppm   & 0.650\,ppm \\
    \midrule
    \chem{H_2O} & MIPAS         & v3    & 0.426 & 0.596 & 1.851\,ppm    & 2.026\,ppm \\
                &               & v2.x  & 0.432 & 0.377 & 2.816\,ppm    & 1.662\,ppm \\
    \cline{2-7}
                & MIPAS (ESA)   & v3    & 0.428 & 0.437 & 2.892\,ppm    & 2.250\,ppm \\
                &               & v2.x  & 0.455 & 0.320 & 3.146\,ppm    & 1.854\,ppm \\
    \cline{2-7}
                & MLS           & v3    & 0.490 & 0.702 & 0.870\,ppm    & 2.094\,ppm \\
                &               & v2.x  & 0.411 & 0.486 & 2.022\,ppm    & 1.907\,ppm \\
    \cline{2-7}
                & SAGE III      & v3    & 0.608 & 0.496 & 2.106\,ppm    & 1.502\,ppm \\
                &               & v2.x  & 0.285 & 0.161 & 4.460\,ppm    & 1.567\,ppm \\
    \bottomrule
\end{tabular}
\end{table}

\subsection{Conclusions}
\label{sec:fm08:conclusions}
Based on the discussion above, retrievals based on frequency mode~08 can be
used with confidence for the species \chem{O_3} while noting that the values
may be underestimated by 5--10~\%.  The data  should be used with some caution
for \chem{H_2O} but this requires more investigation.
