\section{Frequency mode 13}
\label{sec:fm13}

\subsection{Overview}
\label{sec:fm13:overview}
Frequency mode~13 monitors the band $556.598$--$557.398\,\mathrm{GHz}$. Its
main use is retrievals of \chem{H_2O} and \chem{O_3}.
This is the strongest water vapour line available in the sub-mm wave spectrum
and is used to reach the  highest possible altitude particularly to study
water vapour around the mesopause region.  The line centre is saturated up to
approximately 90~km and therefore temperatures can be derived throughout the
mesosphere. This band and FM~19 remain problematic and have been subject to a 
dedicated study~(\cite{grieco2020b}) since the first version of this report  to understand the effect of instrumental 
and other influences.  Various instrumental uncertainties such as sideband filter values and non-linearity effects in the amplifiers and spectrometers have been investigated without any clear conclusion.  However a value for the maximum suppression value different from that in the first version of the report was selected leading to some improvements in the agreement in both temperature and water vapour.   Even the uncertainty in the pressure broadening coefficient has been revisited.  
Spectra from this observation mode are shown in Figure~\ref{fig:spectra:13}.

\begin{figure}[ht]
    \centering
    \includegraphics[width=0.95\textwidth]{../DDS/figures/spectra/fm_13_spectra_winter}
    \caption{Annual median spectra for FM~13 for altitude interval 65--75~km at
        equatorial latitudes during the arctic winter.
    }\label{fig:spectra:13}
\end{figure}


\subsection{Comparison of retrieved profiles}
\label{sec:fm13:comparison}


%%%%%%
% O3 %
%%%%%%

\subsubsection{\chem{O_3}}
\label{sec:fm13:comparison:O3}
The retrievals for \chem{O_3} have been compared with data from the MIPAS, MLS
and OSIRIS instruments. Annual average differences to these instruments are
shown in Figure~\ref{fig:fm13:O3:profiles}. In Figure~\ref{fig:fm13:O3:scatter}
individual retrievals for the instruments for the entire period are plotted
against the retrievals from the new and old versions of the \smr\ processing
chain. The results show a better overall coherency with the updated version of
the processing compared to all considered instruments, but a systematic under
estimation of the concentrations has been introduced. The reason for this is
that a previously introduced  (ver 2.1) empirical correction to the intensities
has been remove since there was no physical basis for its inclusion. We suspect
that some sort of non-linearity is causing the underestimation. Investigations
of this band are continuing.  Above about 65 km diurnal variability in the
ozone concentration introduce problems when comparing with instruments on
different platforms.   We assume that this is the main reason for the
differences at such altitudes.  Figure~\ref{fig:fm13:O3:mr_avk} suggests that
the product is useful over the range 44--80~km with a vertical resolution of
around 6~km.


\begin{figure}[tbhp]
    \centering
    \begin{subfigure}[b]{0.49\textwidth}
        \includegraphics[width=\textwidth]{ALL13lowTunc_fm13_O3_perdiff_mipas}
        \caption{average difference to MIPAS}
        \label{fig:fm13:O3:profiles:MIPAS}
    \end{subfigure}
    \,
    \begin{subfigure}[b]{0.49\textwidth}
        \includegraphics[width=\textwidth]{ALL13lowTunc_fm13_O3_perdiff_mls}
        \caption{average difference to MLS}
        \label{fig:fm13:O3:profiles:MLS}
    \end{subfigure}

    \begin{subfigure}[b]{0.49\textwidth}
        \includegraphics[width=\textwidth]{ALL13lowTunc_fm13_O3_perdiff_osiris}
        \caption{average difference to OSIRIS}
        \label{fig:fm13:O3:profiles:OSIRIS}
    \end{subfigure}
    \caption{Average difference in percent between retrievals of \chem{O_3}
    from \smr~v3 and collocated measurements from various instruments at
    different altitudes for frequency mode~13.}

    \label{fig:fm13:O3:profiles}
\end{figure}

\begin{figure}[tbhp]
    \centering
    \begin{subfigure}[b]{0.49\textwidth}
        \includegraphics[width=\textwidth]{ALL13lowTunc_fm13_O3_scatter_v2}
        \caption{correlation of collcated instruments with \smr~v2.X}
        \label{fig:fm13:O3:scatter:v2}
    \end{subfigure}
    \,
    \begin{subfigure}[b]{0.49\textwidth}
        \includegraphics[width=\textwidth]{ALL13lowTunc_fm13_O3_scatter_v3}
        \caption{correlation of collcated instruments with \smr~v3}
        \label{fig:fm13:O3:scatter:v3}
    \end{subfigure}
    \caption{Correlation between retrievals of \chem{O_3} using \smr\
    versions~2.X and~3 and collocated measurements from various instruments
    for frequency mode~13.}
    \label{fig:fm13:O3:scatter}
\end{figure}

\begin{figure}[tbhp]
    \centering
    \begin{subfigure}[b]{0.49\textwidth}
        \includegraphics[width=\textwidth]{ALL13lowTunc_fm13_O3_mr}
        \caption{median measurement response with $1\sigma$ and $2\sigma$
        percentiles}
        \label{fig:fm13:O3:mr}
    \end{subfigure}
    \,
    \begin{subfigure}[b]{0.49\textwidth}
        \includegraphics[width=\textwidth]{ALL13lowTunc_fm13_O3_avk}
        \caption{median averaging kernels\newline~}
        \label{fig:fm13:O3:avk}
    \end{subfigure}
    \caption{Measurement response and averaging kernels for \chem{O_3}
    retrievals for \smr~v3 at different altitudes for frequency mode~13.}
    \label{fig:fm13:O3:mr_avk}
\end{figure}


%%%%%%%
% H2O %
%%%%%%%
\newpage
\subsubsection{\chem{H_2O}}
\label{sec:fm13:comparison:H2O}
The retrievals for \chem{H_2O} have been compared with data from the MIPAS
and MLS instruments. SAGE does not provide reliable water vapour profiles
above 33km~(\cite{VDS:2016}). Annual average differences to these instruments
are shown in Figure~\ref{fig:fm13:H2O:profiles}. In
Figure~\ref{fig:fm13:H2O:scatter} individual retrievals for the instruments
for the entire period are plotted against the retrievals from the new and old
versions of the \smr\ processing chain. The results show a slightly improved
overall coherency with the updated version of the processing compared to both
considered instruments, but a systematic under estimation of the
concentrations has been introduced, and the correlation, in particular with
MIPAS, remains poor. This would suggest that the MIPAS values are unreliable
for this data product. After consultation with KIT we have compared with
MIPAS middle atmosphere and upper atmosphere modes and found much better
agreement~(\cite{grieco2020b}). Figure~\ref{fig:fm13:H2O:mr_avk} suggests
that the product is useful over the range 44--110~km with a vertical
resolution of around 4.5~km.


\begin{figure}[tbhp]
    \centering
    \begin{subfigure}[b]{0.49\textwidth}
        \includegraphics[width=\textwidth]{ALL13lowTunc_fm13_H2O_perdiff_mipas}
        \caption{average difference to MIPAS}
        \label{fig:fm13:H2O:profiles:MIPAS}
    \end{subfigure}
    \,
    \begin{subfigure}[b]{0.49\textwidth}
        \includegraphics[width=\textwidth]{ALL13lowTunc_fm13_H2O_perdiff_mls}
        \caption{average difference to MLS}
        \label{fig:fm13:H2O:profiles:MLS}
    \end{subfigure}
    \caption{Average difference in percent between retrievals of \chem{H_2O}
    from \smr~v3 and collocated measurements from various instruments at
    different altitudes for frequency mode~13.}

    \label{fig:fm13:H2O:profiles}
\end{figure}

\begin{figure}[tbhp]
    \centering
    \begin{subfigure}[b]{0.49\textwidth}
        \includegraphics[width=\textwidth]{ALL13lowTunc_fm13_H2O_scatter_v2}
        \caption{correlation of collcated instruments with \smr~v2.X}
        \label{fig:fm13:H2O:scatter:v2}
    \end{subfigure}
    \,
    \begin{subfigure}[b]{0.49\textwidth}
        \includegraphics[width=\textwidth]{ALL13lowTunc_fm13_H2O_scatter_v3}
        \caption{correlation of collcated instruments with \smr~v3}
        \label{fig:fm13:H2O:scatter:v3}
    \end{subfigure}
    \caption{Correlation between retrievals of \chem{H_2O} using \smr\
    versions~2.X and~3 and collocated measurements from various instruments
    for frequency mode~13.}
    \label{fig:fm13:H2O:scatter}
\end{figure}

\begin{figure}[tbhp]
    \centering
    \begin{subfigure}[b]{0.49\textwidth}
        \includegraphics[width=\textwidth]{ALL13lowTunc_fm13_H2O_mr}
        \caption{median measurement response with $1\sigma$ and $2\sigma$
        percentiles}
        \label{fig:fm13:H2O:mr}
    \end{subfigure}
    \,
    \begin{subfigure}[b]{0.49\textwidth}
        \includegraphics[width=\textwidth]{ALL13lowTunc_fm13_H2O_avk}
        \caption{median averaging kernels\newline~}
        \label{fig:fm13:H2O:avk}
    \end{subfigure}
    \caption{Measurement response and averaging kernels for \chem{H_2O}
    retrievals for \smr~v3 at different altitudes for frequency mode~13.}
    \label{fig:fm13:H2O:mr_avk}
\end{figure}


%%%%%%%%%%%%%%%
% Temperature %
%%%%%%%%%%%%%%%
\newpage
\subsubsection{\chem{Temperature}}
\label{sec:fm13:comparison:temperature}
The retrievals for temperature have been compared with data from the MLS
instrument. Annual average differences to this instruments are shown in
Figure~\ref{fig:fm13:T:profiles}. In Figure~\ref{fig:fm13:T:scatter} individual
retrievals from MLS for 2003-2019 are plotted against the retrievals
from the new and old versions of the \smr\ processing chain. The results show
little or no improvement with the updated version of the processing. Whereas
with the previous version of the system, the temperature was systematically
over estimated, a small under estimation of the temperature is now seen. This
is a result of the removal of the arbitrary scaling factor.
Figure~\ref{fig:fm13:T:mr_avk} suggests that the product is useful over the
range 44--95~km with a vertical resolution of around 5.5~km.

\begin{figure}[tbhp]
    \centering
    \includegraphics[width=0.618\textwidth]{ALL13lowTunc_fm13_T_absdiff_mls}
    \caption{Average difference in K between retrievals of temperature from
    \smr~v3 and collocated measurements from MLS at different altitudes.}
    \label{fig:fm13:T:profiles}
        \label{fig:fm13:T:profiles:MLS}
\end{figure}

\begin{figure}[tbhp]
    \centering
    \begin{subfigure}[b]{0.49\textwidth}
        \includegraphics[width=\textwidth]{ALL13lowTunc_fm13_T_scatter_v2}
        \caption{correlation of collcated instruments with \smr~v2.X}
        \label{fig:fm13:T:scatter:v2}
    \end{subfigure}
    \,
    \begin{subfigure}[b]{0.49\textwidth}
        \includegraphics[width=\textwidth]{ALL13lowTunc_fm13_T_scatter_v3}
        \caption{correlation of collcated instruments with \smr~v3}
        \label{fig:fm13:T:scatter:v3}
    \end{subfigure}
    \caption{Correlation between retrievals of temperature using \smr\
    versions~2.X and~3 and collocated measurements from various instruments.}
    \label{fig:fm13:T:scatter}
\end{figure}

\begin{figure}[tbhp]
    \centering
    \begin{subfigure}[b]{0.49\textwidth}
        \includegraphics[width=\textwidth]{ALL13lowTunc_fm13_T_mr}
        \caption{median measurement response with $1\sigma$ and $2\sigma$
        percentiles}
        \label{fig:fm13:T:mr}
    \end{subfigure}
    \,
    \begin{subfigure}[b]{0.49\textwidth}
        \includegraphics[width=\textwidth]{ALL13lowTunc_fm13_T_avk}
        \caption{median averaging kernels\newline~}
        \label{fig:fm13:T:avk}
    \end{subfigure}
    \caption{Measurement response and averaging kernels for temperature
    retrievals for \smr~v3 at different altitudes for frequency mode~13.}
    \label{fig:fm13:T:mr_avk}
\end{figure}

\newpage
\subsection{Discussion}
\label{sec:fm13:discussion}
The Pearson correlation between the \smr\ retrievals and the other instruments
was calculated for the entire period for both versions of the processing chain.
The results are summarised in Table~\ref{tab:fm13:stats}, and show that the
new algorithm shows little or no improvement compared to all the instruments
for the species used in this investigation.


\begin{table}[tbhp]
\centering
\caption{Pearson correlation and fit parameters of the old and new \smr\
retrievals for frequency mode~13, compared with collocated data from other
instruments for the period 2003--2019.
}
\label{tab:fm13:stats}
\begin{tabular}{lllrrrr}
    \toprule
    \textbf{Species} & \textbf{Instrument} & \textbf{SMR} & \textbf{corr.} & \textbf{slope} & \textbf{intercept} & \textbf{$\left|\left<\right.\right.$res.$\left.\left.\right>\right|$} \\
    \midrule
    \chem{O3}   & MIPAS (KIT)   & v3    & 0.643 & 0.470 & 0.503\,ppm    & 0.801\,ppm \\
                &               & v2.x  & 0.750 & 0.886 & 0.128\,ppm    & 0.489\,ppm \\
    \cline{2-7}
                & MIPAS (ESA)   & v3    & 0.750 & 0.696 & 0.282\,ppm    & 0.560\,ppm \\
                &               & v2.x  & 0.788 & 0.905 & 0.201\,ppm    & 0.451\,ppm \\
    \cline{2-7}
                & MLS           & v3    & 0.436 & 0.262 &  0.658\,ppm   & 1.183\,ppm \\
                &               & v2.x  & 0.829 & 0.979 & -0.083\,ppm   & 0.500\,ppm \\
    \cline{2-7}
                & OSIRIS        & v3    & 0.845 & 0.845 & 0.176\,ppm    & 0.332\,ppm \\
                &               & v2.x  & 0.765 & 0.884 & 0.207\,ppm    & 0.372\,ppm \\
    \midrule
    \chem{H_2O} & MIPAS (KIT)   & v3    & 0.432 & 0.706 & -0.376\,ppm   & 3.254\,ppm \\
                &               & v2.x  & 0.445 & 0.823 & -0.187\,ppm   & 3.086\,ppm \\
    \cline{2-7}
                & MIPAS (ESA)   & v3    & 0.824 & 0.803 &  1.061\,ppm   & 1.821\,ppm \\
                &               & v2.x  & 0.787 & 0.842 &  1.508\,ppm   & 2.275\,ppm \\
    \cline{2-7}
                & MLS           & v3    & 0.923 & 0.935 & -0.362\,ppm   & 1.345\,ppm \\
                &               & v2.x  & 0.897 & 0.964 & -0.448\,ppm   & 1.534\,ppm \\
    \midrule
    Temp.       & MLS           & v3    & 0.968 & 1.029 & -2.859\,K     &  8.773\,K \\
                &               & v2.x  & 0.965 & 1.043 & -3.836\,K     & 11.260\,K \\
    \bottomrule
\end{tabular}
\end{table}

\subsection{Conclusions}
\label{sec:fm13:conclusions}
Based on the discussion above, retrievals based on frequency mode~13 should be
used with caution for both \chem{O_3} and \chem{H_2O}. The temperature
retrievals, on the other hand, appear reliable albeit with a cold bias of
3--5~K. For water vapour he comparisons with MLS seem to suggest a consistent
-20~\% bias.
