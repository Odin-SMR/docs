\chapter{\smr\ Level2 data products}

\section{The Odin mission}
\label{sec:odin}
%

The Odin satellite was launched on the 20th of February 2001, into a
sun-synchronous 18:00 hour ascending node orbit, carrying two co-aligned limb
sounding instruments: \OSIRIS\ (\OSIRISlong) and \SMR\ (\SMRlong)
\cite{murtagh:anove:02}.  Originally, Odin was used for both atmospheric and
astronomical observations, but since 2007 only its aeronomy mission is active.
Odin is a Swedish-led project, in cooperation with Canada, France and Finland.
Both of Odin's instruments are still functional, and the present operation of
the satellite is partly performed as an \ESA\ third party mission.

\section{The SMR instrument}
% The folloiwing text comes from the from ATBD-L2:

The \smr\ package is highly flexible \cite{frisk:theod:03}.  In short, the
four main receiver chains can be tuned to cover frequencies in the ranges
486--504\,GHz and 541--581\,GHz, but the maximum total instantaneous bandwidth
is only 1.6\,GHz. This bandwidth is determined by the two auto-correlation
spectrometers (ACs) used for atmospheric observations. The two ACs can be
coupled to any of the four front-ends, but only two or three front-ends are
used simultaneously. The ACs cover 400 or 800\,MHz per front-end, depending on
configuration. In the configuration applied for atmospheric sounding, the
channels of the ACs have a spacing of 1\,MHz, while the frequency resolution is
only 2\,MHz.  To cover all molecular transitions of interest (see
Table~\ref{table:level2} and Table~\ref{table:level2b} for an overview), a
number of ``observation modes'' have been defined. Each observation mode makes
use of two or three frequency bands. Single sideband operation is obtained by
tunable Martin--Pupplet interferometers. The nominal sideband suppression is
better than 19\,dB across the image band.

\smr\ also has a receiver chain around the 118\,GHz oxygen transition that was
heavily used during Odin's astronomy mission. For the atmospheric mission, this
front-end was planned to be used for retrieving temperature profiles, but a
technical problem (drifting LO frequency) and the fact that the analysis
requires treatment of Zeeman splitting have given these data low priority.

The main reflector of \smr\ has a diameter of 1.1\,m, giving a vertical
resolution at the tangent point of about 2\,km. The vertical scanning of the
two instruments' line-of-sight is achieved by a rotation of the satellite
platform, with a rate matching a vertical speed of the tangent altitude of
750\,m/s. Measurements are performed during both upward and downward scanning.
The lower end of the scan is typically at about 7\,km, the upper end varies
between 70 and 110\,km, depending on observation mode. In correspondence, the
horizontal sampling ranges from 1 scan per 600 km to 1 scan per 1000 km.
Measurements are in general performed along the orbit plane, providing a
latitude coverage between 82.5$^{\circ}$S and 82.5$^{\circ}$N. Since the end of
2004 Odin is also pointing off-track during certain periods, e.g.\ during the
austral summer season, allowing the latitudinal coverage to be extended towards
the poles.


\begin{table}
\caption{Characteristics of \smr\ Level2 main data products  for version~2 of
    the processing chain.}
\label{table:level2}
\scalebox{0.965}{%
\begin{tabular}{lrrrrr}
  \toprule
  \textbf{Product}      & \textbf{Frequency} & \textbf{Vertical}          & \textbf{Vertical}            & \textbf{Precision} & \textbf{Reference} \\
                        & \textbf{{[}GHz{]}} & \textbf{coverage}          & \textbf{resolution}          &                    &                 \\
  \midrule
  \chem{O_{3}}          & 501.5 (FM 01)             & \(\sim\)19--50\,km         &  \(\sim\)2\,km               & 0.5--2\,ppmv       &  \cite{urban:odins:05-b} \\
  \chem{ClO}            & 501.3   (FM 01)           & \(\sim\)19--67\,km         &  1.5--2\,km                  & 0.15--0.2\,ppbv    & \cite{urban:odins:05-b} \\
  \chem{N_{2}O}         & 502.3   (FM 01)           & \(\sim\)15--70\,km         &  \(\sim\)1.5\,km             & 15--35\,ppbv       & \cite{urban:odins:05-b} \\
  \midrule
  \chem{O_{3}}          & 544.9  (FM 02)            & \(\sim\)18--70\,km         &  \(\sim\)1.5\,km             & 0.2--0.4\,ppmv     & \cite{urban:odins:05-b} \\
  \chem{HNO_{3}}        & 544.4  (FM 02)            & \(\sim\)21--67\,km         &  1.5--2\,km                  & 1 ppbv             & \cite{urban:odins:05-b} \\
  \bottomrule
\end{tabular}}
\end{table}

\begin{table}
\caption{Characteristics of \smr\ Level2 science data products for version~2
    of the processing chain.}
\label{table:level2b}
\scalebox{0.9}{%
\begin{tabular}{lrrrrr}
  \toprule
  \textbf{Product}      & \textbf{Frequency} & \textbf{Vertical}          & \textbf{Vertical}            & \textbf{Precision} & \textbf{Reference} \\
                        & \textbf{{[}GHz{]}} & \textbf{coverage}          & \textbf{resolution}          &                    &                 \\
  \midrule
  \chem{CO}             & 578.6  (FM 22)            & \(\sim\)17--110\,km        &  3--4\,km                    & 25\,ppbv--2\,ppmv  & \cite{dupuy:strat:04} \\
  \chem{H_{2}\,^{16}O}  & 556.9   (FM 19)           & \(\sim\)40--100\,km        &  \(\sim\)3\,km               & 0.5--1\,ppmv       & \cite{urban:globa:07} \\
  \midrule
  \chem{H_{2}\,^{16}O}  & 488.5  (FM 17)            & \(\sim\)20--70\,km         &  \(\sim\)3\,km               & 0.5--1\,ppmv       & \cite{urban:globa:07} \\
  \chem{HDO}            & 490.6   (FM  17)          & \(\sim\)20--70\,km         &  3--4\,km                    & 0.5\,ppbv          & \cite{urban:globa:07} \\
  \chem{H_{2}\,^{18}O}  & 489.1   (FM 08)           & \(\sim\)20--65\,km         &  3--4\,km                    & 20-30\,ppbv        & \cite{urban:globa:07} \\
  \midrule
  \chem{H_{2}\,^{17}O}  & 552.0  (FM 21)            & \(\sim\)20--70\,km         & \(\sim\)3\,km                & 0.4\,ppbv          & \cite{urban:globa:07} \\
  \chem{NO}             & 551.7    (FM 21)          & \(\sim\)40--100\,km        & \(\sim\)7\,km               & 40\,\(\%\)          & \cite{sheese:odino:2013} \\
  \midrule
  \chem{^{16}O\,^{18}O\,^{16}O}  & 490.4 (FM 17)     & \(\sim\)27--41\,km         & 4--6\,km                     & 25\,\(\%\)         & \cite{urban:globa:13} \\
  \chem{^{16}O\,^{16}O\,^{18}O}  & 490.0  (FM 17)   & \(\sim\)25--45\,km         & 3--4\,km                     & 25\,\(\%\)         & \cite{urban:globa:13} \\
  \chem{^{16}O\,^{16}O\,^{17}O}  & 490.6  (FM 17)   & \(\sim\)31--39\,km         & 5--6\,km                     & 25\,\(\%\)         & \cite{urban:globa:13}  \\
  \bottomrule
\end{tabular}}
\end{table}


\section{\smr\ Level2 data products}

\smr\ data are categorized into main and science Level2 products, and
Table~\ref{table:level2} and Table~\ref{table:level2b} describe the
characteristics of these products, respectively.  The main products are
retrieved from the so called ``stratospheric'' observation mode of \smr, and
this mode cover approximately 50\,\(\%\) of the \smr\ observation time.  In
this mode spectra in frequency bands around 501 and 544\,GHz are collected.
The science data products are derived from less frequently applied observation
modes (typically applied a few days per month).


\subsection{Main data products}

Ozone, \chem{ClO}, \chem{N_{2}O}, and \chem{HNO_{3}} profiles are the main
\smr\ Level2 products.  \chem{ClO} and \chem{N_{2}O} profiles are retrieved
from spectra covering transitions around 501\,GHz, and \chem{HNO_{3}} from
spectra around 544\,GHz.  Ozone can be retrieved from both the 501 and the
544\,GHz band.  Table~\ref{table:level2} describes characteristics of these
Level2 products that have been derived from earlier \smr\ Level2 data studies.
The characteristics can not be expected to be changed/improved dramatically for
a new Level2 data product, because these characteristics depend on the physics
of the measurement and the sensor.

Possibly more important than the characteristics described in
Table~\ref{table:level2} are the accuracy and stability of the profiles, since
the latter enable trend studies.  The overall aim of the new Level2 data
processing also reflects this aspect, and the objective is therefore that the
accuracy and stability outperforms that from earlier \smr\ Level2 data
products.

\subsection{Science data products}

Profiles of \chem{H_{2}O}, \chem{CO}, \chem{NO} and isotopologues of
\chem{H_{2}O}, and \chem{O_{3}} are considered as science data products for
\smr, and characteristics of these products are described in
Table~\ref{table:level2b}.  Observations covering the science data products are
performed on a less frequent basis than the main data products. The aim of the
Level2 processing of the science data products is in principle identical to
that for the main data products, although the main data products will be given
a higher priority.
