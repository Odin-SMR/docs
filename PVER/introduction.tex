\chapter{Introduction}
\label{chapter:introduction}


% The page numbering must be reset here inside the file
\pagenumbering{arabic}
\setcounter{page}{1}


\section{Aim and scope of this document}
\label{sec:aim}
\smr\ performs passive limb measurements of the atmosphere, mainly at
wavelengths and frequencies around 0.6\,mm and 500\,GHz, respectively.  From
these measurements, profiles of \chem{O_3}, \chem{ClO}, \chem{N_{2}O},
\chem{HNO_{3}}, \chem{H_{2}O}, \chem{CO}, and isotopologues of \chem{H_{2}O},
and \chem{O_{3}}, that are species of interest for studying
stratospheric and mesospheric chemistry and dynamics, can be derived.  \smr\
has been in operation for approximately 20 years, and thus, the Level2 dataset
can potentially be applied for scientifically interesting trend analysis.

The project aims at producing a  new \smr\ Level2 product dataset with improved
long term consistency and precision.   This dataset will be
based on updated/revised processing algorithms and input data.  A verification
dataset (VDS) has been created. The VDS is a representative subset of the \smr\
Level1B dataset and collocated correlative measurements from similar
instruments, i.e.  data from Odin/OSIRIS, Aura/MLS, ENVISAT/MIPAS,
ISS/JEM/SMILES, and Meteor3M/SAGEIII. See~\cite{VDS:2016} for descriptions of
data and instruments included in the VDS.

From the VDS a diagnostic dataset (DDS) has been created, containing Level2
products produced with the updated processing system. The aim of this document
is to describe the DDS and the main results of comparing it with the
correlative measurements from other instruments, as well as with the Level2
data from the older 2.0/2.1 versions of the \smr\ processing chain.

\section{Document structure}
This document is organized as follows: Chapter~2 describes the \smr\ Level2
data products. Chapter~3 contains comparisons for several of the \smr\
frequency modes and their Level2 data products with collocated measurements from
various instruments. Chapter~4 contains conlsusions, and the appendix contains
additional information on the various observation modes of \smr.
