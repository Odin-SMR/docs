\chapter{Introduction}
\label{chapter:introduction}


% The page numbering must be reset here inside the file
\pagenumbering{arabic}
\setcounter{page}{1}


\section{Aim and scope of this document}
\label{sec:aim}

\smr\ has been in operation since 2001 and performs passive
limb measurements of the atmosphere, mainly at frequencies around 500\,GHz.
From these measurements, profiles of species that are of interest for studying
stratospheric and mesospheric chemistry and dynamics can be derived, such as
\chem{O_3}, \chem{ClO}, \chem{N_{2}O}, \chem{HNO_{3}}, \chem{H_{2}O},
\chem{CO}, and isotopologues of \chem{H_{2}O}, and \chem{O_{3}}. These profiles
are denoted as the Level2 products of \smr. The Level2 products are generated
by a Level2 processor. The input to the Level2 processor is geolocated and
calibrated measurements (\smr\ Level1B data, described in \citet{atbdl1b}) and
dynamic and static auxilliary/ancilliary data.

The aim of this Input/Output Data Definition Document (IODD) is to describe
the input and output data used and generated by the Level2 processor.
The details of the algorithms applied by the Level2 processor
is described in an Algorithms Theoretical Basis Document (ATBD) - 
Level 2 processing \citep{atbdl2}, and are not covered by this document.
The Level2 processor is a component of a Level2 processing system,
and the aim is also to describe this processing system.

\section{Document structure}

This document is organized as follows:
Chapter 2 describes the design of the \smr\ Level2 processing system. 
Chapter 3 describes the input data used and the output data generated 
by the \smr\ Level2 processor.


