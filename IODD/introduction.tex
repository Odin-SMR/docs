\chapter{Introduction}
\label{chapter:introduction}


% The page numbering must be reset here inside the file
\pagenumbering{arabic}
\setcounter{page}{1}


\section{Aim and scope of this document}
\label{sec:aim}
\lcomment{BR}{Intro copied from VDS: revise late:\\
\smr\ performs passive limb measurements of the atmosphere,
mainly at wavelengths and frequencies around 0.6\,mm and 500\,GHz,
respectively.
From these measurements, profiles of 
\chem{O_3}, \chem{ClO}, \chem{N_{2}O}, \chem{HNO_{3}}, 
\chem{H_{2}O}, \chem{CO}, and isotopologues of \chem{H_{2}O}, and \chem{O_{3}},
that are species that are of interest for studying stratospheric and 
mesospheric chemistry and dynamics, can be derived. 
\smr\ has been in operation for approximately 14 years, and thus, the Level2
dataset can potentially be applied for scientifically interesting trend analysis.

A new \smr\ Level2 product dataset will be generated, and this dataset will be based
on updated/revised processing algorithms and input data.
A verification dataset (VDS) will be used as a tool to verify the new 
processing system/Level2 products.
}

The aim of this Input/Output Data Definition Document (IODD) is to describe
the design of the \smr\ Level2 processing system, including the dependencies and 
format of input and output data used and generated by 
each algorithm in the processing system.

\section{Document structure}

This document is organized as follows:
Chapter 2 describes the design of the \smr\ Level2 processing system,
including the depencies between the algorithms involved. 
Chapter 3 describes the input data used and generated 
by the \smr\ Level2 processor.


